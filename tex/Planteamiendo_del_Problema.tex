\chapter{Planteamiento del Problema}

El grafeno es un material bidimensional con una estructura electrónica única que, durante los últimos diez años, ha sido la base de muchos fenómenos nuevos diversos. Muchos de estos fenómenos se basan en la naturaleza especial de los portadores de grafeno, que se comportan como fermiones de Dirac sin masa con dos quiralidades diferentes, cada una de ellas relacionada con dos regiones especiales de la zona de Brillouin, definidas como valles de Dirac o conos de Dirac porque están ubicados en el mínimo de la banda de conducción \autocite{CastroNeto2009}. 

El uso de valles para codificar y procesar información se remonta al silicio, que también presenta valles, pero se intensificó en el grafeno debido a la naturaleza relativista de los portadores de carga \autocite{Tsuneya2015}. Recientemente, con los avances del proceso de fabricación de grafeno, la valletrónica ha ganado mucha atención \autocite{CastroNeto2011, Cresti2016, MarmolejoTejada2018}. La posibilidad de manipular los valles usando super-redes o la capacidad de su impacto en las propiedades de transporte de espín han intensificado la necesidad de caracterizar la dispersión intervalle \autocite{Cummings2017, Benitez2017}. 