\section[Sobre la convergencia de la aproximación]{Sobre la convergencia numérica de la exponencial imaginaria con dominio restringido.}

Para estudiar la velocidad de convergencia de la serie truncada \eqref{eq:iexpChebExpan} en el dominio $ x \in (-1, 1) $, primero es útil analizar 
individualmente cada uno de los factores de los términos de la sumatoria, principalmente los polinomios de Chebyshev y las
funciones de Bessel de primera especie.

Para los polinomios de Chebyshev se tiene que 
\begin{equation*}
	\abs{T_n(x)} \leq 1 \quad \forall x \in [-1, 1] \land n \geq 0,
\end{equation*}
de modo que no aportan nada concreto al estudio de la velocidad a la que converge la serie y se puede descartar.

Para las funciones de Bessel de primera especie, se tiene que para valores de orden $n$ enteros positivos mucho mayores
al argumento $z$, las funciones de Bessel se aproximan como
\begin{align*}
	J_n(z) &\sim \frac1{n!}\left( \frac{z}{2} \right)^n,\\
	\intertext{donde para gran $n$ el factorial se puede aproximar mediante la aproximación de Stirling $n! \sim \sqrt{2\pi n}\left( \frac{n}{e}\right)^n$}
	J_n(z) &\sim (2\pi n)^{-\frac12}\left( \frac{ez}{2n} \right)^n,
\end{align*}
la cuál converge rápidamente para $2n > ez$.

Debido a que la velocidad de convergencia está ligada al parámetro $z$, es útil analizar específicamente para qué casos
se aproximará la exponencial. Para la investigación corriente la exponencial se desea aproximar para el operador de
evolución temporal cuántico
\begin{equation}\label{eq:timeEvOp}
	\hat{U}(t) = \exp({\frac{i\hat{H}t}\hbar}),
\end{equation}
para aproximarlo (mediante polinomios de Chebyshev) en función del hamiltoniano, hace falta reescalarlo para que sus 
autovalores se encuentren entre -1 y 1, es decir
\begin{equation}\label{eq:reescaledHam}
	\tilde{H} = \frac{\hat{H} - \bar{E}}{\Delta E},
\end{equation}
donde $\bar E = \frac{E_\mathrm{max} + E_\mathrm{min}}2$ es el centro de banda y 
$\Delta E = \frac{E_\mathrm{max} - E_\mathrm{min}}2$ es el ancho de banda.

Sustituyendo el hamiltoniano reescalado \eqref{eq:reescaledHam} en el operador de evolución temporal \eqref{eq:timeEvOp}
\begin{align*}
	\hat{U}(t) &= \exp({\frac{i\Delta E\tilde{H}t}\hbar}) \exp(\frac{i\bar E t}\hbar) \\
	\hat{U}(t) &= \exp({\frac{i\tilde{H}t}\tau}) \exp(\frac{i\bar E t}\hbar)\\
	\hat{U}(t) &= \exp(i\tilde{H}\tilde t) \exp(\frac{i\bar E t}\hbar),
\end{align*}
donde $\tau = \frac\hbar{\Delta E}$ tiene unidades en el order de 
$\tau \sim \frac{0.6 \electronvolt\femto\second}{\electronvolt} = 0.6\femto\second$ 
por lo cuál la escala de $t$ está en el orden de los \femto\second. El fenómeno a estudiar 
(el tiempo de relajación en transporte eléctrico) está en el orden de los 
\femto\second{} a los \pico\second{} \autocite{Fan2018}, de modo que el cociente 
$\frac{t}{\tau} = \tilde{t} \in [10^0, 10^4)$.

\iffalse
\begin{figure}[bth]
	\centering
	%% Creator: Matplotlib, PGF backend
%%
%% To include the figure in your LaTeX document, write
%%   \input{<filename>.pgf}
%%
%% Make sure the required packages are loaded in your preamble
%%   \usepackage{pgf}
%%
%% Figures using additional raster images can only be included by \input if
%% they are in the same directory as the main LaTeX file. For loading figures
%% from other directories you can use the `import` package
%%   \usepackage{import}
%% and then include the figures with
%%   \import{<path to file>}{<filename>.pgf}
%%
%% Matplotlib used the following preamble
%%   \usepackage[utf8x]{inputenc}
%%   \usepackage[T1]{fontenc}
%%   \usepackage{fontspec}
%%
\begingroup%
\makeatletter%
\begin{pgfpicture}%
\pgfpathrectangle{\pgfpointorigin}{\pgfqpoint{5.500000in}{3.399187in}}%
\pgfusepath{use as bounding box, clip}%
\begin{pgfscope}%
\pgfsetbuttcap%
\pgfsetmiterjoin%
\definecolor{currentfill}{rgb}{1.000000,1.000000,1.000000}%
\pgfsetfillcolor{currentfill}%
\pgfsetlinewidth{0.000000pt}%
\definecolor{currentstroke}{rgb}{1.000000,1.000000,1.000000}%
\pgfsetstrokecolor{currentstroke}%
\pgfsetdash{}{0pt}%
\pgfpathmoveto{\pgfqpoint{0.000000in}{0.000000in}}%
\pgfpathlineto{\pgfqpoint{5.500000in}{0.000000in}}%
\pgfpathlineto{\pgfqpoint{5.500000in}{3.399187in}}%
\pgfpathlineto{\pgfqpoint{0.000000in}{3.399187in}}%
\pgfpathclose%
\pgfusepath{fill}%
\end{pgfscope}%
\begin{pgfscope}%
\pgfsetbuttcap%
\pgfsetmiterjoin%
\definecolor{currentfill}{rgb}{1.000000,1.000000,1.000000}%
\pgfsetfillcolor{currentfill}%
\pgfsetlinewidth{0.000000pt}%
\definecolor{currentstroke}{rgb}{0.000000,0.000000,0.000000}%
\pgfsetstrokecolor{currentstroke}%
\pgfsetstrokeopacity{0.000000}%
\pgfsetdash{}{0pt}%
\pgfpathmoveto{\pgfqpoint{0.687500in}{0.373911in}}%
\pgfpathlineto{\pgfqpoint{4.950000in}{0.373911in}}%
\pgfpathlineto{\pgfqpoint{4.950000in}{2.991285in}}%
\pgfpathlineto{\pgfqpoint{0.687500in}{2.991285in}}%
\pgfpathclose%
\pgfusepath{fill}%
\end{pgfscope}%
\begin{pgfscope}%
\pgfsetbuttcap%
\pgfsetroundjoin%
\definecolor{currentfill}{rgb}{0.000000,0.000000,0.000000}%
\pgfsetfillcolor{currentfill}%
\pgfsetlinewidth{0.803000pt}%
\definecolor{currentstroke}{rgb}{0.000000,0.000000,0.000000}%
\pgfsetstrokecolor{currentstroke}%
\pgfsetdash{}{0pt}%
\pgfsys@defobject{currentmarker}{\pgfqpoint{0.000000in}{-0.048611in}}{\pgfqpoint{0.000000in}{0.000000in}}{%
\pgfpathmoveto{\pgfqpoint{0.000000in}{0.000000in}}%
\pgfpathlineto{\pgfqpoint{0.000000in}{-0.048611in}}%
\pgfusepath{stroke,fill}%
}%
\begin{pgfscope}%
\pgfsys@transformshift{0.881250in}{0.373911in}%
\pgfsys@useobject{currentmarker}{}%
\end{pgfscope}%
\end{pgfscope}%
\begin{pgfscope}%
\definecolor{textcolor}{rgb}{0.000000,0.000000,0.000000}%
\pgfsetstrokecolor{textcolor}%
\pgfsetfillcolor{textcolor}%
\pgftext[x=0.881250in,y=0.276688in,,top]{\color{textcolor}\rmfamily\fontsize{10.000000}{12.000000}\selectfont \(\displaystyle 0\)}%
\end{pgfscope}%
\begin{pgfscope}%
\pgfsetbuttcap%
\pgfsetroundjoin%
\definecolor{currentfill}{rgb}{0.000000,0.000000,0.000000}%
\pgfsetfillcolor{currentfill}%
\pgfsetlinewidth{0.803000pt}%
\definecolor{currentstroke}{rgb}{0.000000,0.000000,0.000000}%
\pgfsetstrokecolor{currentstroke}%
\pgfsetdash{}{0pt}%
\pgfsys@defobject{currentmarker}{\pgfqpoint{0.000000in}{-0.048611in}}{\pgfqpoint{0.000000in}{0.000000in}}{%
\pgfpathmoveto{\pgfqpoint{0.000000in}{0.000000in}}%
\pgfpathlineto{\pgfqpoint{0.000000in}{-0.048611in}}%
\pgfusepath{stroke,fill}%
}%
\begin{pgfscope}%
\pgfsys@transformshift{1.443207in}{0.373911in}%
\pgfsys@useobject{currentmarker}{}%
\end{pgfscope}%
\end{pgfscope}%
\begin{pgfscope}%
\definecolor{textcolor}{rgb}{0.000000,0.000000,0.000000}%
\pgfsetstrokecolor{textcolor}%
\pgfsetfillcolor{textcolor}%
\pgftext[x=1.443207in,y=0.276688in,,top]{\color{textcolor}\rmfamily\fontsize{10.000000}{12.000000}\selectfont \(\displaystyle 10^{0}\)}%
\end{pgfscope}%
\begin{pgfscope}%
\pgfsetbuttcap%
\pgfsetroundjoin%
\definecolor{currentfill}{rgb}{0.000000,0.000000,0.000000}%
\pgfsetfillcolor{currentfill}%
\pgfsetlinewidth{0.803000pt}%
\definecolor{currentstroke}{rgb}{0.000000,0.000000,0.000000}%
\pgfsetstrokecolor{currentstroke}%
\pgfsetdash{}{0pt}%
\pgfsys@defobject{currentmarker}{\pgfqpoint{0.000000in}{-0.048611in}}{\pgfqpoint{0.000000in}{0.000000in}}{%
\pgfpathmoveto{\pgfqpoint{0.000000in}{0.000000in}}%
\pgfpathlineto{\pgfqpoint{0.000000in}{-0.048611in}}%
\pgfusepath{stroke,fill}%
}%
\begin{pgfscope}%
\pgfsys@transformshift{2.712189in}{0.373911in}%
\pgfsys@useobject{currentmarker}{}%
\end{pgfscope}%
\end{pgfscope}%
\begin{pgfscope}%
\definecolor{textcolor}{rgb}{0.000000,0.000000,0.000000}%
\pgfsetstrokecolor{textcolor}%
\pgfsetfillcolor{textcolor}%
\pgftext[x=2.712189in,y=0.276688in,,top]{\color{textcolor}\rmfamily\fontsize{10.000000}{12.000000}\selectfont \(\displaystyle 10^{1}\)}%
\end{pgfscope}%
\begin{pgfscope}%
\pgfsetbuttcap%
\pgfsetroundjoin%
\definecolor{currentfill}{rgb}{0.000000,0.000000,0.000000}%
\pgfsetfillcolor{currentfill}%
\pgfsetlinewidth{0.803000pt}%
\definecolor{currentstroke}{rgb}{0.000000,0.000000,0.000000}%
\pgfsetstrokecolor{currentstroke}%
\pgfsetdash{}{0pt}%
\pgfsys@defobject{currentmarker}{\pgfqpoint{0.000000in}{-0.048611in}}{\pgfqpoint{0.000000in}{0.000000in}}{%
\pgfpathmoveto{\pgfqpoint{0.000000in}{0.000000in}}%
\pgfpathlineto{\pgfqpoint{0.000000in}{-0.048611in}}%
\pgfusepath{stroke,fill}%
}%
\begin{pgfscope}%
\pgfsys@transformshift{3.723712in}{0.373911in}%
\pgfsys@useobject{currentmarker}{}%
\end{pgfscope}%
\end{pgfscope}%
\begin{pgfscope}%
\definecolor{textcolor}{rgb}{0.000000,0.000000,0.000000}%
\pgfsetstrokecolor{textcolor}%
\pgfsetfillcolor{textcolor}%
\pgftext[x=3.723712in,y=0.276688in,,top]{\color{textcolor}\rmfamily\fontsize{10.000000}{12.000000}\selectfont \(\displaystyle 10^{2}\)}%
\end{pgfscope}%
\begin{pgfscope}%
\pgfsetbuttcap%
\pgfsetroundjoin%
\definecolor{currentfill}{rgb}{0.000000,0.000000,0.000000}%
\pgfsetfillcolor{currentfill}%
\pgfsetlinewidth{0.803000pt}%
\definecolor{currentstroke}{rgb}{0.000000,0.000000,0.000000}%
\pgfsetstrokecolor{currentstroke}%
\pgfsetdash{}{0pt}%
\pgfsys@defobject{currentmarker}{\pgfqpoint{0.000000in}{-0.048611in}}{\pgfqpoint{0.000000in}{0.000000in}}{%
\pgfpathmoveto{\pgfqpoint{0.000000in}{0.000000in}}%
\pgfpathlineto{\pgfqpoint{0.000000in}{-0.048611in}}%
\pgfusepath{stroke,fill}%
}%
\begin{pgfscope}%
\pgfsys@transformshift{4.735235in}{0.373911in}%
\pgfsys@useobject{currentmarker}{}%
\end{pgfscope}%
\end{pgfscope}%
\begin{pgfscope}%
\definecolor{textcolor}{rgb}{0.000000,0.000000,0.000000}%
\pgfsetstrokecolor{textcolor}%
\pgfsetfillcolor{textcolor}%
\pgftext[x=4.735235in,y=0.276688in,,top]{\color{textcolor}\rmfamily\fontsize{10.000000}{12.000000}\selectfont \(\displaystyle 10^{3}\)}%
\end{pgfscope}%
\begin{pgfscope}%
\pgfsetbuttcap%
\pgfsetroundjoin%
\definecolor{currentfill}{rgb}{0.000000,0.000000,0.000000}%
\pgfsetfillcolor{currentfill}%
\pgfsetlinewidth{0.602250pt}%
\definecolor{currentstroke}{rgb}{0.000000,0.000000,0.000000}%
\pgfsetstrokecolor{currentstroke}%
\pgfsetdash{}{0pt}%
\pgfsys@defobject{currentmarker}{\pgfqpoint{0.000000in}{-0.027778in}}{\pgfqpoint{0.000000in}{0.000000in}}{%
\pgfpathmoveto{\pgfqpoint{0.000000in}{0.000000in}}%
\pgfpathlineto{\pgfqpoint{0.000000in}{-0.027778in}}%
\pgfusepath{stroke,fill}%
}%
\begin{pgfscope}%
\pgfsys@transformshift{0.881250in}{0.373911in}%
\pgfsys@useobject{currentmarker}{}%
\end{pgfscope}%
\end{pgfscope}%
\begin{pgfscope}%
\pgfsetbuttcap%
\pgfsetroundjoin%
\definecolor{currentfill}{rgb}{0.000000,0.000000,0.000000}%
\pgfsetfillcolor{currentfill}%
\pgfsetlinewidth{0.602250pt}%
\definecolor{currentstroke}{rgb}{0.000000,0.000000,0.000000}%
\pgfsetstrokecolor{currentstroke}%
\pgfsetdash{}{0pt}%
\pgfsys@defobject{currentmarker}{\pgfqpoint{0.000000in}{-0.027778in}}{\pgfqpoint{0.000000in}{0.000000in}}{%
\pgfpathmoveto{\pgfqpoint{0.000000in}{0.000000in}}%
\pgfpathlineto{\pgfqpoint{0.000000in}{-0.027778in}}%
\pgfusepath{stroke,fill}%
}%
\begin{pgfscope}%
\pgfsys@transformshift{1.443207in}{0.373911in}%
\pgfsys@useobject{currentmarker}{}%
\end{pgfscope}%
\end{pgfscope}%
\begin{pgfscope}%
\pgfsetbuttcap%
\pgfsetroundjoin%
\definecolor{currentfill}{rgb}{0.000000,0.000000,0.000000}%
\pgfsetfillcolor{currentfill}%
\pgfsetlinewidth{0.602250pt}%
\definecolor{currentstroke}{rgb}{0.000000,0.000000,0.000000}%
\pgfsetstrokecolor{currentstroke}%
\pgfsetdash{}{0pt}%
\pgfsys@defobject{currentmarker}{\pgfqpoint{0.000000in}{-0.027778in}}{\pgfqpoint{0.000000in}{0.000000in}}{%
\pgfpathmoveto{\pgfqpoint{0.000000in}{0.000000in}}%
\pgfpathlineto{\pgfqpoint{0.000000in}{-0.027778in}}%
\pgfusepath{stroke,fill}%
}%
\begin{pgfscope}%
\pgfsys@transformshift{2.712189in}{0.373911in}%
\pgfsys@useobject{currentmarker}{}%
\end{pgfscope}%
\end{pgfscope}%
\begin{pgfscope}%
\pgfsetbuttcap%
\pgfsetroundjoin%
\definecolor{currentfill}{rgb}{0.000000,0.000000,0.000000}%
\pgfsetfillcolor{currentfill}%
\pgfsetlinewidth{0.602250pt}%
\definecolor{currentstroke}{rgb}{0.000000,0.000000,0.000000}%
\pgfsetstrokecolor{currentstroke}%
\pgfsetdash{}{0pt}%
\pgfsys@defobject{currentmarker}{\pgfqpoint{0.000000in}{-0.027778in}}{\pgfqpoint{0.000000in}{0.000000in}}{%
\pgfpathmoveto{\pgfqpoint{0.000000in}{0.000000in}}%
\pgfpathlineto{\pgfqpoint{0.000000in}{-0.027778in}}%
\pgfusepath{stroke,fill}%
}%
\begin{pgfscope}%
\pgfsys@transformshift{3.723712in}{0.373911in}%
\pgfsys@useobject{currentmarker}{}%
\end{pgfscope}%
\end{pgfscope}%
\begin{pgfscope}%
\pgfsetbuttcap%
\pgfsetroundjoin%
\definecolor{currentfill}{rgb}{0.000000,0.000000,0.000000}%
\pgfsetfillcolor{currentfill}%
\pgfsetlinewidth{0.602250pt}%
\definecolor{currentstroke}{rgb}{0.000000,0.000000,0.000000}%
\pgfsetstrokecolor{currentstroke}%
\pgfsetdash{}{0pt}%
\pgfsys@defobject{currentmarker}{\pgfqpoint{0.000000in}{-0.027778in}}{\pgfqpoint{0.000000in}{0.000000in}}{%
\pgfpathmoveto{\pgfqpoint{0.000000in}{0.000000in}}%
\pgfpathlineto{\pgfqpoint{0.000000in}{-0.027778in}}%
\pgfusepath{stroke,fill}%
}%
\begin{pgfscope}%
\pgfsys@transformshift{4.735235in}{0.373911in}%
\pgfsys@useobject{currentmarker}{}%
\end{pgfscope}%
\end{pgfscope}%
\begin{pgfscope}%
\definecolor{textcolor}{rgb}{0.000000,0.000000,0.000000}%
\pgfsetstrokecolor{textcolor}%
\pgfsetfillcolor{textcolor}%
\pgftext[x=2.818750in,y=0.097800in,,top]{\color{textcolor}\rmfamily\fontsize{10.000000}{12.000000}\selectfont \(\displaystyle n\)}%
\end{pgfscope}%
\begin{pgfscope}%
\pgfsetbuttcap%
\pgfsetroundjoin%
\definecolor{currentfill}{rgb}{0.000000,0.000000,0.000000}%
\pgfsetfillcolor{currentfill}%
\pgfsetlinewidth{0.803000pt}%
\definecolor{currentstroke}{rgb}{0.000000,0.000000,0.000000}%
\pgfsetstrokecolor{currentstroke}%
\pgfsetdash{}{0pt}%
\pgfsys@defobject{currentmarker}{\pgfqpoint{-0.048611in}{0.000000in}}{\pgfqpoint{0.000000in}{0.000000in}}{%
\pgfpathmoveto{\pgfqpoint{0.000000in}{0.000000in}}%
\pgfpathlineto{\pgfqpoint{-0.048611in}{0.000000in}}%
\pgfusepath{stroke,fill}%
}%
\begin{pgfscope}%
\pgfsys@transformshift{0.687500in}{0.600980in}%
\pgfsys@useobject{currentmarker}{}%
\end{pgfscope}%
\end{pgfscope}%
\begin{pgfscope}%
\definecolor{textcolor}{rgb}{0.000000,0.000000,0.000000}%
\pgfsetstrokecolor{textcolor}%
\pgfsetfillcolor{textcolor}%
\pgftext[x=0.304783in,y=0.552785in,left,base]{\color{textcolor}\rmfamily\fontsize{10.000000}{12.000000}\selectfont \(\displaystyle -0.2\)}%
\end{pgfscope}%
\begin{pgfscope}%
\pgfsetbuttcap%
\pgfsetroundjoin%
\definecolor{currentfill}{rgb}{0.000000,0.000000,0.000000}%
\pgfsetfillcolor{currentfill}%
\pgfsetlinewidth{0.803000pt}%
\definecolor{currentstroke}{rgb}{0.000000,0.000000,0.000000}%
\pgfsetstrokecolor{currentstroke}%
\pgfsetdash{}{0pt}%
\pgfsys@defobject{currentmarker}{\pgfqpoint{-0.048611in}{0.000000in}}{\pgfqpoint{0.000000in}{0.000000in}}{%
\pgfpathmoveto{\pgfqpoint{0.000000in}{0.000000in}}%
\pgfpathlineto{\pgfqpoint{-0.048611in}{0.000000in}}%
\pgfusepath{stroke,fill}%
}%
\begin{pgfscope}%
\pgfsys@transformshift{0.687500in}{1.071626in}%
\pgfsys@useobject{currentmarker}{}%
\end{pgfscope}%
\end{pgfscope}%
\begin{pgfscope}%
\definecolor{textcolor}{rgb}{0.000000,0.000000,0.000000}%
\pgfsetstrokecolor{textcolor}%
\pgfsetfillcolor{textcolor}%
\pgftext[x=0.412808in,y=1.023431in,left,base]{\color{textcolor}\rmfamily\fontsize{10.000000}{12.000000}\selectfont \(\displaystyle 0.0\)}%
\end{pgfscope}%
\begin{pgfscope}%
\pgfsetbuttcap%
\pgfsetroundjoin%
\definecolor{currentfill}{rgb}{0.000000,0.000000,0.000000}%
\pgfsetfillcolor{currentfill}%
\pgfsetlinewidth{0.803000pt}%
\definecolor{currentstroke}{rgb}{0.000000,0.000000,0.000000}%
\pgfsetstrokecolor{currentstroke}%
\pgfsetdash{}{0pt}%
\pgfsys@defobject{currentmarker}{\pgfqpoint{-0.048611in}{0.000000in}}{\pgfqpoint{0.000000in}{0.000000in}}{%
\pgfpathmoveto{\pgfqpoint{0.000000in}{0.000000in}}%
\pgfpathlineto{\pgfqpoint{-0.048611in}{0.000000in}}%
\pgfusepath{stroke,fill}%
}%
\begin{pgfscope}%
\pgfsys@transformshift{0.687500in}{1.542272in}%
\pgfsys@useobject{currentmarker}{}%
\end{pgfscope}%
\end{pgfscope}%
\begin{pgfscope}%
\definecolor{textcolor}{rgb}{0.000000,0.000000,0.000000}%
\pgfsetstrokecolor{textcolor}%
\pgfsetfillcolor{textcolor}%
\pgftext[x=0.412808in,y=1.494078in,left,base]{\color{textcolor}\rmfamily\fontsize{10.000000}{12.000000}\selectfont \(\displaystyle 0.2\)}%
\end{pgfscope}%
\begin{pgfscope}%
\pgfsetbuttcap%
\pgfsetroundjoin%
\definecolor{currentfill}{rgb}{0.000000,0.000000,0.000000}%
\pgfsetfillcolor{currentfill}%
\pgfsetlinewidth{0.803000pt}%
\definecolor{currentstroke}{rgb}{0.000000,0.000000,0.000000}%
\pgfsetstrokecolor{currentstroke}%
\pgfsetdash{}{0pt}%
\pgfsys@defobject{currentmarker}{\pgfqpoint{-0.048611in}{0.000000in}}{\pgfqpoint{0.000000in}{0.000000in}}{%
\pgfpathmoveto{\pgfqpoint{0.000000in}{0.000000in}}%
\pgfpathlineto{\pgfqpoint{-0.048611in}{0.000000in}}%
\pgfusepath{stroke,fill}%
}%
\begin{pgfscope}%
\pgfsys@transformshift{0.687500in}{2.012918in}%
\pgfsys@useobject{currentmarker}{}%
\end{pgfscope}%
\end{pgfscope}%
\begin{pgfscope}%
\definecolor{textcolor}{rgb}{0.000000,0.000000,0.000000}%
\pgfsetstrokecolor{textcolor}%
\pgfsetfillcolor{textcolor}%
\pgftext[x=0.412808in,y=1.964724in,left,base]{\color{textcolor}\rmfamily\fontsize{10.000000}{12.000000}\selectfont \(\displaystyle 0.4\)}%
\end{pgfscope}%
\begin{pgfscope}%
\pgfsetbuttcap%
\pgfsetroundjoin%
\definecolor{currentfill}{rgb}{0.000000,0.000000,0.000000}%
\pgfsetfillcolor{currentfill}%
\pgfsetlinewidth{0.803000pt}%
\definecolor{currentstroke}{rgb}{0.000000,0.000000,0.000000}%
\pgfsetstrokecolor{currentstroke}%
\pgfsetdash{}{0pt}%
\pgfsys@defobject{currentmarker}{\pgfqpoint{-0.048611in}{0.000000in}}{\pgfqpoint{0.000000in}{0.000000in}}{%
\pgfpathmoveto{\pgfqpoint{0.000000in}{0.000000in}}%
\pgfpathlineto{\pgfqpoint{-0.048611in}{0.000000in}}%
\pgfusepath{stroke,fill}%
}%
\begin{pgfscope}%
\pgfsys@transformshift{0.687500in}{2.483565in}%
\pgfsys@useobject{currentmarker}{}%
\end{pgfscope}%
\end{pgfscope}%
\begin{pgfscope}%
\definecolor{textcolor}{rgb}{0.000000,0.000000,0.000000}%
\pgfsetstrokecolor{textcolor}%
\pgfsetfillcolor{textcolor}%
\pgftext[x=0.412808in,y=2.435370in,left,base]{\color{textcolor}\rmfamily\fontsize{10.000000}{12.000000}\selectfont \(\displaystyle 0.6\)}%
\end{pgfscope}%
\begin{pgfscope}%
\pgfsetbuttcap%
\pgfsetroundjoin%
\definecolor{currentfill}{rgb}{0.000000,0.000000,0.000000}%
\pgfsetfillcolor{currentfill}%
\pgfsetlinewidth{0.803000pt}%
\definecolor{currentstroke}{rgb}{0.000000,0.000000,0.000000}%
\pgfsetstrokecolor{currentstroke}%
\pgfsetdash{}{0pt}%
\pgfsys@defobject{currentmarker}{\pgfqpoint{-0.048611in}{0.000000in}}{\pgfqpoint{0.000000in}{0.000000in}}{%
\pgfpathmoveto{\pgfqpoint{0.000000in}{0.000000in}}%
\pgfpathlineto{\pgfqpoint{-0.048611in}{0.000000in}}%
\pgfusepath{stroke,fill}%
}%
\begin{pgfscope}%
\pgfsys@transformshift{0.687500in}{2.954211in}%
\pgfsys@useobject{currentmarker}{}%
\end{pgfscope}%
\end{pgfscope}%
\begin{pgfscope}%
\definecolor{textcolor}{rgb}{0.000000,0.000000,0.000000}%
\pgfsetstrokecolor{textcolor}%
\pgfsetfillcolor{textcolor}%
\pgftext[x=0.412808in,y=2.906016in,left,base]{\color{textcolor}\rmfamily\fontsize{10.000000}{12.000000}\selectfont \(\displaystyle 0.8\)}%
\end{pgfscope}%
\begin{pgfscope}%
\definecolor{textcolor}{rgb}{0.000000,0.000000,0.000000}%
\pgfsetstrokecolor{textcolor}%
\pgfsetfillcolor{textcolor}%
\pgftext[x=0.249228in,y=1.682598in,,bottom,rotate=90.000000]{\color{textcolor}\rmfamily\fontsize{10.000000}{12.000000}\selectfont \(\displaystyle J_{n}(\tilde{t})\)}%
\end{pgfscope}%
\begin{pgfscope}%
\pgfpathrectangle{\pgfqpoint{0.687500in}{0.373911in}}{\pgfqpoint{4.262500in}{2.617374in}}%
\pgfusepath{clip}%
\pgfsetrectcap%
\pgfsetroundjoin%
\pgfsetlinewidth{1.505625pt}%
\definecolor{currentstroke}{rgb}{0.121569,0.466667,0.705882}%
\pgfsetstrokecolor{currentstroke}%
\pgfsetdash{}{0pt}%
\pgfpathmoveto{\pgfqpoint{0.881250in}{2.872313in}}%
\pgfpathlineto{\pgfqpoint{2.005165in}{1.342020in}}%
\pgfpathlineto{\pgfqpoint{2.183285in}{1.117663in}}%
\pgfpathlineto{\pgfqpoint{2.309663in}{1.077454in}}%
\pgfpathlineto{\pgfqpoint{2.407690in}{1.072214in}}%
\pgfpathlineto{\pgfqpoint{2.614162in}{1.071626in}}%
\pgfpathlineto{\pgfqpoint{4.756250in}{1.071626in}}%
\pgfpathlineto{\pgfqpoint{4.756250in}{1.071626in}}%
\pgfusepath{stroke}%
\end{pgfscope}%
\begin{pgfscope}%
\pgfpathrectangle{\pgfqpoint{0.687500in}{0.373911in}}{\pgfqpoint{4.262500in}{2.617374in}}%
\pgfusepath{clip}%
\pgfsetbuttcap%
\pgfsetroundjoin%
\pgfsetlinewidth{1.505625pt}%
\definecolor{currentstroke}{rgb}{1.000000,0.498039,0.054902}%
\pgfsetstrokecolor{currentstroke}%
\pgfsetdash{{5.550000pt}{2.400000pt}}{0.000000pt}%
\pgfpathmoveto{\pgfqpoint{0.881250in}{0.492882in}}%
\pgfpathlineto{\pgfqpoint{1.443207in}{1.173927in}}%
\pgfpathlineto{\pgfqpoint{2.005165in}{1.670830in}}%
\pgfpathlineto{\pgfqpoint{2.183285in}{1.209006in}}%
\pgfpathlineto{\pgfqpoint{2.309663in}{0.554850in}}%
\pgfpathlineto{\pgfqpoint{2.407690in}{0.520825in}}%
\pgfpathlineto{\pgfqpoint{2.487784in}{1.037601in}}%
\pgfpathlineto{\pgfqpoint{2.555502in}{1.581597in}}%
\pgfpathlineto{\pgfqpoint{2.614162in}{1.819610in}}%
\pgfpathlineto{\pgfqpoint{2.665904in}{1.758430in}}%
\pgfpathlineto{\pgfqpoint{2.712189in}{1.559889in}}%
\pgfpathlineto{\pgfqpoint{2.754059in}{1.361348in}}%
\pgfpathlineto{\pgfqpoint{2.792283in}{1.220751in}}%
\pgfpathlineto{\pgfqpoint{2.827445in}{1.139804in}}%
\pgfpathlineto{\pgfqpoint{2.860001in}{1.099764in}}%
\pgfpathlineto{\pgfqpoint{2.890309in}{1.082234in}}%
\pgfpathlineto{\pgfqpoint{2.918661in}{1.075313in}}%
\pgfpathlineto{\pgfqpoint{2.945293in}{1.072816in}}%
\pgfpathlineto{\pgfqpoint{2.994155in}{1.071727in}}%
\pgfpathlineto{\pgfqpoint{3.352957in}{1.071626in}}%
\pgfpathlineto{\pgfqpoint{4.756250in}{1.071626in}}%
\pgfpathlineto{\pgfqpoint{4.756250in}{1.071626in}}%
\pgfusepath{stroke}%
\end{pgfscope}%
\begin{pgfscope}%
\pgfpathrectangle{\pgfqpoint{0.687500in}{0.373911in}}{\pgfqpoint{4.262500in}{2.617374in}}%
\pgfusepath{clip}%
\pgfsetbuttcap%
\pgfsetroundjoin%
\pgfsetlinewidth{1.505625pt}%
\definecolor{currentstroke}{rgb}{0.172549,0.627451,0.172549}%
\pgfsetstrokecolor{currentstroke}%
\pgfsetdash{{1.500000pt}{2.475000pt}}{0.000000pt}%
\pgfpathmoveto{\pgfqpoint{0.881250in}{1.118657in}}%
\pgfpathlineto{\pgfqpoint{1.443207in}{0.890085in}}%
\pgfpathlineto{\pgfqpoint{2.005165in}{1.020964in}}%
\pgfpathlineto{\pgfqpoint{2.183285in}{1.251140in}}%
\pgfpathlineto{\pgfqpoint{2.309663in}{1.133059in}}%
\pgfpathlineto{\pgfqpoint{2.407690in}{0.897026in}}%
\pgfpathlineto{\pgfqpoint{2.487784in}{0.992733in}}%
\pgfpathlineto{\pgfqpoint{2.555502in}{1.236758in}}%
\pgfpathlineto{\pgfqpoint{2.614162in}{1.173637in}}%
\pgfpathlineto{\pgfqpoint{2.665904in}{0.922815in}}%
\pgfpathlineto{\pgfqpoint{2.712189in}{0.942828in}}%
\pgfpathlineto{\pgfqpoint{2.754059in}{1.194677in}}%
\pgfpathlineto{\pgfqpoint{2.792283in}{1.227495in}}%
\pgfpathlineto{\pgfqpoint{2.827445in}{0.985983in}}%
\pgfpathlineto{\pgfqpoint{2.860001in}{0.893490in}}%
\pgfpathlineto{\pgfqpoint{2.890309in}{1.107391in}}%
\pgfpathlineto{\pgfqpoint{2.918661in}{1.260491in}}%
\pgfpathlineto{\pgfqpoint{2.945293in}{1.096298in}}%
\pgfpathlineto{\pgfqpoint{2.970403in}{0.891149in}}%
\pgfpathlineto{\pgfqpoint{2.994155in}{0.981982in}}%
\pgfpathlineto{\pgfqpoint{3.016688in}{1.218038in}}%
\pgfpathlineto{\pgfqpoint{3.038121in}{1.219834in}}%
\pgfpathlineto{\pgfqpoint{3.058557in}{0.987461in}}%
\pgfpathlineto{\pgfqpoint{3.078085in}{0.886385in}}%
\pgfpathlineto{\pgfqpoint{3.114714in}{1.256365in}}%
\pgfpathlineto{\pgfqpoint{3.131944in}{1.165042in}}%
\pgfpathlineto{\pgfqpoint{3.148523in}{0.935463in}}%
\pgfpathlineto{\pgfqpoint{3.164500in}{0.904682in}}%
\pgfpathlineto{\pgfqpoint{3.179915in}{1.114300in}}%
\pgfpathlineto{\pgfqpoint{3.194808in}{1.263320in}}%
\pgfpathlineto{\pgfqpoint{3.209213in}{1.143968in}}%
\pgfpathlineto{\pgfqpoint{3.223160in}{0.924784in}}%
\pgfpathlineto{\pgfqpoint{3.236678in}{0.905304in}}%
\pgfpathlineto{\pgfqpoint{3.249792in}{1.108696in}}%
\pgfpathlineto{\pgfqpoint{3.262526in}{1.263155in}}%
\pgfpathlineto{\pgfqpoint{3.274902in}{1.168626in}}%
\pgfpathlineto{\pgfqpoint{3.286938in}{0.949937in}}%
\pgfpathlineto{\pgfqpoint{3.298653in}{0.884576in}}%
\pgfpathlineto{\pgfqpoint{3.310064in}{1.051157in}}%
\pgfpathlineto{\pgfqpoint{3.321186in}{1.242710in}}%
\pgfpathlineto{\pgfqpoint{3.332034in}{1.228962in}}%
\pgfpathlineto{\pgfqpoint{3.342620in}{1.029558in}}%
\pgfpathlineto{\pgfqpoint{3.352957in}{0.878952in}}%
\pgfpathlineto{\pgfqpoint{3.363056in}{0.947994in}}%
\pgfpathlineto{\pgfqpoint{3.372928in}{1.155504in}}%
\pgfpathlineto{\pgfqpoint{3.382584in}{1.270748in}}%
\pgfpathlineto{\pgfqpoint{3.392031in}{1.170940in}}%
\pgfpathlineto{\pgfqpoint{3.401280in}{0.965859in}}%
\pgfpathlineto{\pgfqpoint{3.410338in}{0.870776in}}%
\pgfpathlineto{\pgfqpoint{3.419213in}{0.980560in}}%
\pgfpathlineto{\pgfqpoint{3.427912in}{1.181410in}}%
\pgfpathlineto{\pgfqpoint{3.436443in}{1.274671in}}%
\pgfpathlineto{\pgfqpoint{3.444811in}{1.173009in}}%
\pgfpathlineto{\pgfqpoint{3.453022in}{0.976047in}}%
\pgfpathlineto{\pgfqpoint{3.461083in}{0.867017in}}%
\pgfpathlineto{\pgfqpoint{3.468998in}{0.942135in}}%
\pgfpathlineto{\pgfqpoint{3.476774in}{1.131205in}}%
\pgfpathlineto{\pgfqpoint{3.484414in}{1.269037in}}%
\pgfpathlineto{\pgfqpoint{3.491924in}{1.241043in}}%
\pgfpathlineto{\pgfqpoint{3.506568in}{0.905210in}}%
\pgfpathlineto{\pgfqpoint{3.513711in}{0.866097in}}%
\pgfpathlineto{\pgfqpoint{3.520740in}{0.983186in}}%
\pgfpathlineto{\pgfqpoint{3.527659in}{1.165720in}}%
\pgfpathlineto{\pgfqpoint{3.534470in}{1.280506in}}%
\pgfpathlineto{\pgfqpoint{3.541177in}{1.249077in}}%
\pgfpathlineto{\pgfqpoint{3.554291in}{0.928150in}}%
\pgfpathlineto{\pgfqpoint{3.560704in}{0.851144in}}%
\pgfpathlineto{\pgfqpoint{3.567025in}{0.910837in}}%
\pgfpathlineto{\pgfqpoint{3.579401in}{1.225850in}}%
\pgfpathlineto{\pgfqpoint{3.585460in}{1.298332in}}%
\pgfpathlineto{\pgfqpoint{3.591437in}{1.248392in}}%
\pgfpathlineto{\pgfqpoint{3.608895in}{0.847623in}}%
\pgfpathlineto{\pgfqpoint{3.614563in}{0.851066in}}%
\pgfpathlineto{\pgfqpoint{3.620159in}{0.951556in}}%
\pgfpathlineto{\pgfqpoint{3.631142in}{1.241054in}}%
\pgfpathlineto{\pgfqpoint{3.636533in}{1.315250in}}%
\pgfpathlineto{\pgfqpoint{3.641858in}{1.301742in}}%
\pgfpathlineto{\pgfqpoint{3.647119in}{1.209995in}}%
\pgfpathlineto{\pgfqpoint{3.657456in}{0.937240in}}%
\pgfpathlineto{\pgfqpoint{3.662534in}{0.838140in}}%
\pgfpathlineto{\pgfqpoint{3.667555in}{0.799745in}}%
\pgfpathlineto{\pgfqpoint{3.672519in}{0.826602in}}%
\pgfpathlineto{\pgfqpoint{3.677427in}{0.907365in}}%
\pgfpathlineto{\pgfqpoint{3.696530in}{1.340445in}}%
\pgfpathlineto{\pgfqpoint{3.701179in}{1.393725in}}%
\pgfpathlineto{\pgfqpoint{3.705779in}{1.414795in}}%
\pgfpathlineto{\pgfqpoint{3.710331in}{1.408411in}}%
\pgfpathlineto{\pgfqpoint{3.714837in}{1.381820in}}%
\pgfpathlineto{\pgfqpoint{3.723712in}{1.298399in}}%
\pgfpathlineto{\pgfqpoint{3.732411in}{1.213201in}}%
\pgfpathlineto{\pgfqpoint{3.740942in}{1.149364in}}%
\pgfpathlineto{\pgfqpoint{3.749309in}{1.109875in}}%
\pgfpathlineto{\pgfqpoint{3.757521in}{1.088710in}}%
\pgfpathlineto{\pgfqpoint{3.765582in}{1.078619in}}%
\pgfpathlineto{\pgfqpoint{3.773497in}{1.074269in}}%
\pgfpathlineto{\pgfqpoint{3.785109in}{1.072162in}}%
\pgfpathlineto{\pgfqpoint{3.811067in}{1.071633in}}%
\pgfpathlineto{\pgfqpoint{4.756250in}{1.071626in}}%
\pgfpathlineto{\pgfqpoint{4.756250in}{1.071626in}}%
\pgfusepath{stroke}%
\end{pgfscope}%
\begin{pgfscope}%
\pgfpathrectangle{\pgfqpoint{0.687500in}{0.373911in}}{\pgfqpoint{4.262500in}{2.617374in}}%
\pgfusepath{clip}%
\pgfsetbuttcap%
\pgfsetroundjoin%
\pgfsetlinewidth{1.505625pt}%
\definecolor{currentstroke}{rgb}{0.839216,0.152941,0.156863}%
\pgfsetstrokecolor{currentstroke}%
\pgfsetdash{{9.600000pt}{2.400000pt}{1.500000pt}{2.400000pt}}{0.000000pt}%
\pgfpathmoveto{\pgfqpoint{0.881250in}{1.129955in}}%
\pgfpathlineto{\pgfqpoint{1.443207in}{1.082753in}}%
\pgfpathlineto{\pgfqpoint{2.005165in}{1.013319in}}%
\pgfpathlineto{\pgfqpoint{2.183285in}{1.060266in}}%
\pgfpathlineto{\pgfqpoint{2.309663in}{1.129864in}}%
\pgfpathlineto{\pgfqpoint{2.407690in}{1.083452in}}%
\pgfpathlineto{\pgfqpoint{2.487784in}{1.013506in}}%
\pgfpathlineto{\pgfqpoint{2.555502in}{1.059102in}}%
\pgfpathlineto{\pgfqpoint{2.614162in}{1.129571in}}%
\pgfpathlineto{\pgfqpoint{2.665904in}{1.085076in}}%
\pgfpathlineto{\pgfqpoint{2.712189in}{1.013923in}}%
\pgfpathlineto{\pgfqpoint{2.754059in}{1.057021in}}%
\pgfpathlineto{\pgfqpoint{2.792283in}{1.129007in}}%
\pgfpathlineto{\pgfqpoint{2.827445in}{1.087608in}}%
\pgfpathlineto{\pgfqpoint{2.860001in}{1.014660in}}%
\pgfpathlineto{\pgfqpoint{2.890309in}{1.054049in}}%
\pgfpathlineto{\pgfqpoint{2.918661in}{1.128064in}}%
\pgfpathlineto{\pgfqpoint{2.945293in}{1.091009in}}%
\pgfpathlineto{\pgfqpoint{2.970403in}{1.015846in}}%
\pgfpathlineto{\pgfqpoint{2.994155in}{1.050235in}}%
\pgfpathlineto{\pgfqpoint{3.016688in}{1.126593in}}%
\pgfpathlineto{\pgfqpoint{3.038121in}{1.095215in}}%
\pgfpathlineto{\pgfqpoint{3.058557in}{1.017650in}}%
\pgfpathlineto{\pgfqpoint{3.078085in}{1.045661in}}%
\pgfpathlineto{\pgfqpoint{3.096781in}{1.124407in}}%
\pgfpathlineto{\pgfqpoint{3.114714in}{1.100124in}}%
\pgfpathlineto{\pgfqpoint{3.131944in}{1.020269in}}%
\pgfpathlineto{\pgfqpoint{3.148523in}{1.040457in}}%
\pgfpathlineto{\pgfqpoint{3.164500in}{1.121299in}}%
\pgfpathlineto{\pgfqpoint{3.179915in}{1.105576in}}%
\pgfpathlineto{\pgfqpoint{3.194808in}{1.023921in}}%
\pgfpathlineto{\pgfqpoint{3.209213in}{1.034813in}}%
\pgfpathlineto{\pgfqpoint{3.223160in}{1.117048in}}%
\pgfpathlineto{\pgfqpoint{3.236678in}{1.111345in}}%
\pgfpathlineto{\pgfqpoint{3.249792in}{1.028825in}}%
\pgfpathlineto{\pgfqpoint{3.262526in}{1.028996in}}%
\pgfpathlineto{\pgfqpoint{3.274902in}{1.111442in}}%
\pgfpathlineto{\pgfqpoint{3.286938in}{1.117123in}}%
\pgfpathlineto{\pgfqpoint{3.298653in}{1.035176in}}%
\pgfpathlineto{\pgfqpoint{3.310064in}{1.023359in}}%
\pgfpathlineto{\pgfqpoint{3.321186in}{1.104311in}}%
\pgfpathlineto{\pgfqpoint{3.332034in}{1.122508in}}%
\pgfpathlineto{\pgfqpoint{3.342620in}{1.043113in}}%
\pgfpathlineto{\pgfqpoint{3.352957in}{1.018349in}}%
\pgfpathlineto{\pgfqpoint{3.363056in}{1.095557in}}%
\pgfpathlineto{\pgfqpoint{3.372928in}{1.127009in}}%
\pgfpathlineto{\pgfqpoint{3.382584in}{1.052679in}}%
\pgfpathlineto{\pgfqpoint{3.392031in}{1.014500in}}%
\pgfpathlineto{\pgfqpoint{3.401280in}{1.085202in}}%
\pgfpathlineto{\pgfqpoint{3.410338in}{1.130055in}}%
\pgfpathlineto{\pgfqpoint{3.419213in}{1.063775in}}%
\pgfpathlineto{\pgfqpoint{3.427912in}{1.012412in}}%
\pgfpathlineto{\pgfqpoint{3.444811in}{1.131028in}}%
\pgfpathlineto{\pgfqpoint{3.453022in}{1.076112in}}%
\pgfpathlineto{\pgfqpoint{3.461083in}{1.012708in}}%
\pgfpathlineto{\pgfqpoint{3.468998in}{1.060659in}}%
\pgfpathlineto{\pgfqpoint{3.476774in}{1.129316in}}%
\pgfpathlineto{\pgfqpoint{3.484414in}{1.089169in}}%
\pgfpathlineto{\pgfqpoint{3.491924in}{1.015971in}}%
\pgfpathlineto{\pgfqpoint{3.499307in}{1.047515in}}%
\pgfpathlineto{\pgfqpoint{3.506568in}{1.124387in}}%
\pgfpathlineto{\pgfqpoint{3.513711in}{1.102174in}}%
\pgfpathlineto{\pgfqpoint{3.520740in}{1.022652in}}%
\pgfpathlineto{\pgfqpoint{3.527659in}{1.034907in}}%
\pgfpathlineto{\pgfqpoint{3.534470in}{1.115899in}}%
\pgfpathlineto{\pgfqpoint{3.541177in}{1.114100in}}%
\pgfpathlineto{\pgfqpoint{3.547783in}{1.032959in}}%
\pgfpathlineto{\pgfqpoint{3.554291in}{1.023971in}}%
\pgfpathlineto{\pgfqpoint{3.560704in}{1.103812in}}%
\pgfpathlineto{\pgfqpoint{3.567025in}{1.123723in}}%
\pgfpathlineto{\pgfqpoint{3.573256in}{1.046734in}}%
\pgfpathlineto{\pgfqpoint{3.579401in}{1.015994in}}%
\pgfpathlineto{\pgfqpoint{3.585460in}{1.088507in}}%
\pgfpathlineto{\pgfqpoint{3.591437in}{1.129722in}}%
\pgfpathlineto{\pgfqpoint{3.597334in}{1.063343in}}%
\pgfpathlineto{\pgfqpoint{3.603152in}{1.012287in}}%
\pgfpathlineto{\pgfqpoint{3.614563in}{1.130851in}}%
\pgfpathlineto{\pgfqpoint{3.620159in}{1.081601in}}%
\pgfpathlineto{\pgfqpoint{3.625685in}{1.013977in}}%
\pgfpathlineto{\pgfqpoint{3.631142in}{1.052426in}}%
\pgfpathlineto{\pgfqpoint{3.636533in}{1.126165in}}%
\pgfpathlineto{\pgfqpoint{3.641858in}{1.099770in}}%
\pgfpathlineto{\pgfqpoint{3.647119in}{1.021759in}}%
\pgfpathlineto{\pgfqpoint{3.652318in}{1.035104in}}%
\pgfpathlineto{\pgfqpoint{3.657456in}{1.115284in}}%
\pgfpathlineto{\pgfqpoint{3.662534in}{1.115657in}}%
\pgfpathlineto{\pgfqpoint{3.667555in}{1.035629in}}%
\pgfpathlineto{\pgfqpoint{3.672519in}{1.021260in}}%
\pgfpathlineto{\pgfqpoint{3.677427in}{1.098658in}}%
\pgfpathlineto{\pgfqpoint{3.682281in}{1.126858in}}%
\pgfpathlineto{\pgfqpoint{3.687083in}{1.054646in}}%
\pgfpathlineto{\pgfqpoint{3.691832in}{1.013270in}}%
\pgfpathlineto{\pgfqpoint{3.701179in}{1.131134in}}%
\pgfpathlineto{\pgfqpoint{3.710331in}{1.013113in}}%
\pgfpathlineto{\pgfqpoint{3.714837in}{1.055093in}}%
\pgfpathlineto{\pgfqpoint{3.719297in}{1.126898in}}%
\pgfpathlineto{\pgfqpoint{3.723712in}{1.099102in}}%
\pgfpathlineto{\pgfqpoint{3.728083in}{1.021849in}}%
\pgfpathlineto{\pgfqpoint{3.732411in}{1.034094in}}%
\pgfpathlineto{\pgfqpoint{3.736697in}{1.113747in}}%
\pgfpathlineto{\pgfqpoint{3.740942in}{1.117834in}}%
\pgfpathlineto{\pgfqpoint{3.745145in}{1.039116in}}%
\pgfpathlineto{\pgfqpoint{3.749309in}{1.018590in}}%
\pgfpathlineto{\pgfqpoint{3.753434in}{1.092892in}}%
\pgfpathlineto{\pgfqpoint{3.757521in}{1.129212in}}%
\pgfpathlineto{\pgfqpoint{3.765582in}{1.012115in}}%
\pgfpathlineto{\pgfqpoint{3.773497in}{1.130190in}}%
\pgfpathlineto{\pgfqpoint{3.777402in}{1.089009in}}%
\pgfpathlineto{\pgfqpoint{3.781273in}{1.016991in}}%
\pgfpathlineto{\pgfqpoint{3.785109in}{1.041786in}}%
\pgfpathlineto{\pgfqpoint{3.788913in}{1.119398in}}%
\pgfpathlineto{\pgfqpoint{3.792684in}{1.112549in}}%
\pgfpathlineto{\pgfqpoint{3.796422in}{1.033430in}}%
\pgfpathlineto{\pgfqpoint{3.800129in}{1.021688in}}%
\pgfpathlineto{\pgfqpoint{3.803806in}{1.097937in}}%
\pgfpathlineto{\pgfqpoint{3.807451in}{1.127878in}}%
\pgfpathlineto{\pgfqpoint{3.811067in}{1.058928in}}%
\pgfpathlineto{\pgfqpoint{3.814653in}{1.012275in}}%
\pgfpathlineto{\pgfqpoint{3.821739in}{1.130504in}}%
\pgfpathlineto{\pgfqpoint{3.825239in}{1.088248in}}%
\pgfpathlineto{\pgfqpoint{3.828712in}{1.016936in}}%
\pgfpathlineto{\pgfqpoint{3.832157in}{1.041112in}}%
\pgfpathlineto{\pgfqpoint{3.835576in}{1.118504in}}%
\pgfpathlineto{\pgfqpoint{3.838968in}{1.114234in}}%
\pgfpathlineto{\pgfqpoint{3.842335in}{1.035826in}}%
\pgfpathlineto{\pgfqpoint{3.845675in}{1.019638in}}%
\pgfpathlineto{\pgfqpoint{3.848991in}{1.093701in}}%
\pgfpathlineto{\pgfqpoint{3.852281in}{1.129486in}}%
\pgfpathlineto{\pgfqpoint{3.858790in}{1.011992in}}%
\pgfpathlineto{\pgfqpoint{3.865203in}{1.128615in}}%
\pgfpathlineto{\pgfqpoint{3.868375in}{1.097006in}}%
\pgfpathlineto{\pgfqpoint{3.871524in}{1.021693in}}%
\pgfpathlineto{\pgfqpoint{3.874651in}{1.032264in}}%
\pgfpathlineto{\pgfqpoint{3.877755in}{1.110459in}}%
\pgfpathlineto{\pgfqpoint{3.880838in}{1.122016in}}%
\pgfpathlineto{\pgfqpoint{3.883899in}{1.047204in}}%
\pgfpathlineto{\pgfqpoint{3.886939in}{1.014202in}}%
\pgfpathlineto{\pgfqpoint{3.892957in}{1.131318in}}%
\pgfpathlineto{\pgfqpoint{3.898894in}{1.014829in}}%
\pgfpathlineto{\pgfqpoint{3.901832in}{1.044920in}}%
\pgfpathlineto{\pgfqpoint{3.904751in}{1.120411in}}%
\pgfpathlineto{\pgfqpoint{3.907651in}{1.113065in}}%
\pgfpathlineto{\pgfqpoint{3.910532in}{1.035438in}}%
\pgfpathlineto{\pgfqpoint{3.913394in}{1.019113in}}%
\pgfpathlineto{\pgfqpoint{3.916237in}{1.091640in}}%
\pgfpathlineto{\pgfqpoint{3.919062in}{1.130343in}}%
\pgfpathlineto{\pgfqpoint{3.924658in}{1.012377in}}%
\pgfpathlineto{\pgfqpoint{3.927430in}{1.054597in}}%
\pgfpathlineto{\pgfqpoint{3.930184in}{1.125460in}}%
\pgfpathlineto{\pgfqpoint{3.932921in}{1.105881in}}%
\pgfpathlineto{\pgfqpoint{3.935641in}{1.028822in}}%
\pgfpathlineto{\pgfqpoint{3.938345in}{1.023502in}}%
\pgfpathlineto{\pgfqpoint{3.941032in}{1.098741in}}%
\pgfpathlineto{\pgfqpoint{3.943702in}{1.128643in}}%
\pgfpathlineto{\pgfqpoint{3.948995in}{1.011853in}}%
\pgfpathlineto{\pgfqpoint{3.954225in}{1.127480in}}%
\pgfpathlineto{\pgfqpoint{3.956816in}{1.102169in}}%
\pgfpathlineto{\pgfqpoint{3.959393in}{1.026157in}}%
\pgfpathlineto{\pgfqpoint{3.961954in}{1.025532in}}%
\pgfpathlineto{\pgfqpoint{3.964501in}{1.101239in}}%
\pgfpathlineto{\pgfqpoint{3.967033in}{1.127965in}}%
\pgfpathlineto{\pgfqpoint{3.972054in}{1.011784in}}%
\pgfpathlineto{\pgfqpoint{3.977018in}{1.127553in}}%
\pgfpathlineto{\pgfqpoint{3.979479in}{1.102594in}}%
\pgfpathlineto{\pgfqpoint{3.981926in}{1.026785in}}%
\pgfpathlineto{\pgfqpoint{3.984360in}{1.024515in}}%
\pgfpathlineto{\pgfqpoint{3.986780in}{1.099413in}}%
\pgfpathlineto{\pgfqpoint{3.989187in}{1.128851in}}%
\pgfpathlineto{\pgfqpoint{3.993962in}{1.011883in}}%
\pgfpathlineto{\pgfqpoint{3.996331in}{1.056363in}}%
\pgfpathlineto{\pgfqpoint{3.998686in}{1.125691in}}%
\pgfpathlineto{\pgfqpoint{4.001029in}{1.107109in}}%
\pgfpathlineto{\pgfqpoint{4.003360in}{1.030902in}}%
\pgfpathlineto{\pgfqpoint{4.005678in}{1.020749in}}%
\pgfpathlineto{\pgfqpoint{4.010278in}{1.130674in}}%
\pgfpathlineto{\pgfqpoint{4.014830in}{1.013073in}}%
\pgfpathlineto{\pgfqpoint{4.017089in}{1.047624in}}%
\pgfpathlineto{\pgfqpoint{4.019336in}{1.120818in}}%
\pgfpathlineto{\pgfqpoint{4.021571in}{1.114911in}}%
\pgfpathlineto{\pgfqpoint{4.023796in}{1.039488in}}%
\pgfpathlineto{\pgfqpoint{4.026009in}{1.015614in}}%
\pgfpathlineto{\pgfqpoint{4.030402in}{1.131575in}}%
\pgfpathlineto{\pgfqpoint{4.034751in}{1.017435in}}%
\pgfpathlineto{\pgfqpoint{4.036910in}{1.035370in}}%
\pgfpathlineto{\pgfqpoint{4.039058in}{1.111024in}}%
\pgfpathlineto{\pgfqpoint{4.041196in}{1.124035in}}%
\pgfpathlineto{\pgfqpoint{4.045440in}{1.011845in}}%
\pgfpathlineto{\pgfqpoint{4.049644in}{1.128454in}}%
\pgfpathlineto{\pgfqpoint{4.051731in}{1.102557in}}%
\pgfpathlineto{\pgfqpoint{4.053808in}{1.027851in}}%
\pgfpathlineto{\pgfqpoint{4.055876in}{1.022134in}}%
\pgfpathlineto{\pgfqpoint{4.059981in}{1.130830in}}%
\pgfpathlineto{\pgfqpoint{4.064049in}{1.013617in}}%
\pgfpathlineto{\pgfqpoint{4.066069in}{1.043684in}}%
\pgfpathlineto{\pgfqpoint{4.068079in}{1.117452in}}%
\pgfpathlineto{\pgfqpoint{4.070080in}{1.119640in}}%
\pgfpathlineto{\pgfqpoint{4.074056in}{1.012694in}}%
\pgfpathlineto{\pgfqpoint{4.077996in}{1.129904in}}%
\pgfpathlineto{\pgfqpoint{4.079953in}{1.099200in}}%
\pgfpathlineto{\pgfqpoint{4.081901in}{1.025756in}}%
\pgfpathlineto{\pgfqpoint{4.083840in}{1.023319in}}%
\pgfpathlineto{\pgfqpoint{4.087694in}{1.130849in}}%
\pgfpathlineto{\pgfqpoint{4.091514in}{1.013869in}}%
\pgfpathlineto{\pgfqpoint{4.093412in}{1.041756in}}%
\pgfpathlineto{\pgfqpoint{4.095301in}{1.115524in}}%
\pgfpathlineto{\pgfqpoint{4.097182in}{1.121952in}}%
\pgfpathlineto{\pgfqpoint{4.100921in}{1.011738in}}%
\pgfpathlineto{\pgfqpoint{4.104628in}{1.127759in}}%
\pgfpathlineto{\pgfqpoint{4.106470in}{1.106267in}}%
\pgfpathlineto{\pgfqpoint{4.108304in}{1.032051in}}%
\pgfpathlineto{\pgfqpoint{4.110131in}{1.017989in}}%
\pgfpathlineto{\pgfqpoint{4.113762in}{1.131904in}}%
\pgfpathlineto{\pgfqpoint{4.117362in}{1.018947in}}%
\pgfpathlineto{\pgfqpoint{4.119152in}{1.030240in}}%
\pgfpathlineto{\pgfqpoint{4.120934in}{1.103942in}}%
\pgfpathlineto{\pgfqpoint{4.122709in}{1.128976in}}%
\pgfpathlineto{\pgfqpoint{4.126238in}{1.012348in}}%
\pgfpathlineto{\pgfqpoint{4.127991in}{1.045857in}}%
\pgfpathlineto{\pgfqpoint{4.129738in}{1.117968in}}%
\pgfpathlineto{\pgfqpoint{4.131478in}{1.120751in}}%
\pgfpathlineto{\pgfqpoint{4.134937in}{1.011586in}}%
\pgfpathlineto{\pgfqpoint{4.138369in}{1.126988in}}%
\pgfpathlineto{\pgfqpoint{4.140075in}{1.109217in}}%
\pgfpathlineto{\pgfqpoint{4.141774in}{1.035661in}}%
\pgfpathlineto{\pgfqpoint{4.143467in}{1.015405in}}%
\pgfpathlineto{\pgfqpoint{4.146833in}{1.131360in}}%
\pgfpathlineto{\pgfqpoint{4.148507in}{1.096197in}}%
\pgfpathlineto{\pgfqpoint{4.150174in}{1.024816in}}%
\pgfpathlineto{\pgfqpoint{4.151835in}{1.022339in}}%
\pgfpathlineto{\pgfqpoint{4.155138in}{1.131918in}}%
\pgfpathlineto{\pgfqpoint{4.158416in}{1.017501in}}%
\pgfpathlineto{\pgfqpoint{4.160046in}{1.030999in}}%
\pgfpathlineto{\pgfqpoint{4.163288in}{1.129697in}}%
\pgfpathlineto{\pgfqpoint{4.166507in}{1.013229in}}%
\pgfpathlineto{\pgfqpoint{4.168107in}{1.040219in}}%
\pgfpathlineto{\pgfqpoint{4.169702in}{1.112749in}}%
\pgfpathlineto{\pgfqpoint{4.171291in}{1.125732in}}%
\pgfpathlineto{\pgfqpoint{4.174451in}{1.011321in}}%
\pgfpathlineto{\pgfqpoint{4.176023in}{1.049124in}}%
\pgfpathlineto{\pgfqpoint{4.177589in}{1.119329in}}%
\pgfpathlineto{\pgfqpoint{4.179149in}{1.120937in}}%
\pgfpathlineto{\pgfqpoint{4.182254in}{1.011056in}}%
\pgfpathlineto{\pgfqpoint{4.185337in}{1.123924in}}%
\pgfpathlineto{\pgfqpoint{4.186870in}{1.116052in}}%
\pgfpathlineto{\pgfqpoint{4.189921in}{1.011764in}}%
\pgfpathlineto{\pgfqpoint{4.192950in}{1.126962in}}%
\pgfpathlineto{\pgfqpoint{4.194458in}{1.111634in}}%
\pgfpathlineto{\pgfqpoint{4.197456in}{1.012882in}}%
\pgfpathlineto{\pgfqpoint{4.200434in}{1.128853in}}%
\pgfpathlineto{\pgfqpoint{4.201916in}{1.108074in}}%
\pgfpathlineto{\pgfqpoint{4.204864in}{1.013973in}}%
\pgfpathlineto{\pgfqpoint{4.207793in}{1.129939in}}%
\pgfpathlineto{\pgfqpoint{4.209250in}{1.105629in}}%
\pgfpathlineto{\pgfqpoint{4.212150in}{1.014731in}}%
\pgfpathlineto{\pgfqpoint{4.215030in}{1.130462in}}%
\pgfpathlineto{\pgfqpoint{4.216464in}{1.104451in}}%
\pgfpathlineto{\pgfqpoint{4.219316in}{1.014973in}}%
\pgfpathlineto{\pgfqpoint{4.222151in}{1.130555in}}%
\pgfpathlineto{\pgfqpoint{4.223561in}{1.104609in}}%
\pgfpathlineto{\pgfqpoint{4.226368in}{1.014637in}}%
\pgfpathlineto{\pgfqpoint{4.229157in}{1.130226in}}%
\pgfpathlineto{\pgfqpoint{4.230545in}{1.106103in}}%
\pgfpathlineto{\pgfqpoint{4.233308in}{1.013781in}}%
\pgfpathlineto{\pgfqpoint{4.236054in}{1.129366in}}%
\pgfpathlineto{\pgfqpoint{4.237420in}{1.108859in}}%
\pgfpathlineto{\pgfqpoint{4.240140in}{1.012582in}}%
\pgfpathlineto{\pgfqpoint{4.242843in}{1.127746in}}%
\pgfpathlineto{\pgfqpoint{4.244189in}{1.112714in}}%
\pgfpathlineto{\pgfqpoint{4.246868in}{1.011351in}}%
\pgfpathlineto{\pgfqpoint{4.249530in}{1.125028in}}%
\pgfpathlineto{\pgfqpoint{4.250855in}{1.117391in}}%
\pgfpathlineto{\pgfqpoint{4.253494in}{1.010533in}}%
\pgfpathlineto{\pgfqpoint{4.257422in}{1.122463in}}%
\pgfpathlineto{\pgfqpoint{4.260021in}{1.010712in}}%
\pgfpathlineto{\pgfqpoint{4.261315in}{1.045232in}}%
\pgfpathlineto{\pgfqpoint{4.263892in}{1.127322in}}%
\pgfpathlineto{\pgfqpoint{4.266453in}{1.012589in}}%
\pgfpathlineto{\pgfqpoint{4.267728in}{1.035878in}}%
\pgfpathlineto{\pgfqpoint{4.270268in}{1.131158in}}%
\pgfpathlineto{\pgfqpoint{4.272792in}{1.016927in}}%
\pgfpathlineto{\pgfqpoint{4.274049in}{1.026501in}}%
\pgfpathlineto{\pgfqpoint{4.276553in}{1.132975in}}%
\pgfpathlineto{\pgfqpoint{4.280281in}{1.018149in}}%
\pgfpathlineto{\pgfqpoint{4.282749in}{1.131657in}}%
\pgfpathlineto{\pgfqpoint{4.283977in}{1.105283in}}%
\pgfpathlineto{\pgfqpoint{4.286425in}{1.012169in}}%
\pgfpathlineto{\pgfqpoint{4.288859in}{1.126118in}}%
\pgfpathlineto{\pgfqpoint{4.290070in}{1.117955in}}%
\pgfpathlineto{\pgfqpoint{4.292484in}{1.010113in}}%
\pgfpathlineto{\pgfqpoint{4.296080in}{1.127903in}}%
\pgfpathlineto{\pgfqpoint{4.298461in}{1.013510in}}%
\pgfpathlineto{\pgfqpoint{4.299647in}{1.031111in}}%
\pgfpathlineto{\pgfqpoint{4.302009in}{1.133012in}}%
\pgfpathlineto{\pgfqpoint{4.305528in}{1.017798in}}%
\pgfpathlineto{\pgfqpoint{4.307858in}{1.131213in}}%
\pgfpathlineto{\pgfqpoint{4.309019in}{1.109036in}}%
\pgfpathlineto{\pgfqpoint{4.311331in}{1.010481in}}%
\pgfpathlineto{\pgfqpoint{4.314777in}{1.124835in}}%
\pgfpathlineto{\pgfqpoint{4.317059in}{1.011838in}}%
\pgfpathlineto{\pgfqpoint{4.318195in}{1.034013in}}%
\pgfpathlineto{\pgfqpoint{4.320460in}{1.133042in}}%
\pgfpathlineto{\pgfqpoint{4.323835in}{1.016926in}}%
\pgfpathlineto{\pgfqpoint{4.326070in}{1.130404in}}%
\pgfpathlineto{\pgfqpoint{4.327184in}{1.112755in}}%
\pgfpathlineto{\pgfqpoint{4.329402in}{1.009678in}}%
\pgfpathlineto{\pgfqpoint{4.332710in}{1.129414in}}%
\pgfpathlineto{\pgfqpoint{4.334901in}{1.015852in}}%
\pgfpathlineto{\pgfqpoint{4.335992in}{1.024272in}}%
\pgfpathlineto{\pgfqpoint{4.338167in}{1.133206in}}%
\pgfpathlineto{\pgfqpoint{4.339250in}{1.103898in}}%
\pgfpathlineto{\pgfqpoint{4.341409in}{1.010558in}}%
\pgfpathlineto{\pgfqpoint{4.344627in}{1.126461in}}%
\pgfpathlineto{\pgfqpoint{4.346760in}{1.013356in}}%
\pgfpathlineto{\pgfqpoint{4.347822in}{1.027663in}}%
\pgfpathlineto{\pgfqpoint{4.349939in}{1.133739in}}%
\pgfpathlineto{\pgfqpoint{4.350994in}{1.101436in}}%
\pgfpathlineto{\pgfqpoint{4.353096in}{1.010673in}}%
\pgfpathlineto{\pgfqpoint{4.356230in}{1.126995in}}%
\pgfpathlineto{\pgfqpoint{4.358307in}{1.014051in}}%
\pgfpathlineto{\pgfqpoint{4.359342in}{1.025409in}}%
\pgfpathlineto{\pgfqpoint{4.361404in}{1.133426in}}%
\pgfpathlineto{\pgfqpoint{4.362432in}{1.106113in}}%
\pgfpathlineto{\pgfqpoint{4.364480in}{1.009443in}}%
\pgfpathlineto{\pgfqpoint{4.367534in}{1.130762in}}%
\pgfpathlineto{\pgfqpoint{4.370567in}{1.018390in}}%
\pgfpathlineto{\pgfqpoint{4.372578in}{1.130435in}}%
\pgfpathlineto{\pgfqpoint{4.373580in}{1.116766in}}%
\pgfpathlineto{\pgfqpoint{4.375577in}{1.009611in}}%
\pgfpathlineto{\pgfqpoint{4.376572in}{1.036104in}}%
\pgfpathlineto{\pgfqpoint{4.378555in}{1.134271in}}%
\pgfpathlineto{\pgfqpoint{4.381513in}{1.010590in}}%
\pgfpathlineto{\pgfqpoint{4.384452in}{1.129207in}}%
\pgfpathlineto{\pgfqpoint{4.386400in}{1.017198in}}%
\pgfpathlineto{\pgfqpoint{4.387370in}{1.018927in}}%
\pgfpathlineto{\pgfqpoint{4.389306in}{1.130393in}}%
\pgfpathlineto{\pgfqpoint{4.390270in}{1.118421in}}%
\pgfpathlineto{\pgfqpoint{4.392193in}{1.010125in}}%
\pgfpathlineto{\pgfqpoint{4.393151in}{1.031380in}}%
\pgfpathlineto{\pgfqpoint{4.395061in}{1.134463in}}%
\pgfpathlineto{\pgfqpoint{4.396013in}{1.105006in}}%
\pgfpathlineto{\pgfqpoint{4.397910in}{1.008620in}}%
\pgfpathlineto{\pgfqpoint{4.400741in}{1.133875in}}%
\pgfpathlineto{\pgfqpoint{4.403555in}{1.010825in}}%
\pgfpathlineto{\pgfqpoint{4.406350in}{1.130506in}}%
\pgfpathlineto{\pgfqpoint{4.409127in}{1.014943in}}%
\pgfpathlineto{\pgfqpoint{4.410969in}{1.126200in}}%
\pgfpathlineto{\pgfqpoint{4.411887in}{1.126009in}}%
\pgfpathlineto{\pgfqpoint{4.413718in}{1.014866in}}%
\pgfpathlineto{\pgfqpoint{4.414630in}{1.019498in}}%
\pgfpathlineto{\pgfqpoint{4.416449in}{1.130041in}}%
\pgfpathlineto{\pgfqpoint{4.417356in}{1.121663in}}%
\pgfpathlineto{\pgfqpoint{4.419163in}{1.011987in}}%
\pgfpathlineto{\pgfqpoint{4.420065in}{1.023416in}}%
\pgfpathlineto{\pgfqpoint{4.421861in}{1.132145in}}%
\pgfpathlineto{\pgfqpoint{4.422757in}{1.118349in}}%
\pgfpathlineto{\pgfqpoint{4.424543in}{1.010500in}}%
\pgfpathlineto{\pgfqpoint{4.425433in}{1.025993in}}%
\pgfpathlineto{\pgfqpoint{4.427208in}{1.133139in}}%
\pgfpathlineto{\pgfqpoint{4.428093in}{1.116606in}}%
\pgfpathlineto{\pgfqpoint{4.429857in}{1.009910in}}%
\pgfpathlineto{\pgfqpoint{4.430736in}{1.026836in}}%
\pgfpathlineto{\pgfqpoint{4.432490in}{1.133376in}}%
\pgfpathlineto{\pgfqpoint{4.433364in}{1.116696in}}%
\pgfpathlineto{\pgfqpoint{4.435107in}{1.010015in}}%
\pgfpathlineto{\pgfqpoint{4.435977in}{1.025808in}}%
\pgfpathlineto{\pgfqpoint{4.437709in}{1.132900in}}%
\pgfpathlineto{\pgfqpoint{4.438573in}{1.118637in}}%
\pgfpathlineto{\pgfqpoint{4.440296in}{1.010929in}}%
\pgfpathlineto{\pgfqpoint{4.441155in}{1.023011in}}%
\pgfpathlineto{\pgfqpoint{4.442868in}{1.131439in}}%
\pgfpathlineto{\pgfqpoint{4.443721in}{1.122200in}}%
\pgfpathlineto{\pgfqpoint{4.445424in}{1.013082in}}%
\pgfpathlineto{\pgfqpoint{4.446273in}{1.018814in}}%
\pgfpathlineto{\pgfqpoint{4.447966in}{1.128414in}}%
\pgfpathlineto{\pgfqpoint{4.448810in}{1.126854in}}%
\pgfpathlineto{\pgfqpoint{4.451332in}{1.013927in}}%
\pgfpathlineto{\pgfqpoint{4.453840in}{1.131694in}}%
\pgfpathlineto{\pgfqpoint{4.456334in}{1.009475in}}%
\pgfpathlineto{\pgfqpoint{4.458813in}{1.135360in}}%
\pgfpathlineto{\pgfqpoint{4.461279in}{1.007051in}}%
\pgfpathlineto{\pgfqpoint{4.463731in}{1.136038in}}%
\pgfpathlineto{\pgfqpoint{4.464545in}{1.111475in}}%
\pgfpathlineto{\pgfqpoint{4.466169in}{1.008648in}}%
\pgfpathlineto{\pgfqpoint{4.466979in}{1.024734in}}%
\pgfpathlineto{\pgfqpoint{4.468594in}{1.131631in}}%
\pgfpathlineto{\pgfqpoint{4.469399in}{1.124979in}}%
\pgfpathlineto{\pgfqpoint{4.471807in}{1.012804in}}%
\pgfpathlineto{\pgfqpoint{4.474201in}{1.134469in}}%
\pgfpathlineto{\pgfqpoint{4.476582in}{1.006689in}}%
\pgfpathlineto{\pgfqpoint{4.478950in}{1.136264in}}%
\pgfpathlineto{\pgfqpoint{4.479736in}{1.114696in}}%
\pgfpathlineto{\pgfqpoint{4.481305in}{1.010092in}}%
\pgfpathlineto{\pgfqpoint{4.482088in}{1.019071in}}%
\pgfpathlineto{\pgfqpoint{4.484426in}{1.131573in}}%
\pgfpathlineto{\pgfqpoint{4.486753in}{1.007167in}}%
\pgfpathlineto{\pgfqpoint{4.489067in}{1.136983in}}%
\pgfpathlineto{\pgfqpoint{4.489836in}{1.112906in}}%
\pgfpathlineto{\pgfqpoint{4.491369in}{1.009566in}}%
\pgfpathlineto{\pgfqpoint{4.492134in}{1.018513in}}%
\pgfpathlineto{\pgfqpoint{4.494420in}{1.133241in}}%
\pgfpathlineto{\pgfqpoint{4.496694in}{1.005983in}}%
\pgfpathlineto{\pgfqpoint{4.498956in}{1.135902in}}%
\pgfpathlineto{\pgfqpoint{4.499708in}{1.121018in}}%
\pgfpathlineto{\pgfqpoint{4.501955in}{1.011201in}}%
\pgfpathlineto{\pgfqpoint{4.504191in}{1.137442in}}%
\pgfpathlineto{\pgfqpoint{4.506415in}{1.007286in}}%
\pgfpathlineto{\pgfqpoint{4.507154in}{1.020825in}}%
\pgfpathlineto{\pgfqpoint{4.509363in}{1.133927in}}%
\pgfpathlineto{\pgfqpoint{4.511562in}{1.005213in}}%
\pgfpathlineto{\pgfqpoint{4.513021in}{1.089417in}}%
\pgfpathlineto{\pgfqpoint{4.513749in}{1.133465in}}%
\pgfpathlineto{\pgfqpoint{4.514476in}{1.128537in}}%
\pgfpathlineto{\pgfqpoint{4.516649in}{1.005846in}}%
\pgfpathlineto{\pgfqpoint{4.518811in}{1.136526in}}%
\pgfpathlineto{\pgfqpoint{4.519529in}{1.123629in}}%
\pgfpathlineto{\pgfqpoint{4.521677in}{1.007135in}}%
\pgfpathlineto{\pgfqpoint{4.523815in}{1.137789in}}%
\pgfpathlineto{\pgfqpoint{4.524525in}{1.120687in}}%
\pgfpathlineto{\pgfqpoint{4.526649in}{1.007762in}}%
\pgfpathlineto{\pgfqpoint{4.528763in}{1.138198in}}%
\pgfpathlineto{\pgfqpoint{4.529465in}{1.120455in}}%
\pgfpathlineto{\pgfqpoint{4.531566in}{1.007170in}}%
\pgfpathlineto{\pgfqpoint{4.533656in}{1.137959in}}%
\pgfpathlineto{\pgfqpoint{4.534350in}{1.123065in}}%
\pgfpathlineto{\pgfqpoint{4.536427in}{1.005559in}}%
\pgfpathlineto{\pgfqpoint{4.538495in}{1.136494in}}%
\pgfpathlineto{\pgfqpoint{4.539182in}{1.128034in}}%
\pgfpathlineto{\pgfqpoint{4.541236in}{1.003937in}}%
\pgfpathlineto{\pgfqpoint{4.542600in}{1.087740in}}%
\pgfpathlineto{\pgfqpoint{4.543960in}{1.134104in}}%
\pgfpathlineto{\pgfqpoint{4.545993in}{1.004185in}}%
\pgfpathlineto{\pgfqpoint{4.546668in}{1.020489in}}%
\pgfpathlineto{\pgfqpoint{4.548688in}{1.139016in}}%
\pgfpathlineto{\pgfqpoint{4.550698in}{1.009003in}}%
\pgfpathlineto{\pgfqpoint{4.551366in}{1.009717in}}%
\pgfpathlineto{\pgfqpoint{4.553365in}{1.139402in}}%
\pgfpathlineto{\pgfqpoint{4.554029in}{1.124159in}}%
\pgfpathlineto{\pgfqpoint{4.556015in}{1.003215in}}%
\pgfpathlineto{\pgfqpoint{4.558650in}{1.137140in}}%
\pgfpathlineto{\pgfqpoint{4.559306in}{1.099730in}}%
\pgfpathlineto{\pgfqpoint{4.560615in}{1.006140in}}%
\pgfpathlineto{\pgfqpoint{4.561268in}{1.011468in}}%
\pgfpathlineto{\pgfqpoint{4.563222in}{1.140143in}}%
\pgfpathlineto{\pgfqpoint{4.563872in}{1.124979in}}%
\pgfpathlineto{\pgfqpoint{4.565814in}{1.002305in}}%
\pgfpathlineto{\pgfqpoint{4.567104in}{1.078333in}}%
\pgfpathlineto{\pgfqpoint{4.568391in}{1.140213in}}%
\pgfpathlineto{\pgfqpoint{4.570952in}{1.004671in}}%
\pgfpathlineto{\pgfqpoint{4.571590in}{1.039549in}}%
\pgfpathlineto{\pgfqpoint{4.573499in}{1.136596in}}%
\pgfpathlineto{\pgfqpoint{4.575399in}{1.005874in}}%
\pgfpathlineto{\pgfqpoint{4.576031in}{1.008551in}}%
\pgfpathlineto{\pgfqpoint{4.577920in}{1.138820in}}%
\pgfpathlineto{\pgfqpoint{4.578548in}{1.133224in}}%
\pgfpathlineto{\pgfqpoint{4.580427in}{1.003584in}}%
\pgfpathlineto{\pgfqpoint{4.581051in}{1.010858in}}%
\pgfpathlineto{\pgfqpoint{4.582919in}{1.140070in}}%
\pgfpathlineto{\pgfqpoint{4.583540in}{1.132340in}}%
\pgfpathlineto{\pgfqpoint{4.585398in}{1.003176in}}%
\pgfpathlineto{\pgfqpoint{4.586015in}{1.010152in}}%
\pgfpathlineto{\pgfqpoint{4.587862in}{1.139628in}}%
\pgfpathlineto{\pgfqpoint{4.588476in}{1.134616in}}%
\pgfpathlineto{\pgfqpoint{4.590313in}{1.004687in}}%
\pgfpathlineto{\pgfqpoint{4.590924in}{1.006523in}}%
\pgfpathlineto{\pgfqpoint{4.593357in}{1.139164in}}%
\pgfpathlineto{\pgfqpoint{4.595778in}{1.001735in}}%
\pgfpathlineto{\pgfqpoint{4.598185in}{1.143242in}}%
\pgfpathlineto{\pgfqpoint{4.598785in}{1.119526in}}%
\pgfpathlineto{\pgfqpoint{4.600579in}{0.999598in}}%
\pgfpathlineto{\pgfqpoint{4.601771in}{1.060995in}}%
\pgfpathlineto{\pgfqpoint{4.602960in}{1.141948in}}%
\pgfpathlineto{\pgfqpoint{4.603553in}{1.135317in}}%
\pgfpathlineto{\pgfqpoint{4.605918in}{1.001931in}}%
\pgfpathlineto{\pgfqpoint{4.608271in}{1.144456in}}%
\pgfpathlineto{\pgfqpoint{4.608857in}{1.123559in}}%
\pgfpathlineto{\pgfqpoint{4.610610in}{0.999901in}}%
\pgfpathlineto{\pgfqpoint{4.611193in}{1.007927in}}%
\pgfpathlineto{\pgfqpoint{4.613518in}{1.143217in}}%
\pgfpathlineto{\pgfqpoint{4.615830in}{0.998092in}}%
\pgfpathlineto{\pgfqpoint{4.616406in}{1.011583in}}%
\pgfpathlineto{\pgfqpoint{4.618703in}{1.142885in}}%
\pgfpathlineto{\pgfqpoint{4.620988in}{0.997515in}}%
\pgfpathlineto{\pgfqpoint{4.621557in}{1.010202in}}%
\pgfpathlineto{\pgfqpoint{4.623828in}{1.144867in}}%
\pgfpathlineto{\pgfqpoint{4.626086in}{0.998414in}}%
\pgfpathlineto{\pgfqpoint{4.627211in}{1.039323in}}%
\pgfpathlineto{\pgfqpoint{4.628893in}{1.147062in}}%
\pgfpathlineto{\pgfqpoint{4.630011in}{1.089668in}}%
\pgfpathlineto{\pgfqpoint{4.631683in}{0.996962in}}%
\pgfpathlineto{\pgfqpoint{4.633348in}{1.113144in}}%
\pgfpathlineto{\pgfqpoint{4.634454in}{1.144146in}}%
\pgfpathlineto{\pgfqpoint{4.636659in}{0.997149in}}%
\pgfpathlineto{\pgfqpoint{4.637208in}{1.001330in}}%
\pgfpathlineto{\pgfqpoint{4.639399in}{1.147503in}}%
\pgfpathlineto{\pgfqpoint{4.639945in}{1.140522in}}%
\pgfpathlineto{\pgfqpoint{4.642123in}{0.995046in}}%
\pgfpathlineto{\pgfqpoint{4.642666in}{1.002783in}}%
\pgfpathlineto{\pgfqpoint{4.644830in}{1.148399in}}%
\pgfpathlineto{\pgfqpoint{4.645369in}{1.141917in}}%
\pgfpathlineto{\pgfqpoint{4.647520in}{0.995394in}}%
\pgfpathlineto{\pgfqpoint{4.648056in}{0.998611in}}%
\pgfpathlineto{\pgfqpoint{4.650726in}{1.147976in}}%
\pgfpathlineto{\pgfqpoint{4.651259in}{1.123308in}}%
\pgfpathlineto{\pgfqpoint{4.653381in}{0.992506in}}%
\pgfpathlineto{\pgfqpoint{4.654438in}{1.048608in}}%
\pgfpathlineto{\pgfqpoint{4.656019in}{1.151206in}}%
\pgfpathlineto{\pgfqpoint{4.656545in}{1.142934in}}%
\pgfpathlineto{\pgfqpoint{4.659165in}{0.992746in}}%
\pgfpathlineto{\pgfqpoint{4.659687in}{1.014429in}}%
\pgfpathlineto{\pgfqpoint{4.661769in}{1.152592in}}%
\pgfpathlineto{\pgfqpoint{4.662288in}{1.144266in}}%
\pgfpathlineto{\pgfqpoint{4.664873in}{0.990030in}}%
\pgfpathlineto{\pgfqpoint{4.665388in}{1.006515in}}%
\pgfpathlineto{\pgfqpoint{4.667956in}{1.152171in}}%
\pgfpathlineto{\pgfqpoint{4.669489in}{1.049023in}}%
\pgfpathlineto{\pgfqpoint{4.670509in}{0.990342in}}%
\pgfpathlineto{\pgfqpoint{4.671017in}{0.991589in}}%
\pgfpathlineto{\pgfqpoint{4.673552in}{1.153765in}}%
\pgfpathlineto{\pgfqpoint{4.674562in}{1.130709in}}%
\pgfpathlineto{\pgfqpoint{4.677077in}{0.987831in}}%
\pgfpathlineto{\pgfqpoint{4.678579in}{1.082722in}}%
\pgfpathlineto{\pgfqpoint{4.680075in}{1.158046in}}%
\pgfpathlineto{\pgfqpoint{4.681070in}{1.117257in}}%
\pgfpathlineto{\pgfqpoint{4.683054in}{0.985736in}}%
\pgfpathlineto{\pgfqpoint{4.683548in}{0.987335in}}%
\pgfpathlineto{\pgfqpoint{4.685028in}{1.082872in}}%
\pgfpathlineto{\pgfqpoint{4.686503in}{1.160430in}}%
\pgfpathlineto{\pgfqpoint{4.686994in}{1.153147in}}%
\pgfpathlineto{\pgfqpoint{4.689925in}{0.981267in}}%
\pgfpathlineto{\pgfqpoint{4.690898in}{1.013916in}}%
\pgfpathlineto{\pgfqpoint{4.693322in}{1.163296in}}%
\pgfpathlineto{\pgfqpoint{4.693805in}{1.158281in}}%
\pgfpathlineto{\pgfqpoint{4.695731in}{1.029513in}}%
\pgfpathlineto{\pgfqpoint{4.697171in}{0.978544in}}%
\pgfpathlineto{\pgfqpoint{4.698128in}{1.017084in}}%
\pgfpathlineto{\pgfqpoint{4.700987in}{1.166019in}}%
\pgfpathlineto{\pgfqpoint{4.702409in}{1.092197in}}%
\pgfpathlineto{\pgfqpoint{4.704770in}{0.973077in}}%
\pgfpathlineto{\pgfqpoint{4.705710in}{1.003792in}}%
\pgfpathlineto{\pgfqpoint{4.708987in}{1.172960in}}%
\pgfpathlineto{\pgfqpoint{4.709919in}{1.141533in}}%
\pgfpathlineto{\pgfqpoint{4.713164in}{0.964944in}}%
\pgfpathlineto{\pgfqpoint{4.713626in}{0.968473in}}%
\pgfpathlineto{\pgfqpoint{4.715008in}{1.031877in}}%
\pgfpathlineto{\pgfqpoint{4.718216in}{1.184802in}}%
\pgfpathlineto{\pgfqpoint{4.719129in}{1.164747in}}%
\pgfpathlineto{\pgfqpoint{4.723662in}{0.947147in}}%
\pgfpathlineto{\pgfqpoint{4.724563in}{0.956612in}}%
\pgfpathlineto{\pgfqpoint{4.726360in}{1.033910in}}%
\pgfpathlineto{\pgfqpoint{4.730376in}{1.218572in}}%
\pgfpathlineto{\pgfqpoint{4.731707in}{1.230603in}}%
\pgfpathlineto{\pgfqpoint{4.732149in}{1.229148in}}%
\pgfpathlineto{\pgfqpoint{4.733474in}{1.212993in}}%
\pgfpathlineto{\pgfqpoint{4.741343in}{1.086312in}}%
\pgfpathlineto{\pgfqpoint{4.744365in}{1.075271in}}%
\pgfpathlineto{\pgfqpoint{4.747366in}{1.072354in}}%
\pgfpathlineto{\pgfqpoint{4.752042in}{1.071665in}}%
\pgfpathlineto{\pgfqpoint{4.756250in}{1.071628in}}%
\pgfpathlineto{\pgfqpoint{4.756250in}{1.071628in}}%
\pgfusepath{stroke}%
\end{pgfscope}%
\begin{pgfscope}%
\pgfsetrectcap%
\pgfsetmiterjoin%
\pgfsetlinewidth{0.803000pt}%
\definecolor{currentstroke}{rgb}{0.000000,0.000000,0.000000}%
\pgfsetstrokecolor{currentstroke}%
\pgfsetdash{}{0pt}%
\pgfpathmoveto{\pgfqpoint{0.687500in}{0.373911in}}%
\pgfpathlineto{\pgfqpoint{0.687500in}{2.991285in}}%
\pgfusepath{stroke}%
\end{pgfscope}%
\begin{pgfscope}%
\pgfsetrectcap%
\pgfsetmiterjoin%
\pgfsetlinewidth{0.803000pt}%
\definecolor{currentstroke}{rgb}{0.000000,0.000000,0.000000}%
\pgfsetstrokecolor{currentstroke}%
\pgfsetdash{}{0pt}%
\pgfpathmoveto{\pgfqpoint{4.950000in}{0.373911in}}%
\pgfpathlineto{\pgfqpoint{4.950000in}{2.991285in}}%
\pgfusepath{stroke}%
\end{pgfscope}%
\begin{pgfscope}%
\pgfsetrectcap%
\pgfsetmiterjoin%
\pgfsetlinewidth{0.803000pt}%
\definecolor{currentstroke}{rgb}{0.000000,0.000000,0.000000}%
\pgfsetstrokecolor{currentstroke}%
\pgfsetdash{}{0pt}%
\pgfpathmoveto{\pgfqpoint{0.687500in}{0.373911in}}%
\pgfpathlineto{\pgfqpoint{4.950000in}{0.373911in}}%
\pgfusepath{stroke}%
\end{pgfscope}%
\begin{pgfscope}%
\pgfsetrectcap%
\pgfsetmiterjoin%
\pgfsetlinewidth{0.803000pt}%
\definecolor{currentstroke}{rgb}{0.000000,0.000000,0.000000}%
\pgfsetstrokecolor{currentstroke}%
\pgfsetdash{}{0pt}%
\pgfpathmoveto{\pgfqpoint{0.687500in}{2.991285in}}%
\pgfpathlineto{\pgfqpoint{4.950000in}{2.991285in}}%
\pgfusepath{stroke}%
\end{pgfscope}%
\begin{pgfscope}%
\pgfsetbuttcap%
\pgfsetmiterjoin%
\definecolor{currentfill}{rgb}{1.000000,1.000000,1.000000}%
\pgfsetfillcolor{currentfill}%
\pgfsetfillopacity{0.800000}%
\pgfsetlinewidth{1.003750pt}%
\definecolor{currentstroke}{rgb}{0.800000,0.800000,0.800000}%
\pgfsetstrokecolor{currentstroke}%
\pgfsetstrokeopacity{0.800000}%
\pgfsetdash{}{0pt}%
\pgfpathmoveto{\pgfqpoint{0.784722in}{2.105730in}}%
\pgfpathlineto{\pgfqpoint{1.665701in}{2.105730in}}%
\pgfpathquadraticcurveto{\pgfqpoint{1.693479in}{2.105730in}}{\pgfqpoint{1.693479in}{2.133507in}}%
\pgfpathlineto{\pgfqpoint{1.693479in}{2.894062in}}%
\pgfpathquadraticcurveto{\pgfqpoint{1.693479in}{2.921840in}}{\pgfqpoint{1.665701in}{2.921840in}}%
\pgfpathlineto{\pgfqpoint{0.784722in}{2.921840in}}%
\pgfpathquadraticcurveto{\pgfqpoint{0.756944in}{2.921840in}}{\pgfqpoint{0.756944in}{2.894062in}}%
\pgfpathlineto{\pgfqpoint{0.756944in}{2.133507in}}%
\pgfpathquadraticcurveto{\pgfqpoint{0.756944in}{2.105730in}}{\pgfqpoint{0.784722in}{2.105730in}}%
\pgfpathclose%
\pgfusepath{stroke,fill}%
\end{pgfscope}%
\begin{pgfscope}%
\pgfsetrectcap%
\pgfsetroundjoin%
\pgfsetlinewidth{1.505625pt}%
\definecolor{currentstroke}{rgb}{0.121569,0.466667,0.705882}%
\pgfsetstrokecolor{currentstroke}%
\pgfsetdash{}{0pt}%
\pgfpathmoveto{\pgfqpoint{0.812500in}{2.817673in}}%
\pgfpathlineto{\pgfqpoint{1.090278in}{2.817673in}}%
\pgfusepath{stroke}%
\end{pgfscope}%
\begin{pgfscope}%
\definecolor{textcolor}{rgb}{0.000000,0.000000,0.000000}%
\pgfsetstrokecolor{textcolor}%
\pgfsetfillcolor{textcolor}%
\pgftext[x=1.201389in,y=2.769062in,left,base]{\color{textcolor}\rmfamily\fontsize{10.000000}{12.000000}\selectfont \(\displaystyle \tilde{t} = 10^0\)}%
\end{pgfscope}%
\begin{pgfscope}%
\pgfsetbuttcap%
\pgfsetroundjoin%
\pgfsetlinewidth{1.505625pt}%
\definecolor{currentstroke}{rgb}{1.000000,0.498039,0.054902}%
\pgfsetstrokecolor{currentstroke}%
\pgfsetdash{{5.550000pt}{2.400000pt}}{0.000000pt}%
\pgfpathmoveto{\pgfqpoint{0.812500in}{2.624062in}}%
\pgfpathlineto{\pgfqpoint{1.090278in}{2.624062in}}%
\pgfusepath{stroke}%
\end{pgfscope}%
\begin{pgfscope}%
\definecolor{textcolor}{rgb}{0.000000,0.000000,0.000000}%
\pgfsetstrokecolor{textcolor}%
\pgfsetfillcolor{textcolor}%
\pgftext[x=1.201389in,y=2.575451in,left,base]{\color{textcolor}\rmfamily\fontsize{10.000000}{12.000000}\selectfont \(\displaystyle \tilde{t} = 10^1\)}%
\end{pgfscope}%
\begin{pgfscope}%
\pgfsetbuttcap%
\pgfsetroundjoin%
\pgfsetlinewidth{1.505625pt}%
\definecolor{currentstroke}{rgb}{0.172549,0.627451,0.172549}%
\pgfsetstrokecolor{currentstroke}%
\pgfsetdash{{1.500000pt}{2.475000pt}}{0.000000pt}%
\pgfpathmoveto{\pgfqpoint{0.812500in}{2.430451in}}%
\pgfpathlineto{\pgfqpoint{1.090278in}{2.430451in}}%
\pgfusepath{stroke}%
\end{pgfscope}%
\begin{pgfscope}%
\definecolor{textcolor}{rgb}{0.000000,0.000000,0.000000}%
\pgfsetstrokecolor{textcolor}%
\pgfsetfillcolor{textcolor}%
\pgftext[x=1.201389in,y=2.381840in,left,base]{\color{textcolor}\rmfamily\fontsize{10.000000}{12.000000}\selectfont \(\displaystyle \tilde{t} = 10^2\)}%
\end{pgfscope}%
\begin{pgfscope}%
\pgfsetbuttcap%
\pgfsetroundjoin%
\pgfsetlinewidth{1.505625pt}%
\definecolor{currentstroke}{rgb}{0.839216,0.152941,0.156863}%
\pgfsetstrokecolor{currentstroke}%
\pgfsetdash{{9.600000pt}{2.400000pt}{1.500000pt}{2.400000pt}}{0.000000pt}%
\pgfpathmoveto{\pgfqpoint{0.812500in}{2.236840in}}%
\pgfpathlineto{\pgfqpoint{1.090278in}{2.236840in}}%
\pgfusepath{stroke}%
\end{pgfscope}%
\begin{pgfscope}%
\definecolor{textcolor}{rgb}{0.000000,0.000000,0.000000}%
\pgfsetstrokecolor{textcolor}%
\pgfsetfillcolor{textcolor}%
\pgftext[x=1.201389in,y=2.188229in,left,base]{\color{textcolor}\rmfamily\fontsize{10.000000}{12.000000}\selectfont \(\displaystyle \tilde{t} = 10^3\)}%
\end{pgfscope}%
\begin{pgfscope}%
\pgfsetbuttcap%
\pgfsetmiterjoin%
\definecolor{currentfill}{rgb}{1.000000,1.000000,1.000000}%
\pgfsetfillcolor{currentfill}%
\pgfsetlinewidth{0.000000pt}%
\definecolor{currentstroke}{rgb}{0.000000,0.000000,0.000000}%
\pgfsetstrokecolor{currentstroke}%
\pgfsetstrokeopacity{0.000000}%
\pgfsetdash{}{0pt}%
\pgfpathmoveto{\pgfqpoint{3.025000in}{1.869553in}}%
\pgfpathlineto{\pgfqpoint{4.675000in}{1.869553in}}%
\pgfpathlineto{\pgfqpoint{4.675000in}{2.719350in}}%
\pgfpathlineto{\pgfqpoint{3.025000in}{2.719350in}}%
\pgfpathclose%
\pgfusepath{fill}%
\end{pgfscope}%
\begin{pgfscope}%
\pgfsetbuttcap%
\pgfsetroundjoin%
\definecolor{currentfill}{rgb}{0.000000,0.000000,0.000000}%
\pgfsetfillcolor{currentfill}%
\pgfsetlinewidth{0.803000pt}%
\definecolor{currentstroke}{rgb}{0.000000,0.000000,0.000000}%
\pgfsetstrokecolor{currentstroke}%
\pgfsetdash{}{0pt}%
\pgfsys@defobject{currentmarker}{\pgfqpoint{0.000000in}{-0.048611in}}{\pgfqpoint{0.000000in}{0.000000in}}{%
\pgfpathmoveto{\pgfqpoint{0.000000in}{0.000000in}}%
\pgfpathlineto{\pgfqpoint{0.000000in}{-0.048611in}}%
\pgfusepath{stroke,fill}%
}%
\begin{pgfscope}%
\pgfsys@transformshift{3.529799in}{1.869553in}%
\pgfsys@useobject{currentmarker}{}%
\end{pgfscope}%
\end{pgfscope}%
\begin{pgfscope}%
\definecolor{textcolor}{rgb}{0.000000,0.000000,0.000000}%
\pgfsetstrokecolor{textcolor}%
\pgfsetfillcolor{textcolor}%
\pgftext[x=3.529799in,y=1.772331in,,top]{\color{textcolor}\rmfamily\fontsize{10.000000}{12.000000}\selectfont \(\displaystyle 800\)}%
\end{pgfscope}%
\begin{pgfscope}%
\pgfsetbuttcap%
\pgfsetroundjoin%
\definecolor{currentfill}{rgb}{0.000000,0.000000,0.000000}%
\pgfsetfillcolor{currentfill}%
\pgfsetlinewidth{0.803000pt}%
\definecolor{currentstroke}{rgb}{0.000000,0.000000,0.000000}%
\pgfsetstrokecolor{currentstroke}%
\pgfsetdash{}{0pt}%
\pgfsys@defobject{currentmarker}{\pgfqpoint{0.000000in}{-0.048611in}}{\pgfqpoint{0.000000in}{0.000000in}}{%
\pgfpathmoveto{\pgfqpoint{0.000000in}{0.000000in}}%
\pgfpathlineto{\pgfqpoint{0.000000in}{-0.048611in}}%
\pgfusepath{stroke,fill}%
}%
\begin{pgfscope}%
\pgfsys@transformshift{4.389398in}{1.869553in}%
\pgfsys@useobject{currentmarker}{}%
\end{pgfscope}%
\end{pgfscope}%
\begin{pgfscope}%
\definecolor{textcolor}{rgb}{0.000000,0.000000,0.000000}%
\pgfsetstrokecolor{textcolor}%
\pgfsetfillcolor{textcolor}%
\pgftext[x=4.389398in,y=1.772331in,,top]{\color{textcolor}\rmfamily\fontsize{10.000000}{12.000000}\selectfont \(\displaystyle 1000\)}%
\end{pgfscope}%
\begin{pgfscope}%
\pgfsetbuttcap%
\pgfsetroundjoin%
\definecolor{currentfill}{rgb}{0.000000,0.000000,0.000000}%
\pgfsetfillcolor{currentfill}%
\pgfsetlinewidth{0.803000pt}%
\definecolor{currentstroke}{rgb}{0.000000,0.000000,0.000000}%
\pgfsetstrokecolor{currentstroke}%
\pgfsetdash{}{0pt}%
\pgfsys@defobject{currentmarker}{\pgfqpoint{-0.048611in}{0.000000in}}{\pgfqpoint{0.000000in}{0.000000in}}{%
\pgfpathmoveto{\pgfqpoint{0.000000in}{0.000000in}}%
\pgfpathlineto{\pgfqpoint{-0.048611in}{0.000000in}}%
\pgfusepath{stroke,fill}%
}%
\begin{pgfscope}%
\pgfsys@transformshift{3.025000in}{1.926761in}%
\pgfsys@useobject{currentmarker}{}%
\end{pgfscope}%
\end{pgfscope}%
\begin{pgfscope}%
\definecolor{textcolor}{rgb}{0.000000,0.000000,0.000000}%
\pgfsetstrokecolor{textcolor}%
\pgfsetfillcolor{textcolor}%
\pgftext[x=2.572838in,y=1.878567in,left,base]{\color{textcolor}\rmfamily\fontsize{10.000000}{12.000000}\selectfont \(\displaystyle -0.05\)}%
\end{pgfscope}%
\begin{pgfscope}%
\pgfsetbuttcap%
\pgfsetroundjoin%
\definecolor{currentfill}{rgb}{0.000000,0.000000,0.000000}%
\pgfsetfillcolor{currentfill}%
\pgfsetlinewidth{0.803000pt}%
\definecolor{currentstroke}{rgb}{0.000000,0.000000,0.000000}%
\pgfsetstrokecolor{currentstroke}%
\pgfsetdash{}{0pt}%
\pgfsys@defobject{currentmarker}{\pgfqpoint{-0.048611in}{0.000000in}}{\pgfqpoint{0.000000in}{0.000000in}}{%
\pgfpathmoveto{\pgfqpoint{0.000000in}{0.000000in}}%
\pgfpathlineto{\pgfqpoint{-0.048611in}{0.000000in}}%
\pgfusepath{stroke,fill}%
}%
\begin{pgfscope}%
\pgfsys@transformshift{3.025000in}{2.247441in}%
\pgfsys@useobject{currentmarker}{}%
\end{pgfscope}%
\end{pgfscope}%
\begin{pgfscope}%
\definecolor{textcolor}{rgb}{0.000000,0.000000,0.000000}%
\pgfsetstrokecolor{textcolor}%
\pgfsetfillcolor{textcolor}%
\pgftext[x=2.680863in,y=2.199246in,left,base]{\color{textcolor}\rmfamily\fontsize{10.000000}{12.000000}\selectfont \(\displaystyle 0.00\)}%
\end{pgfscope}%
\begin{pgfscope}%
\pgfsetbuttcap%
\pgfsetroundjoin%
\definecolor{currentfill}{rgb}{0.000000,0.000000,0.000000}%
\pgfsetfillcolor{currentfill}%
\pgfsetlinewidth{0.803000pt}%
\definecolor{currentstroke}{rgb}{0.000000,0.000000,0.000000}%
\pgfsetstrokecolor{currentstroke}%
\pgfsetdash{}{0pt}%
\pgfsys@defobject{currentmarker}{\pgfqpoint{-0.048611in}{0.000000in}}{\pgfqpoint{0.000000in}{0.000000in}}{%
\pgfpathmoveto{\pgfqpoint{0.000000in}{0.000000in}}%
\pgfpathlineto{\pgfqpoint{-0.048611in}{0.000000in}}%
\pgfusepath{stroke,fill}%
}%
\begin{pgfscope}%
\pgfsys@transformshift{3.025000in}{2.568120in}%
\pgfsys@useobject{currentmarker}{}%
\end{pgfscope}%
\end{pgfscope}%
\begin{pgfscope}%
\definecolor{textcolor}{rgb}{0.000000,0.000000,0.000000}%
\pgfsetstrokecolor{textcolor}%
\pgfsetfillcolor{textcolor}%
\pgftext[x=2.680863in,y=2.519926in,left,base]{\color{textcolor}\rmfamily\fontsize{10.000000}{12.000000}\selectfont \(\displaystyle 0.05\)}%
\end{pgfscope}%
\begin{pgfscope}%
\pgfpathrectangle{\pgfqpoint{3.025000in}{1.869553in}}{\pgfqpoint{1.650000in}{0.849797in}}%
\pgfusepath{clip}%
\pgfsetrectcap%
\pgfsetroundjoin%
\pgfsetlinewidth{1.505625pt}%
\definecolor{currentstroke}{rgb}{0.121569,0.466667,0.705882}%
\pgfsetstrokecolor{currentstroke}%
\pgfsetdash{}{0pt}%
\pgfpathmoveto{\pgfqpoint{3.100000in}{2.247441in}}%
\pgfpathlineto{\pgfqpoint{4.600000in}{2.247441in}}%
\pgfpathlineto{\pgfqpoint{4.600000in}{2.247441in}}%
\pgfusepath{stroke}%
\end{pgfscope}%
\begin{pgfscope}%
\pgfpathrectangle{\pgfqpoint{3.025000in}{1.869553in}}{\pgfqpoint{1.650000in}{0.849797in}}%
\pgfusepath{clip}%
\pgfsetbuttcap%
\pgfsetroundjoin%
\pgfsetlinewidth{1.505625pt}%
\definecolor{currentstroke}{rgb}{1.000000,0.498039,0.054902}%
\pgfsetstrokecolor{currentstroke}%
\pgfsetdash{{5.550000pt}{2.400000pt}}{0.000000pt}%
\pgfpathmoveto{\pgfqpoint{3.100000in}{2.247441in}}%
\pgfpathlineto{\pgfqpoint{4.600000in}{2.247441in}}%
\pgfpathlineto{\pgfqpoint{4.600000in}{2.247441in}}%
\pgfusepath{stroke}%
\end{pgfscope}%
\begin{pgfscope}%
\pgfpathrectangle{\pgfqpoint{3.025000in}{1.869553in}}{\pgfqpoint{1.650000in}{0.849797in}}%
\pgfusepath{clip}%
\pgfsetbuttcap%
\pgfsetroundjoin%
\pgfsetlinewidth{1.505625pt}%
\definecolor{currentstroke}{rgb}{0.172549,0.627451,0.172549}%
\pgfsetstrokecolor{currentstroke}%
\pgfsetdash{{1.500000pt}{2.475000pt}}{0.000000pt}%
\pgfpathmoveto{\pgfqpoint{3.100000in}{2.247441in}}%
\pgfpathlineto{\pgfqpoint{4.600000in}{2.247441in}}%
\pgfpathlineto{\pgfqpoint{4.600000in}{2.247441in}}%
\pgfusepath{stroke}%
\end{pgfscope}%
\begin{pgfscope}%
\pgfpathrectangle{\pgfqpoint{3.025000in}{1.869553in}}{\pgfqpoint{1.650000in}{0.849797in}}%
\pgfusepath{clip}%
\pgfsetbuttcap%
\pgfsetroundjoin%
\pgfsetlinewidth{1.505625pt}%
\definecolor{currentstroke}{rgb}{0.839216,0.152941,0.156863}%
\pgfsetstrokecolor{currentstroke}%
\pgfsetdash{{9.600000pt}{2.400000pt}{1.500000pt}{2.400000pt}}{0.000000pt}%
\pgfpathmoveto{\pgfqpoint{3.100000in}{2.415322in}}%
\pgfpathlineto{\pgfqpoint{3.104298in}{2.299341in}}%
\pgfpathlineto{\pgfqpoint{3.108596in}{2.152323in}}%
\pgfpathlineto{\pgfqpoint{3.112894in}{2.061996in}}%
\pgfpathlineto{\pgfqpoint{3.117192in}{2.081823in}}%
\pgfpathlineto{\pgfqpoint{3.130086in}{2.433981in}}%
\pgfpathlineto{\pgfqpoint{3.134384in}{2.412912in}}%
\pgfpathlineto{\pgfqpoint{3.147278in}{2.060886in}}%
\pgfpathlineto{\pgfqpoint{3.151576in}{2.079897in}}%
\pgfpathlineto{\pgfqpoint{3.164470in}{2.432776in}}%
\pgfpathlineto{\pgfqpoint{3.168768in}{2.419117in}}%
\pgfpathlineto{\pgfqpoint{3.173066in}{2.307945in}}%
\pgfpathlineto{\pgfqpoint{3.177364in}{2.162529in}}%
\pgfpathlineto{\pgfqpoint{3.181662in}{2.065002in}}%
\pgfpathlineto{\pgfqpoint{3.185960in}{2.070006in}}%
\pgfpathlineto{\pgfqpoint{3.190258in}{2.174373in}}%
\pgfpathlineto{\pgfqpoint{3.194556in}{2.319512in}}%
\pgfpathlineto{\pgfqpoint{3.198854in}{2.424579in}}%
\pgfpathlineto{\pgfqpoint{3.203152in}{2.431511in}}%
\pgfpathlineto{\pgfqpoint{3.207450in}{2.336837in}}%
\pgfpathlineto{\pgfqpoint{3.216046in}{2.078988in}}%
\pgfpathlineto{\pgfqpoint{3.220344in}{2.056957in}}%
\pgfpathlineto{\pgfqpoint{3.224642in}{2.138549in}}%
\pgfpathlineto{\pgfqpoint{3.233238in}{2.402643in}}%
\pgfpathlineto{\pgfqpoint{3.237536in}{2.442626in}}%
\pgfpathlineto{\pgfqpoint{3.241834in}{2.377990in}}%
\pgfpathlineto{\pgfqpoint{3.250430in}{2.111313in}}%
\pgfpathlineto{\pgfqpoint{3.254728in}{2.051132in}}%
\pgfpathlineto{\pgfqpoint{3.259026in}{2.094603in}}%
\pgfpathlineto{\pgfqpoint{3.271920in}{2.439101in}}%
\pgfpathlineto{\pgfqpoint{3.276218in}{2.421026in}}%
\pgfpathlineto{\pgfqpoint{3.280516in}{2.313034in}}%
\pgfpathlineto{\pgfqpoint{3.284814in}{2.171196in}}%
\pgfpathlineto{\pgfqpoint{3.289112in}{2.068548in}}%
\pgfpathlineto{\pgfqpoint{3.293410in}{2.057492in}}%
\pgfpathlineto{\pgfqpoint{3.297708in}{2.143311in}}%
\pgfpathlineto{\pgfqpoint{3.306304in}{2.403243in}}%
\pgfpathlineto{\pgfqpoint{3.310602in}{2.445934in}}%
\pgfpathlineto{\pgfqpoint{3.314900in}{2.388982in}}%
\pgfpathlineto{\pgfqpoint{3.323496in}{2.126655in}}%
\pgfpathlineto{\pgfqpoint{3.327794in}{2.051960in}}%
\pgfpathlineto{\pgfqpoint{3.332092in}{2.073833in}}%
\pgfpathlineto{\pgfqpoint{3.336390in}{2.181121in}}%
\pgfpathlineto{\pgfqpoint{3.344986in}{2.424839in}}%
\pgfpathlineto{\pgfqpoint{3.349284in}{2.442557in}}%
\pgfpathlineto{\pgfqpoint{3.353582in}{2.365839in}}%
\pgfpathlineto{\pgfqpoint{3.362178in}{2.105653in}}%
\pgfpathlineto{\pgfqpoint{3.366476in}{2.047027in}}%
\pgfpathlineto{\pgfqpoint{3.370774in}{2.083799in}}%
\pgfpathlineto{\pgfqpoint{3.383668in}{2.431852in}}%
\pgfpathlineto{\pgfqpoint{3.387966in}{2.441653in}}%
\pgfpathlineto{\pgfqpoint{3.392264in}{2.360952in}}%
\pgfpathlineto{\pgfqpoint{3.400860in}{2.103386in}}%
\pgfpathlineto{\pgfqpoint{3.405158in}{2.045455in}}%
\pgfpathlineto{\pgfqpoint{3.409456in}{2.080035in}}%
\pgfpathlineto{\pgfqpoint{3.422350in}{2.427883in}}%
\pgfpathlineto{\pgfqpoint{3.426648in}{2.447054in}}%
\pgfpathlineto{\pgfqpoint{3.430946in}{2.376798in}}%
\pgfpathlineto{\pgfqpoint{3.439542in}{2.120274in}}%
\pgfpathlineto{\pgfqpoint{3.443840in}{2.047907in}}%
\pgfpathlineto{\pgfqpoint{3.448138in}{2.063335in}}%
\pgfpathlineto{\pgfqpoint{3.452436in}{2.159401in}}%
\pgfpathlineto{\pgfqpoint{3.461032in}{2.408161in}}%
\pgfpathlineto{\pgfqpoint{3.465330in}{2.453038in}}%
\pgfpathlineto{\pgfqpoint{3.469628in}{2.409509in}}%
\pgfpathlineto{\pgfqpoint{3.482521in}{2.064828in}}%
\pgfpathlineto{\pgfqpoint{3.486819in}{2.043948in}}%
\pgfpathlineto{\pgfqpoint{3.491117in}{2.108535in}}%
\pgfpathlineto{\pgfqpoint{3.499713in}{2.360596in}}%
\pgfpathlineto{\pgfqpoint{3.504011in}{2.443162in}}%
\pgfpathlineto{\pgfqpoint{3.508309in}{2.445090in}}%
\pgfpathlineto{\pgfqpoint{3.512607in}{2.365983in}}%
\pgfpathlineto{\pgfqpoint{3.521203in}{2.114663in}}%
\pgfpathlineto{\pgfqpoint{3.525501in}{2.044458in}}%
\pgfpathlineto{\pgfqpoint{3.529799in}{2.055852in}}%
\pgfpathlineto{\pgfqpoint{3.534097in}{2.143882in}}%
\pgfpathlineto{\pgfqpoint{3.542693in}{2.392202in}}%
\pgfpathlineto{\pgfqpoint{3.546991in}{2.454240in}}%
\pgfpathlineto{\pgfqpoint{3.551289in}{2.435212in}}%
\pgfpathlineto{\pgfqpoint{3.555587in}{2.342954in}}%
\pgfpathlineto{\pgfqpoint{3.564183in}{2.097368in}}%
\pgfpathlineto{\pgfqpoint{3.568481in}{2.038727in}}%
\pgfpathlineto{\pgfqpoint{3.572779in}{2.059814in}}%
\pgfpathlineto{\pgfqpoint{3.577077in}{2.152200in}}%
\pgfpathlineto{\pgfqpoint{3.585673in}{2.396510in}}%
\pgfpathlineto{\pgfqpoint{3.589971in}{2.456682in}}%
\pgfpathlineto{\pgfqpoint{3.594269in}{2.439016in}}%
\pgfpathlineto{\pgfqpoint{3.598567in}{2.350467in}}%
\pgfpathlineto{\pgfqpoint{3.607163in}{2.106119in}}%
\pgfpathlineto{\pgfqpoint{3.611461in}{2.039675in}}%
\pgfpathlineto{\pgfqpoint{3.615759in}{2.048442in}}%
\pgfpathlineto{\pgfqpoint{3.620057in}{2.128849in}}%
\pgfpathlineto{\pgfqpoint{3.628653in}{2.373053in}}%
\pgfpathlineto{\pgfqpoint{3.632951in}{2.449928in}}%
\pgfpathlineto{\pgfqpoint{3.637249in}{2.455528in}}%
\pgfpathlineto{\pgfqpoint{3.641547in}{2.388297in}}%
\pgfpathlineto{\pgfqpoint{3.654441in}{2.056975in}}%
\pgfpathlineto{\pgfqpoint{3.658739in}{2.031805in}}%
\pgfpathlineto{\pgfqpoint{3.663037in}{2.079951in}}%
\pgfpathlineto{\pgfqpoint{3.675931in}{2.415303in}}%
\pgfpathlineto{\pgfqpoint{3.680229in}{2.464332in}}%
\pgfpathlineto{\pgfqpoint{3.684527in}{2.441786in}}%
\pgfpathlineto{\pgfqpoint{3.688825in}{2.355496in}}%
\pgfpathlineto{\pgfqpoint{3.697421in}{2.116824in}}%
\pgfpathlineto{\pgfqpoint{3.701719in}{2.041727in}}%
\pgfpathlineto{\pgfqpoint{3.706017in}{2.032459in}}%
\pgfpathlineto{\pgfqpoint{3.710315in}{2.091555in}}%
\pgfpathlineto{\pgfqpoint{3.723209in}{2.422837in}}%
\pgfpathlineto{\pgfqpoint{3.727507in}{2.468110in}}%
\pgfpathlineto{\pgfqpoint{3.731805in}{2.445417in}}%
\pgfpathlineto{\pgfqpoint{3.736103in}{2.362144in}}%
\pgfpathlineto{\pgfqpoint{3.744699in}{2.126896in}}%
\pgfpathlineto{\pgfqpoint{3.748997in}{2.045955in}}%
\pgfpathlineto{\pgfqpoint{3.753295in}{2.025057in}}%
\pgfpathlineto{\pgfqpoint{3.757593in}{2.069984in}}%
\pgfpathlineto{\pgfqpoint{3.766189in}{2.287648in}}%
\pgfpathlineto{\pgfqpoint{3.770487in}{2.396551in}}%
\pgfpathlineto{\pgfqpoint{3.774785in}{2.462511in}}%
\pgfpathlineto{\pgfqpoint{3.779083in}{2.466961in}}%
\pgfpathlineto{\pgfqpoint{3.783381in}{2.409067in}}%
\pgfpathlineto{\pgfqpoint{3.796275in}{2.083208in}}%
\pgfpathlineto{\pgfqpoint{3.800573in}{2.025906in}}%
\pgfpathlineto{\pgfqpoint{3.804871in}{2.029304in}}%
\pgfpathlineto{\pgfqpoint{3.809169in}{2.092036in}}%
\pgfpathlineto{\pgfqpoint{3.822063in}{2.415318in}}%
\pgfpathlineto{\pgfqpoint{3.826361in}{2.471307in}}%
\pgfpathlineto{\pgfqpoint{3.830659in}{2.468644in}}%
\pgfpathlineto{\pgfqpoint{3.834957in}{2.408467in}}%
\pgfpathlineto{\pgfqpoint{3.852149in}{2.025583in}}%
\pgfpathlineto{\pgfqpoint{3.856447in}{2.019062in}}%
\pgfpathlineto{\pgfqpoint{3.860745in}{2.069178in}}%
\pgfpathlineto{\pgfqpoint{3.869341in}{2.277684in}}%
\pgfpathlineto{\pgfqpoint{3.873639in}{2.384901in}}%
\pgfpathlineto{\pgfqpoint{3.877937in}{2.459129in}}%
\pgfpathlineto{\pgfqpoint{3.882235in}{2.482974in}}%
\pgfpathlineto{\pgfqpoint{3.886533in}{2.451234in}}%
\pgfpathlineto{\pgfqpoint{3.890831in}{2.371806in}}%
\pgfpathlineto{\pgfqpoint{3.903725in}{2.061429in}}%
\pgfpathlineto{\pgfqpoint{3.908023in}{2.013352in}}%
\pgfpathlineto{\pgfqpoint{3.912321in}{2.017710in}}%
\pgfpathlineto{\pgfqpoint{3.916619in}{2.073069in}}%
\pgfpathlineto{\pgfqpoint{3.925215in}{2.278092in}}%
\pgfpathlineto{\pgfqpoint{3.929513in}{2.382774in}}%
\pgfpathlineto{\pgfqpoint{3.933811in}{2.458494in}}%
\pgfpathlineto{\pgfqpoint{3.938109in}{2.489471in}}%
\pgfpathlineto{\pgfqpoint{3.942407in}{2.469622in}}%
\pgfpathlineto{\pgfqpoint{3.946705in}{2.403559in}}%
\pgfpathlineto{\pgfqpoint{3.963897in}{2.026626in}}%
\pgfpathlineto{\pgfqpoint{3.968195in}{2.001174in}}%
\pgfpathlineto{\pgfqpoint{3.972493in}{2.023990in}}%
\pgfpathlineto{\pgfqpoint{3.976791in}{2.090155in}}%
\pgfpathlineto{\pgfqpoint{3.993983in}{2.465481in}}%
\pgfpathlineto{\pgfqpoint{3.998281in}{2.497283in}}%
\pgfpathlineto{\pgfqpoint{4.002579in}{2.483614in}}%
\pgfpathlineto{\pgfqpoint{4.006877in}{2.427434in}}%
\pgfpathlineto{\pgfqpoint{4.015473in}{2.234844in}}%
\pgfpathlineto{\pgfqpoint{4.024069in}{2.050227in}}%
\pgfpathlineto{\pgfqpoint{4.028367in}{2.001315in}}%
\pgfpathlineto{\pgfqpoint{4.032665in}{1.993752in}}%
\pgfpathlineto{\pgfqpoint{4.036963in}{2.028302in}}%
\pgfpathlineto{\pgfqpoint{4.041261in}{2.098791in}}%
\pgfpathlineto{\pgfqpoint{4.058453in}{2.464941in}}%
\pgfpathlineto{\pgfqpoint{4.062751in}{2.504347in}}%
\pgfpathlineto{\pgfqpoint{4.067049in}{2.504702in}}%
\pgfpathlineto{\pgfqpoint{4.071347in}{2.466469in}}%
\pgfpathlineto{\pgfqpoint{4.075645in}{2.395819in}}%
\pgfpathlineto{\pgfqpoint{4.092837in}{2.034312in}}%
\pgfpathlineto{\pgfqpoint{4.097135in}{1.989009in}}%
\pgfpathlineto{\pgfqpoint{4.101433in}{1.978852in}}%
\pgfpathlineto{\pgfqpoint{4.105731in}{2.004687in}}%
\pgfpathlineto{\pgfqpoint{4.110029in}{2.062565in}}%
\pgfpathlineto{\pgfqpoint{4.118625in}{2.239568in}}%
\pgfpathlineto{\pgfqpoint{4.127221in}{2.420796in}}%
\pgfpathlineto{\pgfqpoint{4.131519in}{2.484792in}}%
\pgfpathlineto{\pgfqpoint{4.135817in}{2.520305in}}%
\pgfpathlineto{\pgfqpoint{4.140115in}{2.523621in}}%
\pgfpathlineto{\pgfqpoint{4.144413in}{2.494900in}}%
\pgfpathlineto{\pgfqpoint{4.148711in}{2.437968in}}%
\pgfpathlineto{\pgfqpoint{4.157307in}{2.269078in}}%
\pgfpathlineto{\pgfqpoint{4.165903in}{2.090726in}}%
\pgfpathlineto{\pgfqpoint{4.170201in}{2.021628in}}%
\pgfpathlineto{\pgfqpoint{4.174499in}{1.975563in}}%
\pgfpathlineto{\pgfqpoint{4.178797in}{1.956686in}}%
\pgfpathlineto{\pgfqpoint{4.183095in}{1.966303in}}%
\pgfpathlineto{\pgfqpoint{4.187393in}{2.002909in}}%
\pgfpathlineto{\pgfqpoint{4.191691in}{2.062501in}}%
\pgfpathlineto{\pgfqpoint{4.200287in}{2.225464in}}%
\pgfpathlineto{\pgfqpoint{4.208883in}{2.396341in}}%
\pgfpathlineto{\pgfqpoint{4.217479in}{2.518545in}}%
\pgfpathlineto{\pgfqpoint{4.221777in}{2.548980in}}%
\pgfpathlineto{\pgfqpoint{4.226074in}{2.555895in}}%
\pgfpathlineto{\pgfqpoint{4.230372in}{2.539367in}}%
\pgfpathlineto{\pgfqpoint{4.234670in}{2.501237in}}%
\pgfpathlineto{\pgfqpoint{4.243266in}{2.374613in}}%
\pgfpathlineto{\pgfqpoint{4.264756in}{1.999281in}}%
\pgfpathlineto{\pgfqpoint{4.269054in}{1.952014in}}%
\pgfpathlineto{\pgfqpoint{4.273352in}{1.921291in}}%
\pgfpathlineto{\pgfqpoint{4.277650in}{1.908180in}}%
\pgfpathlineto{\pgfqpoint{4.281948in}{1.912711in}}%
\pgfpathlineto{\pgfqpoint{4.286246in}{1.933978in}}%
\pgfpathlineto{\pgfqpoint{4.290544in}{1.970291in}}%
\pgfpathlineto{\pgfqpoint{4.299140in}{2.078452in}}%
\pgfpathlineto{\pgfqpoint{4.329226in}{2.537570in}}%
\pgfpathlineto{\pgfqpoint{4.337822in}{2.620017in}}%
\pgfpathlineto{\pgfqpoint{4.342120in}{2.647934in}}%
\pgfpathlineto{\pgfqpoint{4.346418in}{2.667040in}}%
\pgfpathlineto{\pgfqpoint{4.350716in}{2.677754in}}%
\pgfpathlineto{\pgfqpoint{4.355014in}{2.680722in}}%
\pgfpathlineto{\pgfqpoint{4.359312in}{2.676758in}}%
\pgfpathlineto{\pgfqpoint{4.363610in}{2.666784in}}%
\pgfpathlineto{\pgfqpoint{4.367908in}{2.651777in}}%
\pgfpathlineto{\pgfqpoint{4.376504in}{2.610595in}}%
\pgfpathlineto{\pgfqpoint{4.393696in}{2.508032in}}%
\pgfpathlineto{\pgfqpoint{4.406590in}{2.433855in}}%
\pgfpathlineto{\pgfqpoint{4.419484in}{2.372625in}}%
\pgfpathlineto{\pgfqpoint{4.428080in}{2.340440in}}%
\pgfpathlineto{\pgfqpoint{4.436676in}{2.314923in}}%
\pgfpathlineto{\pgfqpoint{4.445272in}{2.295332in}}%
\pgfpathlineto{\pgfqpoint{4.453868in}{2.280721in}}%
\pgfpathlineto{\pgfqpoint{4.462464in}{2.270108in}}%
\pgfpathlineto{\pgfqpoint{4.471060in}{2.262586in}}%
\pgfpathlineto{\pgfqpoint{4.479656in}{2.257376in}}%
\pgfpathlineto{\pgfqpoint{4.492550in}{2.252548in}}%
\pgfpathlineto{\pgfqpoint{4.509742in}{2.249424in}}%
\pgfpathlineto{\pgfqpoint{4.535530in}{2.247868in}}%
\pgfpathlineto{\pgfqpoint{4.600000in}{2.247446in}}%
\pgfpathlineto{\pgfqpoint{4.600000in}{2.247446in}}%
\pgfusepath{stroke}%
\end{pgfscope}%
\begin{pgfscope}%
\pgfsetrectcap%
\pgfsetmiterjoin%
\pgfsetlinewidth{0.803000pt}%
\definecolor{currentstroke}{rgb}{0.000000,0.000000,0.000000}%
\pgfsetstrokecolor{currentstroke}%
\pgfsetdash{}{0pt}%
\pgfpathmoveto{\pgfqpoint{3.025000in}{1.869553in}}%
\pgfpathlineto{\pgfqpoint{3.025000in}{2.719350in}}%
\pgfusepath{stroke}%
\end{pgfscope}%
\begin{pgfscope}%
\pgfsetrectcap%
\pgfsetmiterjoin%
\pgfsetlinewidth{0.803000pt}%
\definecolor{currentstroke}{rgb}{0.000000,0.000000,0.000000}%
\pgfsetstrokecolor{currentstroke}%
\pgfsetdash{}{0pt}%
\pgfpathmoveto{\pgfqpoint{4.675000in}{1.869553in}}%
\pgfpathlineto{\pgfqpoint{4.675000in}{2.719350in}}%
\pgfusepath{stroke}%
\end{pgfscope}%
\begin{pgfscope}%
\pgfsetrectcap%
\pgfsetmiterjoin%
\pgfsetlinewidth{0.803000pt}%
\definecolor{currentstroke}{rgb}{0.000000,0.000000,0.000000}%
\pgfsetstrokecolor{currentstroke}%
\pgfsetdash{}{0pt}%
\pgfpathmoveto{\pgfqpoint{3.025000in}{1.869553in}}%
\pgfpathlineto{\pgfqpoint{4.675000in}{1.869553in}}%
\pgfusepath{stroke}%
\end{pgfscope}%
\begin{pgfscope}%
\pgfsetrectcap%
\pgfsetmiterjoin%
\pgfsetlinewidth{0.803000pt}%
\definecolor{currentstroke}{rgb}{0.000000,0.000000,0.000000}%
\pgfsetstrokecolor{currentstroke}%
\pgfsetdash{}{0pt}%
\pgfpathmoveto{\pgfqpoint{3.025000in}{2.719350in}}%
\pgfpathlineto{\pgfqpoint{4.675000in}{2.719350in}}%
\pgfusepath{stroke}%
\end{pgfscope}%
\end{pgfpicture}%
\makeatother%
\endgroup%

	\caption{Gráficas de los polinomios de Chebyshev $J_n(t)$ para $N=1050$ términos. Se puede observar cómo los valores del polinomio decaen monotónicamente a partir de $n \approx t$ para cada valor de $t$.}
	\label{fig:gn_Jn(t)}
\end{figure}
\fi

\begin{figure}[htb]
	\centering
	%% Creator: Matplotlib, PGF backend
%%
%% To include the figure in your LaTeX document, write
%%   \input{<filename>.pgf}
%%
%% Make sure the required packages are loaded in your preamble
%%   \usepackage{pgf}
%%
%% Figures using additional raster images can only be included by \input if
%% they are in the same directory as the main LaTeX file. For loading figures
%% from other directories you can use the `import` package
%%   \usepackage{import}
%% and then include the figures with
%%   \import{<path to file>}{<filename>.pgf}
%%
%% Matplotlib used the following preamble
%%   \usepackage[utf8x]{inputenc}
%%   \usepackage[T1]{fontenc}
%%   \usepackage{fontspec}
%%
\begingroup%
\makeatletter%
\begin{pgfpicture}%
\pgfpathrectangle{\pgfpointorigin}{\pgfqpoint{5.500000in}{2.039512in}}%
\pgfusepath{use as bounding box, clip}%
\begin{pgfscope}%
\pgfsetbuttcap%
\pgfsetmiterjoin%
\definecolor{currentfill}{rgb}{1.000000,1.000000,1.000000}%
\pgfsetfillcolor{currentfill}%
\pgfsetlinewidth{0.000000pt}%
\definecolor{currentstroke}{rgb}{1.000000,1.000000,1.000000}%
\pgfsetstrokecolor{currentstroke}%
\pgfsetdash{}{0pt}%
\pgfpathmoveto{\pgfqpoint{0.000000in}{0.000000in}}%
\pgfpathlineto{\pgfqpoint{5.500000in}{0.000000in}}%
\pgfpathlineto{\pgfqpoint{5.500000in}{2.039512in}}%
\pgfpathlineto{\pgfqpoint{0.000000in}{2.039512in}}%
\pgfpathclose%
\pgfusepath{fill}%
\end{pgfscope}%
\begin{pgfscope}%
\pgfsetbuttcap%
\pgfsetmiterjoin%
\definecolor{currentfill}{rgb}{1.000000,1.000000,1.000000}%
\pgfsetfillcolor{currentfill}%
\pgfsetlinewidth{0.000000pt}%
\definecolor{currentstroke}{rgb}{0.000000,0.000000,0.000000}%
\pgfsetstrokecolor{currentstroke}%
\pgfsetstrokeopacity{0.000000}%
\pgfsetdash{}{0pt}%
\pgfpathmoveto{\pgfqpoint{0.674540in}{0.399444in}}%
\pgfpathlineto{\pgfqpoint{5.465000in}{0.399444in}}%
\pgfpathlineto{\pgfqpoint{5.465000in}{2.004512in}}%
\pgfpathlineto{\pgfqpoint{0.674540in}{2.004512in}}%
\pgfpathclose%
\pgfusepath{fill}%
\end{pgfscope}%
\begin{pgfscope}%
\pgfsetbuttcap%
\pgfsetroundjoin%
\definecolor{currentfill}{rgb}{0.000000,0.000000,0.000000}%
\pgfsetfillcolor{currentfill}%
\pgfsetlinewidth{0.803000pt}%
\definecolor{currentstroke}{rgb}{0.000000,0.000000,0.000000}%
\pgfsetstrokecolor{currentstroke}%
\pgfsetdash{}{0pt}%
\pgfsys@defobject{currentmarker}{\pgfqpoint{0.000000in}{-0.048611in}}{\pgfqpoint{0.000000in}{0.000000in}}{%
\pgfpathmoveto{\pgfqpoint{0.000000in}{0.000000in}}%
\pgfpathlineto{\pgfqpoint{0.000000in}{-0.048611in}}%
\pgfusepath{stroke,fill}%
}%
\begin{pgfscope}%
\pgfsys@transformshift{0.892288in}{0.399444in}%
\pgfsys@useobject{currentmarker}{}%
\end{pgfscope}%
\end{pgfscope}%
\begin{pgfscope}%
\definecolor{textcolor}{rgb}{0.000000,0.000000,0.000000}%
\pgfsetstrokecolor{textcolor}%
\pgfsetfillcolor{textcolor}%
\pgftext[x=0.892288in,y=0.302222in,,top]{\color{textcolor}\rmfamily\fontsize{10.000000}{12.000000}\selectfont \(\displaystyle 1.0\)}%
\end{pgfscope}%
\begin{pgfscope}%
\pgfsetbuttcap%
\pgfsetroundjoin%
\definecolor{currentfill}{rgb}{0.000000,0.000000,0.000000}%
\pgfsetfillcolor{currentfill}%
\pgfsetlinewidth{0.803000pt}%
\definecolor{currentstroke}{rgb}{0.000000,0.000000,0.000000}%
\pgfsetstrokecolor{currentstroke}%
\pgfsetdash{}{0pt}%
\pgfsys@defobject{currentmarker}{\pgfqpoint{0.000000in}{-0.048611in}}{\pgfqpoint{0.000000in}{0.000000in}}{%
\pgfpathmoveto{\pgfqpoint{0.000000in}{0.000000in}}%
\pgfpathlineto{\pgfqpoint{0.000000in}{-0.048611in}}%
\pgfusepath{stroke,fill}%
}%
\begin{pgfscope}%
\pgfsys@transformshift{1.436658in}{0.399444in}%
\pgfsys@useobject{currentmarker}{}%
\end{pgfscope}%
\end{pgfscope}%
\begin{pgfscope}%
\definecolor{textcolor}{rgb}{0.000000,0.000000,0.000000}%
\pgfsetstrokecolor{textcolor}%
\pgfsetfillcolor{textcolor}%
\pgftext[x=1.436658in,y=0.302222in,,top]{\color{textcolor}\rmfamily\fontsize{10.000000}{12.000000}\selectfont \(\displaystyle 1.5\)}%
\end{pgfscope}%
\begin{pgfscope}%
\pgfsetbuttcap%
\pgfsetroundjoin%
\definecolor{currentfill}{rgb}{0.000000,0.000000,0.000000}%
\pgfsetfillcolor{currentfill}%
\pgfsetlinewidth{0.803000pt}%
\definecolor{currentstroke}{rgb}{0.000000,0.000000,0.000000}%
\pgfsetstrokecolor{currentstroke}%
\pgfsetdash{}{0pt}%
\pgfsys@defobject{currentmarker}{\pgfqpoint{0.000000in}{-0.048611in}}{\pgfqpoint{0.000000in}{0.000000in}}{%
\pgfpathmoveto{\pgfqpoint{0.000000in}{0.000000in}}%
\pgfpathlineto{\pgfqpoint{0.000000in}{-0.048611in}}%
\pgfusepath{stroke,fill}%
}%
\begin{pgfscope}%
\pgfsys@transformshift{1.981029in}{0.399444in}%
\pgfsys@useobject{currentmarker}{}%
\end{pgfscope}%
\end{pgfscope}%
\begin{pgfscope}%
\definecolor{textcolor}{rgb}{0.000000,0.000000,0.000000}%
\pgfsetstrokecolor{textcolor}%
\pgfsetfillcolor{textcolor}%
\pgftext[x=1.981029in,y=0.302222in,,top]{\color{textcolor}\rmfamily\fontsize{10.000000}{12.000000}\selectfont \(\displaystyle 2.0\)}%
\end{pgfscope}%
\begin{pgfscope}%
\pgfsetbuttcap%
\pgfsetroundjoin%
\definecolor{currentfill}{rgb}{0.000000,0.000000,0.000000}%
\pgfsetfillcolor{currentfill}%
\pgfsetlinewidth{0.803000pt}%
\definecolor{currentstroke}{rgb}{0.000000,0.000000,0.000000}%
\pgfsetstrokecolor{currentstroke}%
\pgfsetdash{}{0pt}%
\pgfsys@defobject{currentmarker}{\pgfqpoint{0.000000in}{-0.048611in}}{\pgfqpoint{0.000000in}{0.000000in}}{%
\pgfpathmoveto{\pgfqpoint{0.000000in}{0.000000in}}%
\pgfpathlineto{\pgfqpoint{0.000000in}{-0.048611in}}%
\pgfusepath{stroke,fill}%
}%
\begin{pgfscope}%
\pgfsys@transformshift{2.525399in}{0.399444in}%
\pgfsys@useobject{currentmarker}{}%
\end{pgfscope}%
\end{pgfscope}%
\begin{pgfscope}%
\definecolor{textcolor}{rgb}{0.000000,0.000000,0.000000}%
\pgfsetstrokecolor{textcolor}%
\pgfsetfillcolor{textcolor}%
\pgftext[x=2.525399in,y=0.302222in,,top]{\color{textcolor}\rmfamily\fontsize{10.000000}{12.000000}\selectfont \(\displaystyle 2.5\)}%
\end{pgfscope}%
\begin{pgfscope}%
\pgfsetbuttcap%
\pgfsetroundjoin%
\definecolor{currentfill}{rgb}{0.000000,0.000000,0.000000}%
\pgfsetfillcolor{currentfill}%
\pgfsetlinewidth{0.803000pt}%
\definecolor{currentstroke}{rgb}{0.000000,0.000000,0.000000}%
\pgfsetstrokecolor{currentstroke}%
\pgfsetdash{}{0pt}%
\pgfsys@defobject{currentmarker}{\pgfqpoint{0.000000in}{-0.048611in}}{\pgfqpoint{0.000000in}{0.000000in}}{%
\pgfpathmoveto{\pgfqpoint{0.000000in}{0.000000in}}%
\pgfpathlineto{\pgfqpoint{0.000000in}{-0.048611in}}%
\pgfusepath{stroke,fill}%
}%
\begin{pgfscope}%
\pgfsys@transformshift{3.069770in}{0.399444in}%
\pgfsys@useobject{currentmarker}{}%
\end{pgfscope}%
\end{pgfscope}%
\begin{pgfscope}%
\definecolor{textcolor}{rgb}{0.000000,0.000000,0.000000}%
\pgfsetstrokecolor{textcolor}%
\pgfsetfillcolor{textcolor}%
\pgftext[x=3.069770in,y=0.302222in,,top]{\color{textcolor}\rmfamily\fontsize{10.000000}{12.000000}\selectfont \(\displaystyle 3.0\)}%
\end{pgfscope}%
\begin{pgfscope}%
\pgfsetbuttcap%
\pgfsetroundjoin%
\definecolor{currentfill}{rgb}{0.000000,0.000000,0.000000}%
\pgfsetfillcolor{currentfill}%
\pgfsetlinewidth{0.803000pt}%
\definecolor{currentstroke}{rgb}{0.000000,0.000000,0.000000}%
\pgfsetstrokecolor{currentstroke}%
\pgfsetdash{}{0pt}%
\pgfsys@defobject{currentmarker}{\pgfqpoint{0.000000in}{-0.048611in}}{\pgfqpoint{0.000000in}{0.000000in}}{%
\pgfpathmoveto{\pgfqpoint{0.000000in}{0.000000in}}%
\pgfpathlineto{\pgfqpoint{0.000000in}{-0.048611in}}%
\pgfusepath{stroke,fill}%
}%
\begin{pgfscope}%
\pgfsys@transformshift{3.614140in}{0.399444in}%
\pgfsys@useobject{currentmarker}{}%
\end{pgfscope}%
\end{pgfscope}%
\begin{pgfscope}%
\definecolor{textcolor}{rgb}{0.000000,0.000000,0.000000}%
\pgfsetstrokecolor{textcolor}%
\pgfsetfillcolor{textcolor}%
\pgftext[x=3.614140in,y=0.302222in,,top]{\color{textcolor}\rmfamily\fontsize{10.000000}{12.000000}\selectfont \(\displaystyle 3.5\)}%
\end{pgfscope}%
\begin{pgfscope}%
\pgfsetbuttcap%
\pgfsetroundjoin%
\definecolor{currentfill}{rgb}{0.000000,0.000000,0.000000}%
\pgfsetfillcolor{currentfill}%
\pgfsetlinewidth{0.803000pt}%
\definecolor{currentstroke}{rgb}{0.000000,0.000000,0.000000}%
\pgfsetstrokecolor{currentstroke}%
\pgfsetdash{}{0pt}%
\pgfsys@defobject{currentmarker}{\pgfqpoint{0.000000in}{-0.048611in}}{\pgfqpoint{0.000000in}{0.000000in}}{%
\pgfpathmoveto{\pgfqpoint{0.000000in}{0.000000in}}%
\pgfpathlineto{\pgfqpoint{0.000000in}{-0.048611in}}%
\pgfusepath{stroke,fill}%
}%
\begin{pgfscope}%
\pgfsys@transformshift{4.158511in}{0.399444in}%
\pgfsys@useobject{currentmarker}{}%
\end{pgfscope}%
\end{pgfscope}%
\begin{pgfscope}%
\definecolor{textcolor}{rgb}{0.000000,0.000000,0.000000}%
\pgfsetstrokecolor{textcolor}%
\pgfsetfillcolor{textcolor}%
\pgftext[x=4.158511in,y=0.302222in,,top]{\color{textcolor}\rmfamily\fontsize{10.000000}{12.000000}\selectfont \(\displaystyle 4.0\)}%
\end{pgfscope}%
\begin{pgfscope}%
\pgfsetbuttcap%
\pgfsetroundjoin%
\definecolor{currentfill}{rgb}{0.000000,0.000000,0.000000}%
\pgfsetfillcolor{currentfill}%
\pgfsetlinewidth{0.803000pt}%
\definecolor{currentstroke}{rgb}{0.000000,0.000000,0.000000}%
\pgfsetstrokecolor{currentstroke}%
\pgfsetdash{}{0pt}%
\pgfsys@defobject{currentmarker}{\pgfqpoint{0.000000in}{-0.048611in}}{\pgfqpoint{0.000000in}{0.000000in}}{%
\pgfpathmoveto{\pgfqpoint{0.000000in}{0.000000in}}%
\pgfpathlineto{\pgfqpoint{0.000000in}{-0.048611in}}%
\pgfusepath{stroke,fill}%
}%
\begin{pgfscope}%
\pgfsys@transformshift{4.702881in}{0.399444in}%
\pgfsys@useobject{currentmarker}{}%
\end{pgfscope}%
\end{pgfscope}%
\begin{pgfscope}%
\definecolor{textcolor}{rgb}{0.000000,0.000000,0.000000}%
\pgfsetstrokecolor{textcolor}%
\pgfsetfillcolor{textcolor}%
\pgftext[x=4.702881in,y=0.302222in,,top]{\color{textcolor}\rmfamily\fontsize{10.000000}{12.000000}\selectfont \(\displaystyle 4.5\)}%
\end{pgfscope}%
\begin{pgfscope}%
\pgfsetbuttcap%
\pgfsetroundjoin%
\definecolor{currentfill}{rgb}{0.000000,0.000000,0.000000}%
\pgfsetfillcolor{currentfill}%
\pgfsetlinewidth{0.803000pt}%
\definecolor{currentstroke}{rgb}{0.000000,0.000000,0.000000}%
\pgfsetstrokecolor{currentstroke}%
\pgfsetdash{}{0pt}%
\pgfsys@defobject{currentmarker}{\pgfqpoint{0.000000in}{-0.048611in}}{\pgfqpoint{0.000000in}{0.000000in}}{%
\pgfpathmoveto{\pgfqpoint{0.000000in}{0.000000in}}%
\pgfpathlineto{\pgfqpoint{0.000000in}{-0.048611in}}%
\pgfusepath{stroke,fill}%
}%
\begin{pgfscope}%
\pgfsys@transformshift{5.247252in}{0.399444in}%
\pgfsys@useobject{currentmarker}{}%
\end{pgfscope}%
\end{pgfscope}%
\begin{pgfscope}%
\definecolor{textcolor}{rgb}{0.000000,0.000000,0.000000}%
\pgfsetstrokecolor{textcolor}%
\pgfsetfillcolor{textcolor}%
\pgftext[x=5.247252in,y=0.302222in,,top]{\color{textcolor}\rmfamily\fontsize{10.000000}{12.000000}\selectfont \(\displaystyle 5.0\)}%
\end{pgfscope}%
\begin{pgfscope}%
\definecolor{textcolor}{rgb}{0.000000,0.000000,0.000000}%
\pgfsetstrokecolor{textcolor}%
\pgfsetfillcolor{textcolor}%
\pgftext[x=3.069770in,y=0.123333in,,top]{\color{textcolor}\rmfamily\fontsize{10.000000}{12.000000}\selectfont \(\displaystyle n\tilde{t}^{-1}\)}%
\end{pgfscope}%
\begin{pgfscope}%
\pgfsetbuttcap%
\pgfsetroundjoin%
\definecolor{currentfill}{rgb}{0.000000,0.000000,0.000000}%
\pgfsetfillcolor{currentfill}%
\pgfsetlinewidth{0.803000pt}%
\definecolor{currentstroke}{rgb}{0.000000,0.000000,0.000000}%
\pgfsetstrokecolor{currentstroke}%
\pgfsetdash{}{0pt}%
\pgfsys@defobject{currentmarker}{\pgfqpoint{-0.048611in}{0.000000in}}{\pgfqpoint{0.000000in}{0.000000in}}{%
\pgfpathmoveto{\pgfqpoint{0.000000in}{0.000000in}}%
\pgfpathlineto{\pgfqpoint{-0.048611in}{0.000000in}}%
\pgfusepath{stroke,fill}%
}%
\begin{pgfscope}%
\pgfsys@transformshift{0.674540in}{0.472172in}%
\pgfsys@useobject{currentmarker}{}%
\end{pgfscope}%
\end{pgfscope}%
\begin{pgfscope}%
\definecolor{textcolor}{rgb}{0.000000,0.000000,0.000000}%
\pgfsetstrokecolor{textcolor}%
\pgfsetfillcolor{textcolor}%
\pgftext[x=0.399848in,y=0.423978in,left,base]{\color{textcolor}\rmfamily\fontsize{10.000000}{12.000000}\selectfont \(\displaystyle 0.0\)}%
\end{pgfscope}%
\begin{pgfscope}%
\pgfsetbuttcap%
\pgfsetroundjoin%
\definecolor{currentfill}{rgb}{0.000000,0.000000,0.000000}%
\pgfsetfillcolor{currentfill}%
\pgfsetlinewidth{0.803000pt}%
\definecolor{currentstroke}{rgb}{0.000000,0.000000,0.000000}%
\pgfsetstrokecolor{currentstroke}%
\pgfsetdash{}{0pt}%
\pgfsys@defobject{currentmarker}{\pgfqpoint{-0.048611in}{0.000000in}}{\pgfqpoint{0.000000in}{0.000000in}}{%
\pgfpathmoveto{\pgfqpoint{0.000000in}{0.000000in}}%
\pgfpathlineto{\pgfqpoint{-0.048611in}{0.000000in}}%
\pgfusepath{stroke,fill}%
}%
\begin{pgfscope}%
\pgfsys@transformshift{0.674540in}{0.912994in}%
\pgfsys@useobject{currentmarker}{}%
\end{pgfscope}%
\end{pgfscope}%
\begin{pgfscope}%
\definecolor{textcolor}{rgb}{0.000000,0.000000,0.000000}%
\pgfsetstrokecolor{textcolor}%
\pgfsetfillcolor{textcolor}%
\pgftext[x=0.399848in,y=0.864800in,left,base]{\color{textcolor}\rmfamily\fontsize{10.000000}{12.000000}\selectfont \(\displaystyle 0.2\)}%
\end{pgfscope}%
\begin{pgfscope}%
\pgfsetbuttcap%
\pgfsetroundjoin%
\definecolor{currentfill}{rgb}{0.000000,0.000000,0.000000}%
\pgfsetfillcolor{currentfill}%
\pgfsetlinewidth{0.803000pt}%
\definecolor{currentstroke}{rgb}{0.000000,0.000000,0.000000}%
\pgfsetstrokecolor{currentstroke}%
\pgfsetdash{}{0pt}%
\pgfsys@defobject{currentmarker}{\pgfqpoint{-0.048611in}{0.000000in}}{\pgfqpoint{0.000000in}{0.000000in}}{%
\pgfpathmoveto{\pgfqpoint{0.000000in}{0.000000in}}%
\pgfpathlineto{\pgfqpoint{-0.048611in}{0.000000in}}%
\pgfusepath{stroke,fill}%
}%
\begin{pgfscope}%
\pgfsys@transformshift{0.674540in}{1.353816in}%
\pgfsys@useobject{currentmarker}{}%
\end{pgfscope}%
\end{pgfscope}%
\begin{pgfscope}%
\definecolor{textcolor}{rgb}{0.000000,0.000000,0.000000}%
\pgfsetstrokecolor{textcolor}%
\pgfsetfillcolor{textcolor}%
\pgftext[x=0.399848in,y=1.305621in,left,base]{\color{textcolor}\rmfamily\fontsize{10.000000}{12.000000}\selectfont \(\displaystyle 0.4\)}%
\end{pgfscope}%
\begin{pgfscope}%
\pgfsetbuttcap%
\pgfsetroundjoin%
\definecolor{currentfill}{rgb}{0.000000,0.000000,0.000000}%
\pgfsetfillcolor{currentfill}%
\pgfsetlinewidth{0.803000pt}%
\definecolor{currentstroke}{rgb}{0.000000,0.000000,0.000000}%
\pgfsetstrokecolor{currentstroke}%
\pgfsetdash{}{0pt}%
\pgfsys@defobject{currentmarker}{\pgfqpoint{-0.048611in}{0.000000in}}{\pgfqpoint{0.000000in}{0.000000in}}{%
\pgfpathmoveto{\pgfqpoint{0.000000in}{0.000000in}}%
\pgfpathlineto{\pgfqpoint{-0.048611in}{0.000000in}}%
\pgfusepath{stroke,fill}%
}%
\begin{pgfscope}%
\pgfsys@transformshift{0.674540in}{1.794637in}%
\pgfsys@useobject{currentmarker}{}%
\end{pgfscope}%
\end{pgfscope}%
\begin{pgfscope}%
\definecolor{textcolor}{rgb}{0.000000,0.000000,0.000000}%
\pgfsetstrokecolor{textcolor}%
\pgfsetfillcolor{textcolor}%
\pgftext[x=0.399848in,y=1.746443in,left,base]{\color{textcolor}\rmfamily\fontsize{10.000000}{12.000000}\selectfont \(\displaystyle 0.6\)}%
\end{pgfscope}%
\begin{pgfscope}%
\definecolor{textcolor}{rgb}{0.000000,0.000000,0.000000}%
\pgfsetstrokecolor{textcolor}%
\pgfsetfillcolor{textcolor}%
\pgftext[x=0.344292in,y=1.201978in,,bottom,rotate=90.000000]{\color{textcolor}\rmfamily\fontsize{10.000000}{12.000000}\selectfont \(\displaystyle \left|\frac{J_{n}(\tilde{t})}{\max\{J_{n}\}}\right|\)}%
\end{pgfscope}%
\begin{pgfscope}%
\pgfpathrectangle{\pgfqpoint{0.674540in}{0.399444in}}{\pgfqpoint{4.790460in}{1.605068in}}%
\pgfusepath{clip}%
\pgfsetrectcap%
\pgfsetroundjoin%
\pgfsetlinewidth{1.505625pt}%
\definecolor{currentstroke}{rgb}{0.121569,0.466667,0.705882}%
\pgfsetstrokecolor{currentstroke}%
\pgfsetdash{}{0pt}%
\pgfpathmoveto{\pgfqpoint{0.892288in}{1.739713in}}%
\pgfpathlineto{\pgfqpoint{1.981029in}{0.803145in}}%
\pgfpathlineto{\pgfqpoint{3.069770in}{0.528524in}}%
\pgfpathlineto{\pgfqpoint{4.158511in}{0.479306in}}%
\pgfpathlineto{\pgfqpoint{5.247252in}{0.472892in}}%
\pgfusepath{stroke}%
\end{pgfscope}%
\begin{pgfscope}%
\pgfpathrectangle{\pgfqpoint{0.674540in}{0.399444in}}{\pgfqpoint{4.790460in}{1.605068in}}%
\pgfusepath{clip}%
\pgfsetbuttcap%
\pgfsetroundjoin%
\pgfsetlinewidth{1.505625pt}%
\definecolor{currentstroke}{rgb}{1.000000,0.498039,0.054902}%
\pgfsetstrokecolor{currentstroke}%
\pgfsetdash{{5.550000pt}{2.400000pt}}{0.000000pt}%
\pgfpathmoveto{\pgfqpoint{0.892288in}{1.910951in}}%
\pgfpathlineto{\pgfqpoint{1.001162in}{1.325904in}}%
\pgfpathlineto{\pgfqpoint{1.110036in}{0.911603in}}%
\pgfpathlineto{\pgfqpoint{1.218910in}{0.673075in}}%
\pgfpathlineto{\pgfqpoint{1.327784in}{0.555087in}}%
\pgfpathlineto{\pgfqpoint{1.436658in}{0.503432in}}%
\pgfpathlineto{\pgfqpoint{1.545532in}{0.483037in}}%
\pgfpathlineto{\pgfqpoint{1.654406in}{0.475679in}}%
\pgfpathlineto{\pgfqpoint{1.763281in}{0.473229in}}%
\pgfpathlineto{\pgfqpoint{1.872155in}{0.472472in}}%
\pgfusepath{stroke}%
\end{pgfscope}%
\begin{pgfscope}%
\pgfpathrectangle{\pgfqpoint{0.674540in}{0.399444in}}{\pgfqpoint{4.790460in}{1.605068in}}%
\pgfusepath{clip}%
\pgfsetbuttcap%
\pgfsetroundjoin%
\pgfsetlinewidth{1.505625pt}%
\definecolor{currentstroke}{rgb}{0.172549,0.627451,0.172549}%
\pgfsetstrokecolor{currentstroke}%
\pgfsetdash{{1.500000pt}{2.475000pt}}{0.000000pt}%
\pgfpathmoveto{\pgfqpoint{0.892288in}{1.928692in}}%
\pgfpathlineto{\pgfqpoint{0.914063in}{1.381482in}}%
\pgfpathlineto{\pgfqpoint{0.924950in}{1.155961in}}%
\pgfpathlineto{\pgfqpoint{0.935837in}{0.971467in}}%
\pgfpathlineto{\pgfqpoint{0.946725in}{0.826916in}}%
\pgfpathlineto{\pgfqpoint{0.957612in}{0.717841in}}%
\pgfpathlineto{\pgfqpoint{0.968500in}{0.638245in}}%
\pgfpathlineto{\pgfqpoint{0.979387in}{0.581900in}}%
\pgfpathlineto{\pgfqpoint{0.990274in}{0.543110in}}%
\pgfpathlineto{\pgfqpoint{1.001162in}{0.517090in}}%
\pgfpathlineto{\pgfqpoint{1.012049in}{0.500054in}}%
\pgfpathlineto{\pgfqpoint{1.022937in}{0.489151in}}%
\pgfpathlineto{\pgfqpoint{1.033824in}{0.482323in}}%
\pgfpathlineto{\pgfqpoint{1.044712in}{0.478134in}}%
\pgfpathlineto{\pgfqpoint{1.066486in}{0.474126in}}%
\pgfpathlineto{\pgfqpoint{1.099149in}{0.472497in}}%
\pgfpathlineto{\pgfqpoint{1.099149in}{0.472497in}}%
\pgfusepath{stroke}%
\end{pgfscope}%
\begin{pgfscope}%
\pgfpathrectangle{\pgfqpoint{0.674540in}{0.399444in}}{\pgfqpoint{4.790460in}{1.605068in}}%
\pgfusepath{clip}%
\pgfsetbuttcap%
\pgfsetroundjoin%
\pgfsetlinewidth{1.505625pt}%
\definecolor{currentstroke}{rgb}{0.839216,0.152941,0.156863}%
\pgfsetstrokecolor{currentstroke}%
\pgfsetdash{{9.600000pt}{2.400000pt}{1.500000pt}{2.400000pt}}{0.000000pt}%
\pgfpathmoveto{\pgfqpoint{0.892288in}{1.931555in}}%
\pgfpathlineto{\pgfqpoint{0.899909in}{1.108988in}}%
\pgfpathlineto{\pgfqpoint{0.905353in}{0.762147in}}%
\pgfpathlineto{\pgfqpoint{0.909708in}{0.612235in}}%
\pgfpathlineto{\pgfqpoint{0.914063in}{0.534715in}}%
\pgfpathlineto{\pgfqpoint{0.918418in}{0.498154in}}%
\pgfpathlineto{\pgfqpoint{0.922773in}{0.482264in}}%
\pgfpathlineto{\pgfqpoint{0.927127in}{0.475853in}}%
\pgfpathlineto{\pgfqpoint{0.931482in}{0.473437in}}%
\pgfpathlineto{\pgfqpoint{0.938015in}{0.472402in}}%
\pgfpathlineto{\pgfqpoint{0.938015in}{0.472402in}}%
\pgfusepath{stroke}%
\end{pgfscope}%
\begin{pgfscope}%
\pgfsetrectcap%
\pgfsetmiterjoin%
\pgfsetlinewidth{0.803000pt}%
\definecolor{currentstroke}{rgb}{0.000000,0.000000,0.000000}%
\pgfsetstrokecolor{currentstroke}%
\pgfsetdash{}{0pt}%
\pgfpathmoveto{\pgfqpoint{0.674540in}{0.399444in}}%
\pgfpathlineto{\pgfqpoint{0.674540in}{2.004512in}}%
\pgfusepath{stroke}%
\end{pgfscope}%
\begin{pgfscope}%
\pgfsetrectcap%
\pgfsetmiterjoin%
\pgfsetlinewidth{0.803000pt}%
\definecolor{currentstroke}{rgb}{0.000000,0.000000,0.000000}%
\pgfsetstrokecolor{currentstroke}%
\pgfsetdash{}{0pt}%
\pgfpathmoveto{\pgfqpoint{5.465000in}{0.399444in}}%
\pgfpathlineto{\pgfqpoint{5.465000in}{2.004512in}}%
\pgfusepath{stroke}%
\end{pgfscope}%
\begin{pgfscope}%
\pgfsetrectcap%
\pgfsetmiterjoin%
\pgfsetlinewidth{0.803000pt}%
\definecolor{currentstroke}{rgb}{0.000000,0.000000,0.000000}%
\pgfsetstrokecolor{currentstroke}%
\pgfsetdash{}{0pt}%
\pgfpathmoveto{\pgfqpoint{0.674540in}{0.399444in}}%
\pgfpathlineto{\pgfqpoint{5.465000in}{0.399444in}}%
\pgfusepath{stroke}%
\end{pgfscope}%
\begin{pgfscope}%
\pgfsetrectcap%
\pgfsetmiterjoin%
\pgfsetlinewidth{0.803000pt}%
\definecolor{currentstroke}{rgb}{0.000000,0.000000,0.000000}%
\pgfsetstrokecolor{currentstroke}%
\pgfsetdash{}{0pt}%
\pgfpathmoveto{\pgfqpoint{0.674540in}{2.004512in}}%
\pgfpathlineto{\pgfqpoint{5.465000in}{2.004512in}}%
\pgfusepath{stroke}%
\end{pgfscope}%
\begin{pgfscope}%
\pgfsetbuttcap%
\pgfsetmiterjoin%
\definecolor{currentfill}{rgb}{1.000000,1.000000,1.000000}%
\pgfsetfillcolor{currentfill}%
\pgfsetfillopacity{0.800000}%
\pgfsetlinewidth{1.003750pt}%
\definecolor{currentstroke}{rgb}{0.800000,0.800000,0.800000}%
\pgfsetstrokecolor{currentstroke}%
\pgfsetstrokeopacity{0.800000}%
\pgfsetdash{}{0pt}%
\pgfpathmoveto{\pgfqpoint{4.486799in}{1.118957in}}%
\pgfpathlineto{\pgfqpoint{5.367778in}{1.118957in}}%
\pgfpathquadraticcurveto{\pgfqpoint{5.395556in}{1.118957in}}{\pgfqpoint{5.395556in}{1.146735in}}%
\pgfpathlineto{\pgfqpoint{5.395556in}{1.907290in}}%
\pgfpathquadraticcurveto{\pgfqpoint{5.395556in}{1.935068in}}{\pgfqpoint{5.367778in}{1.935068in}}%
\pgfpathlineto{\pgfqpoint{4.486799in}{1.935068in}}%
\pgfpathquadraticcurveto{\pgfqpoint{4.459021in}{1.935068in}}{\pgfqpoint{4.459021in}{1.907290in}}%
\pgfpathlineto{\pgfqpoint{4.459021in}{1.146735in}}%
\pgfpathquadraticcurveto{\pgfqpoint{4.459021in}{1.118957in}}{\pgfqpoint{4.486799in}{1.118957in}}%
\pgfpathclose%
\pgfusepath{stroke,fill}%
\end{pgfscope}%
\begin{pgfscope}%
\pgfsetrectcap%
\pgfsetroundjoin%
\pgfsetlinewidth{1.505625pt}%
\definecolor{currentstroke}{rgb}{0.121569,0.466667,0.705882}%
\pgfsetstrokecolor{currentstroke}%
\pgfsetdash{}{0pt}%
\pgfpathmoveto{\pgfqpoint{4.514576in}{1.830901in}}%
\pgfpathlineto{\pgfqpoint{4.792354in}{1.830901in}}%
\pgfusepath{stroke}%
\end{pgfscope}%
\begin{pgfscope}%
\definecolor{textcolor}{rgb}{0.000000,0.000000,0.000000}%
\pgfsetstrokecolor{textcolor}%
\pgfsetfillcolor{textcolor}%
\pgftext[x=4.903465in,y=1.782290in,left,base]{\color{textcolor}\rmfamily\fontsize{10.000000}{12.000000}\selectfont \(\displaystyle \tilde{t} = 10^0\)}%
\end{pgfscope}%
\begin{pgfscope}%
\pgfsetbuttcap%
\pgfsetroundjoin%
\pgfsetlinewidth{1.505625pt}%
\definecolor{currentstroke}{rgb}{1.000000,0.498039,0.054902}%
\pgfsetstrokecolor{currentstroke}%
\pgfsetdash{{5.550000pt}{2.400000pt}}{0.000000pt}%
\pgfpathmoveto{\pgfqpoint{4.514576in}{1.637290in}}%
\pgfpathlineto{\pgfqpoint{4.792354in}{1.637290in}}%
\pgfusepath{stroke}%
\end{pgfscope}%
\begin{pgfscope}%
\definecolor{textcolor}{rgb}{0.000000,0.000000,0.000000}%
\pgfsetstrokecolor{textcolor}%
\pgfsetfillcolor{textcolor}%
\pgftext[x=4.903465in,y=1.588679in,left,base]{\color{textcolor}\rmfamily\fontsize{10.000000}{12.000000}\selectfont \(\displaystyle \tilde{t} = 10^1\)}%
\end{pgfscope}%
\begin{pgfscope}%
\pgfsetbuttcap%
\pgfsetroundjoin%
\pgfsetlinewidth{1.505625pt}%
\definecolor{currentstroke}{rgb}{0.172549,0.627451,0.172549}%
\pgfsetstrokecolor{currentstroke}%
\pgfsetdash{{1.500000pt}{2.475000pt}}{0.000000pt}%
\pgfpathmoveto{\pgfqpoint{4.514576in}{1.443679in}}%
\pgfpathlineto{\pgfqpoint{4.792354in}{1.443679in}}%
\pgfusepath{stroke}%
\end{pgfscope}%
\begin{pgfscope}%
\definecolor{textcolor}{rgb}{0.000000,0.000000,0.000000}%
\pgfsetstrokecolor{textcolor}%
\pgfsetfillcolor{textcolor}%
\pgftext[x=4.903465in,y=1.395068in,left,base]{\color{textcolor}\rmfamily\fontsize{10.000000}{12.000000}\selectfont \(\displaystyle \tilde{t} = 10^2\)}%
\end{pgfscope}%
\begin{pgfscope}%
\pgfsetbuttcap%
\pgfsetroundjoin%
\pgfsetlinewidth{1.505625pt}%
\definecolor{currentstroke}{rgb}{0.839216,0.152941,0.156863}%
\pgfsetstrokecolor{currentstroke}%
\pgfsetdash{{9.600000pt}{2.400000pt}{1.500000pt}{2.400000pt}}{0.000000pt}%
\pgfpathmoveto{\pgfqpoint{4.514576in}{1.250068in}}%
\pgfpathlineto{\pgfqpoint{4.792354in}{1.250068in}}%
\pgfusepath{stroke}%
\end{pgfscope}%
\begin{pgfscope}%
\definecolor{textcolor}{rgb}{0.000000,0.000000,0.000000}%
\pgfsetstrokecolor{textcolor}%
\pgfsetfillcolor{textcolor}%
\pgftext[x=4.903465in,y=1.201457in,left,base]{\color{textcolor}\rmfamily\fontsize{10.000000}{12.000000}\selectfont \(\displaystyle \tilde{t} = 10^3\)}%
\end{pgfscope}%
\end{pgfpicture}%
\makeatother%
\endgroup%

	\caption{Gráficas correspondientes al valor absoluto de las funciones de Bessel normalizadas en función de $n$ para valores menores de $1\times10^{-4}$; el eje $x$ se presenta escalado por el factor $\tilde{t}^{-1}$. Se puede ver como a partir de un término crítico $n_c$ las funciones de Bessel decaen pronunciadamente, donde $\frac n{\tilde t} \propto 1$, especialmente para valores altos de $\tilde{t}$.}
	\label{fig:normJ_n(t)}
\end{figure}

Todo esto implica que el argumento de las funciones de Bessel en la aproximación \eqref{eq:iexpChebExpan} está en el 
orden de $10^0$ a $10^3$. Para estos valores se tiene que la función de Bessel decae rápidamente para órdenes $N \propto 
z$ \autocite{Fan2018}, tal como se puede observar en la Fig.\ref{fig:normJ_n(t)}, de esta se puede concluir además que la
aproximación \eqref{eq:iexpChebExpan} se puede llevar a precisión arbitraria dependiente del número de términos a
utilizar.

\begin{figure}[htb]
	\centering
	%% Creator: Matplotlib, PGF backend
%%
%% To include the figure in your LaTeX document, write
%%   \input{<filename>.pgf}
%%
%% Make sure the required packages are loaded in your preamble
%%   \usepackage{pgf}
%%
%% Figures using additional raster images can only be included by \input if
%% they are in the same directory as the main LaTeX file. For loading figures
%% from other directories you can use the `import` package
%%   \usepackage{import}
%% and then include the figures with
%%   \import{<path to file>}{<filename>.pgf}
%%
%% Matplotlib used the following preamble
%%   \usepackage[utf8x]{inputenc}
%%   \usepackage[T1]{fontenc}
%%   \usepackage{physics}
%%   \usepackage[amssymb]{SIunits}
%%   \usepackage{fontspec}
%%
\begingroup%
\makeatletter%
\begin{pgfpicture}%
\pgfpathrectangle{\pgfpointorigin}{\pgfqpoint{5.500000in}{3.399187in}}%
\pgfusepath{use as bounding box, clip}%
\begin{pgfscope}%
\pgfsetbuttcap%
\pgfsetmiterjoin%
\definecolor{currentfill}{rgb}{1.000000,1.000000,1.000000}%
\pgfsetfillcolor{currentfill}%
\pgfsetlinewidth{0.000000pt}%
\definecolor{currentstroke}{rgb}{1.000000,1.000000,1.000000}%
\pgfsetstrokecolor{currentstroke}%
\pgfsetdash{}{0pt}%
\pgfpathmoveto{\pgfqpoint{0.000000in}{0.000000in}}%
\pgfpathlineto{\pgfqpoint{5.500000in}{0.000000in}}%
\pgfpathlineto{\pgfqpoint{5.500000in}{3.399187in}}%
\pgfpathlineto{\pgfqpoint{0.000000in}{3.399187in}}%
\pgfpathclose%
\pgfusepath{fill}%
\end{pgfscope}%
\begin{pgfscope}%
\pgfsetbuttcap%
\pgfsetmiterjoin%
\definecolor{currentfill}{rgb}{1.000000,1.000000,1.000000}%
\pgfsetfillcolor{currentfill}%
\pgfsetlinewidth{0.000000pt}%
\definecolor{currentstroke}{rgb}{0.000000,0.000000,0.000000}%
\pgfsetstrokecolor{currentstroke}%
\pgfsetstrokeopacity{0.000000}%
\pgfsetdash{}{0pt}%
\pgfpathmoveto{\pgfqpoint{0.687500in}{0.373911in}}%
\pgfpathlineto{\pgfqpoint{4.950000in}{0.373911in}}%
\pgfpathlineto{\pgfqpoint{4.950000in}{2.991285in}}%
\pgfpathlineto{\pgfqpoint{0.687500in}{2.991285in}}%
\pgfpathclose%
\pgfusepath{fill}%
\end{pgfscope}%
\begin{pgfscope}%
\pgfsetbuttcap%
\pgfsetroundjoin%
\definecolor{currentfill}{rgb}{0.000000,0.000000,0.000000}%
\pgfsetfillcolor{currentfill}%
\pgfsetlinewidth{0.803000pt}%
\definecolor{currentstroke}{rgb}{0.000000,0.000000,0.000000}%
\pgfsetstrokecolor{currentstroke}%
\pgfsetdash{}{0pt}%
\pgfsys@defobject{currentmarker}{\pgfqpoint{0.000000in}{-0.048611in}}{\pgfqpoint{0.000000in}{0.000000in}}{%
\pgfpathmoveto{\pgfqpoint{0.000000in}{0.000000in}}%
\pgfpathlineto{\pgfqpoint{0.000000in}{-0.048611in}}%
\pgfusepath{stroke,fill}%
}%
\begin{pgfscope}%
\pgfsys@transformshift{0.881250in}{0.373911in}%
\pgfsys@useobject{currentmarker}{}%
\end{pgfscope}%
\end{pgfscope}%
\begin{pgfscope}%
\definecolor{textcolor}{rgb}{0.000000,0.000000,0.000000}%
\pgfsetstrokecolor{textcolor}%
\pgfsetfillcolor{textcolor}%
\pgftext[x=0.881250in,y=0.276688in,,top]{\color{textcolor}\rmfamily\fontsize{10.000000}{12.000000}\selectfont \(\displaystyle 0\)}%
\end{pgfscope}%
\begin{pgfscope}%
\pgfsetbuttcap%
\pgfsetroundjoin%
\definecolor{currentfill}{rgb}{0.000000,0.000000,0.000000}%
\pgfsetfillcolor{currentfill}%
\pgfsetlinewidth{0.803000pt}%
\definecolor{currentstroke}{rgb}{0.000000,0.000000,0.000000}%
\pgfsetstrokecolor{currentstroke}%
\pgfsetdash{}{0pt}%
\pgfsys@defobject{currentmarker}{\pgfqpoint{0.000000in}{-0.048611in}}{\pgfqpoint{0.000000in}{0.000000in}}{%
\pgfpathmoveto{\pgfqpoint{0.000000in}{0.000000in}}%
\pgfpathlineto{\pgfqpoint{0.000000in}{-0.048611in}}%
\pgfusepath{stroke,fill}%
}%
\begin{pgfscope}%
\pgfsys@transformshift{1.657026in}{0.373911in}%
\pgfsys@useobject{currentmarker}{}%
\end{pgfscope}%
\end{pgfscope}%
\begin{pgfscope}%
\definecolor{textcolor}{rgb}{0.000000,0.000000,0.000000}%
\pgfsetstrokecolor{textcolor}%
\pgfsetfillcolor{textcolor}%
\pgftext[x=1.657026in,y=0.276688in,,top]{\color{textcolor}\rmfamily\fontsize{10.000000}{12.000000}\selectfont \(\displaystyle 200\)}%
\end{pgfscope}%
\begin{pgfscope}%
\pgfsetbuttcap%
\pgfsetroundjoin%
\definecolor{currentfill}{rgb}{0.000000,0.000000,0.000000}%
\pgfsetfillcolor{currentfill}%
\pgfsetlinewidth{0.803000pt}%
\definecolor{currentstroke}{rgb}{0.000000,0.000000,0.000000}%
\pgfsetstrokecolor{currentstroke}%
\pgfsetdash{}{0pt}%
\pgfsys@defobject{currentmarker}{\pgfqpoint{0.000000in}{-0.048611in}}{\pgfqpoint{0.000000in}{0.000000in}}{%
\pgfpathmoveto{\pgfqpoint{0.000000in}{0.000000in}}%
\pgfpathlineto{\pgfqpoint{0.000000in}{-0.048611in}}%
\pgfusepath{stroke,fill}%
}%
\begin{pgfscope}%
\pgfsys@transformshift{2.432802in}{0.373911in}%
\pgfsys@useobject{currentmarker}{}%
\end{pgfscope}%
\end{pgfscope}%
\begin{pgfscope}%
\definecolor{textcolor}{rgb}{0.000000,0.000000,0.000000}%
\pgfsetstrokecolor{textcolor}%
\pgfsetfillcolor{textcolor}%
\pgftext[x=2.432802in,y=0.276688in,,top]{\color{textcolor}\rmfamily\fontsize{10.000000}{12.000000}\selectfont \(\displaystyle 400\)}%
\end{pgfscope}%
\begin{pgfscope}%
\pgfsetbuttcap%
\pgfsetroundjoin%
\definecolor{currentfill}{rgb}{0.000000,0.000000,0.000000}%
\pgfsetfillcolor{currentfill}%
\pgfsetlinewidth{0.803000pt}%
\definecolor{currentstroke}{rgb}{0.000000,0.000000,0.000000}%
\pgfsetstrokecolor{currentstroke}%
\pgfsetdash{}{0pt}%
\pgfsys@defobject{currentmarker}{\pgfqpoint{0.000000in}{-0.048611in}}{\pgfqpoint{0.000000in}{0.000000in}}{%
\pgfpathmoveto{\pgfqpoint{0.000000in}{0.000000in}}%
\pgfpathlineto{\pgfqpoint{0.000000in}{-0.048611in}}%
\pgfusepath{stroke,fill}%
}%
\begin{pgfscope}%
\pgfsys@transformshift{3.208577in}{0.373911in}%
\pgfsys@useobject{currentmarker}{}%
\end{pgfscope}%
\end{pgfscope}%
\begin{pgfscope}%
\definecolor{textcolor}{rgb}{0.000000,0.000000,0.000000}%
\pgfsetstrokecolor{textcolor}%
\pgfsetfillcolor{textcolor}%
\pgftext[x=3.208577in,y=0.276688in,,top]{\color{textcolor}\rmfamily\fontsize{10.000000}{12.000000}\selectfont \(\displaystyle 600\)}%
\end{pgfscope}%
\begin{pgfscope}%
\pgfsetbuttcap%
\pgfsetroundjoin%
\definecolor{currentfill}{rgb}{0.000000,0.000000,0.000000}%
\pgfsetfillcolor{currentfill}%
\pgfsetlinewidth{0.803000pt}%
\definecolor{currentstroke}{rgb}{0.000000,0.000000,0.000000}%
\pgfsetstrokecolor{currentstroke}%
\pgfsetdash{}{0pt}%
\pgfsys@defobject{currentmarker}{\pgfqpoint{0.000000in}{-0.048611in}}{\pgfqpoint{0.000000in}{0.000000in}}{%
\pgfpathmoveto{\pgfqpoint{0.000000in}{0.000000in}}%
\pgfpathlineto{\pgfqpoint{0.000000in}{-0.048611in}}%
\pgfusepath{stroke,fill}%
}%
\begin{pgfscope}%
\pgfsys@transformshift{3.984353in}{0.373911in}%
\pgfsys@useobject{currentmarker}{}%
\end{pgfscope}%
\end{pgfscope}%
\begin{pgfscope}%
\definecolor{textcolor}{rgb}{0.000000,0.000000,0.000000}%
\pgfsetstrokecolor{textcolor}%
\pgfsetfillcolor{textcolor}%
\pgftext[x=3.984353in,y=0.276688in,,top]{\color{textcolor}\rmfamily\fontsize{10.000000}{12.000000}\selectfont \(\displaystyle 800\)}%
\end{pgfscope}%
\begin{pgfscope}%
\pgfsetbuttcap%
\pgfsetroundjoin%
\definecolor{currentfill}{rgb}{0.000000,0.000000,0.000000}%
\pgfsetfillcolor{currentfill}%
\pgfsetlinewidth{0.803000pt}%
\definecolor{currentstroke}{rgb}{0.000000,0.000000,0.000000}%
\pgfsetstrokecolor{currentstroke}%
\pgfsetdash{}{0pt}%
\pgfsys@defobject{currentmarker}{\pgfqpoint{0.000000in}{-0.048611in}}{\pgfqpoint{0.000000in}{0.000000in}}{%
\pgfpathmoveto{\pgfqpoint{0.000000in}{0.000000in}}%
\pgfpathlineto{\pgfqpoint{0.000000in}{-0.048611in}}%
\pgfusepath{stroke,fill}%
}%
\begin{pgfscope}%
\pgfsys@transformshift{4.760129in}{0.373911in}%
\pgfsys@useobject{currentmarker}{}%
\end{pgfscope}%
\end{pgfscope}%
\begin{pgfscope}%
\definecolor{textcolor}{rgb}{0.000000,0.000000,0.000000}%
\pgfsetstrokecolor{textcolor}%
\pgfsetfillcolor{textcolor}%
\pgftext[x=4.760129in,y=0.276688in,,top]{\color{textcolor}\rmfamily\fontsize{10.000000}{12.000000}\selectfont \(\displaystyle 1000\)}%
\end{pgfscope}%
\begin{pgfscope}%
\definecolor{textcolor}{rgb}{0.000000,0.000000,0.000000}%
\pgfsetstrokecolor{textcolor}%
\pgfsetfillcolor{textcolor}%
\pgftext[x=2.818750in,y=0.097800in,,top]{\color{textcolor}\rmfamily\fontsize{10.000000}{12.000000}\selectfont \(\displaystyle t\) (\femto\second)}%
\end{pgfscope}%
\begin{pgfscope}%
\pgfsetbuttcap%
\pgfsetroundjoin%
\definecolor{currentfill}{rgb}{0.000000,0.000000,0.000000}%
\pgfsetfillcolor{currentfill}%
\pgfsetlinewidth{0.803000pt}%
\definecolor{currentstroke}{rgb}{0.000000,0.000000,0.000000}%
\pgfsetstrokecolor{currentstroke}%
\pgfsetdash{}{0pt}%
\pgfsys@defobject{currentmarker}{\pgfqpoint{-0.048611in}{0.000000in}}{\pgfqpoint{0.000000in}{0.000000in}}{%
\pgfpathmoveto{\pgfqpoint{0.000000in}{0.000000in}}%
\pgfpathlineto{\pgfqpoint{-0.048611in}{0.000000in}}%
\pgfusepath{stroke,fill}%
}%
\begin{pgfscope}%
\pgfsys@transformshift{0.687500in}{0.680670in}%
\pgfsys@useobject{currentmarker}{}%
\end{pgfscope}%
\end{pgfscope}%
\begin{pgfscope}%
\definecolor{textcolor}{rgb}{0.000000,0.000000,0.000000}%
\pgfsetstrokecolor{textcolor}%
\pgfsetfillcolor{textcolor}%
\pgftext[x=0.304783in,y=0.632476in,left,base]{\color{textcolor}\rmfamily\fontsize{10.000000}{12.000000}\selectfont \(\displaystyle -1.5\)}%
\end{pgfscope}%
\begin{pgfscope}%
\pgfsetbuttcap%
\pgfsetroundjoin%
\definecolor{currentfill}{rgb}{0.000000,0.000000,0.000000}%
\pgfsetfillcolor{currentfill}%
\pgfsetlinewidth{0.803000pt}%
\definecolor{currentstroke}{rgb}{0.000000,0.000000,0.000000}%
\pgfsetstrokecolor{currentstroke}%
\pgfsetdash{}{0pt}%
\pgfsys@defobject{currentmarker}{\pgfqpoint{-0.048611in}{0.000000in}}{\pgfqpoint{0.000000in}{0.000000in}}{%
\pgfpathmoveto{\pgfqpoint{0.000000in}{0.000000in}}%
\pgfpathlineto{\pgfqpoint{-0.048611in}{0.000000in}}%
\pgfusepath{stroke,fill}%
}%
\begin{pgfscope}%
\pgfsys@transformshift{0.687500in}{0.993762in}%
\pgfsys@useobject{currentmarker}{}%
\end{pgfscope}%
\end{pgfscope}%
\begin{pgfscope}%
\definecolor{textcolor}{rgb}{0.000000,0.000000,0.000000}%
\pgfsetstrokecolor{textcolor}%
\pgfsetfillcolor{textcolor}%
\pgftext[x=0.304783in,y=0.945568in,left,base]{\color{textcolor}\rmfamily\fontsize{10.000000}{12.000000}\selectfont \(\displaystyle -1.0\)}%
\end{pgfscope}%
\begin{pgfscope}%
\pgfsetbuttcap%
\pgfsetroundjoin%
\definecolor{currentfill}{rgb}{0.000000,0.000000,0.000000}%
\pgfsetfillcolor{currentfill}%
\pgfsetlinewidth{0.803000pt}%
\definecolor{currentstroke}{rgb}{0.000000,0.000000,0.000000}%
\pgfsetstrokecolor{currentstroke}%
\pgfsetdash{}{0pt}%
\pgfsys@defobject{currentmarker}{\pgfqpoint{-0.048611in}{0.000000in}}{\pgfqpoint{0.000000in}{0.000000in}}{%
\pgfpathmoveto{\pgfqpoint{0.000000in}{0.000000in}}%
\pgfpathlineto{\pgfqpoint{-0.048611in}{0.000000in}}%
\pgfusepath{stroke,fill}%
}%
\begin{pgfscope}%
\pgfsys@transformshift{0.687500in}{1.306854in}%
\pgfsys@useobject{currentmarker}{}%
\end{pgfscope}%
\end{pgfscope}%
\begin{pgfscope}%
\definecolor{textcolor}{rgb}{0.000000,0.000000,0.000000}%
\pgfsetstrokecolor{textcolor}%
\pgfsetfillcolor{textcolor}%
\pgftext[x=0.304783in,y=1.258659in,left,base]{\color{textcolor}\rmfamily\fontsize{10.000000}{12.000000}\selectfont \(\displaystyle -0.5\)}%
\end{pgfscope}%
\begin{pgfscope}%
\pgfsetbuttcap%
\pgfsetroundjoin%
\definecolor{currentfill}{rgb}{0.000000,0.000000,0.000000}%
\pgfsetfillcolor{currentfill}%
\pgfsetlinewidth{0.803000pt}%
\definecolor{currentstroke}{rgb}{0.000000,0.000000,0.000000}%
\pgfsetstrokecolor{currentstroke}%
\pgfsetdash{}{0pt}%
\pgfsys@defobject{currentmarker}{\pgfqpoint{-0.048611in}{0.000000in}}{\pgfqpoint{0.000000in}{0.000000in}}{%
\pgfpathmoveto{\pgfqpoint{0.000000in}{0.000000in}}%
\pgfpathlineto{\pgfqpoint{-0.048611in}{0.000000in}}%
\pgfusepath{stroke,fill}%
}%
\begin{pgfscope}%
\pgfsys@transformshift{0.687500in}{1.619946in}%
\pgfsys@useobject{currentmarker}{}%
\end{pgfscope}%
\end{pgfscope}%
\begin{pgfscope}%
\definecolor{textcolor}{rgb}{0.000000,0.000000,0.000000}%
\pgfsetstrokecolor{textcolor}%
\pgfsetfillcolor{textcolor}%
\pgftext[x=0.412808in,y=1.571751in,left,base]{\color{textcolor}\rmfamily\fontsize{10.000000}{12.000000}\selectfont \(\displaystyle 0.0\)}%
\end{pgfscope}%
\begin{pgfscope}%
\pgfsetbuttcap%
\pgfsetroundjoin%
\definecolor{currentfill}{rgb}{0.000000,0.000000,0.000000}%
\pgfsetfillcolor{currentfill}%
\pgfsetlinewidth{0.803000pt}%
\definecolor{currentstroke}{rgb}{0.000000,0.000000,0.000000}%
\pgfsetstrokecolor{currentstroke}%
\pgfsetdash{}{0pt}%
\pgfsys@defobject{currentmarker}{\pgfqpoint{-0.048611in}{0.000000in}}{\pgfqpoint{0.000000in}{0.000000in}}{%
\pgfpathmoveto{\pgfqpoint{0.000000in}{0.000000in}}%
\pgfpathlineto{\pgfqpoint{-0.048611in}{0.000000in}}%
\pgfusepath{stroke,fill}%
}%
\begin{pgfscope}%
\pgfsys@transformshift{0.687500in}{1.933037in}%
\pgfsys@useobject{currentmarker}{}%
\end{pgfscope}%
\end{pgfscope}%
\begin{pgfscope}%
\definecolor{textcolor}{rgb}{0.000000,0.000000,0.000000}%
\pgfsetstrokecolor{textcolor}%
\pgfsetfillcolor{textcolor}%
\pgftext[x=0.412808in,y=1.884843in,left,base]{\color{textcolor}\rmfamily\fontsize{10.000000}{12.000000}\selectfont \(\displaystyle 0.5\)}%
\end{pgfscope}%
\begin{pgfscope}%
\pgfsetbuttcap%
\pgfsetroundjoin%
\definecolor{currentfill}{rgb}{0.000000,0.000000,0.000000}%
\pgfsetfillcolor{currentfill}%
\pgfsetlinewidth{0.803000pt}%
\definecolor{currentstroke}{rgb}{0.000000,0.000000,0.000000}%
\pgfsetstrokecolor{currentstroke}%
\pgfsetdash{}{0pt}%
\pgfsys@defobject{currentmarker}{\pgfqpoint{-0.048611in}{0.000000in}}{\pgfqpoint{0.000000in}{0.000000in}}{%
\pgfpathmoveto{\pgfqpoint{0.000000in}{0.000000in}}%
\pgfpathlineto{\pgfqpoint{-0.048611in}{0.000000in}}%
\pgfusepath{stroke,fill}%
}%
\begin{pgfscope}%
\pgfsys@transformshift{0.687500in}{2.246129in}%
\pgfsys@useobject{currentmarker}{}%
\end{pgfscope}%
\end{pgfscope}%
\begin{pgfscope}%
\definecolor{textcolor}{rgb}{0.000000,0.000000,0.000000}%
\pgfsetstrokecolor{textcolor}%
\pgfsetfillcolor{textcolor}%
\pgftext[x=0.412808in,y=2.197935in,left,base]{\color{textcolor}\rmfamily\fontsize{10.000000}{12.000000}\selectfont \(\displaystyle 1.0\)}%
\end{pgfscope}%
\begin{pgfscope}%
\pgfsetbuttcap%
\pgfsetroundjoin%
\definecolor{currentfill}{rgb}{0.000000,0.000000,0.000000}%
\pgfsetfillcolor{currentfill}%
\pgfsetlinewidth{0.803000pt}%
\definecolor{currentstroke}{rgb}{0.000000,0.000000,0.000000}%
\pgfsetstrokecolor{currentstroke}%
\pgfsetdash{}{0pt}%
\pgfsys@defobject{currentmarker}{\pgfqpoint{-0.048611in}{0.000000in}}{\pgfqpoint{0.000000in}{0.000000in}}{%
\pgfpathmoveto{\pgfqpoint{0.000000in}{0.000000in}}%
\pgfpathlineto{\pgfqpoint{-0.048611in}{0.000000in}}%
\pgfusepath{stroke,fill}%
}%
\begin{pgfscope}%
\pgfsys@transformshift{0.687500in}{2.559221in}%
\pgfsys@useobject{currentmarker}{}%
\end{pgfscope}%
\end{pgfscope}%
\begin{pgfscope}%
\definecolor{textcolor}{rgb}{0.000000,0.000000,0.000000}%
\pgfsetstrokecolor{textcolor}%
\pgfsetfillcolor{textcolor}%
\pgftext[x=0.412808in,y=2.511027in,left,base]{\color{textcolor}\rmfamily\fontsize{10.000000}{12.000000}\selectfont \(\displaystyle 1.5\)}%
\end{pgfscope}%
\begin{pgfscope}%
\pgfsetbuttcap%
\pgfsetroundjoin%
\definecolor{currentfill}{rgb}{0.000000,0.000000,0.000000}%
\pgfsetfillcolor{currentfill}%
\pgfsetlinewidth{0.803000pt}%
\definecolor{currentstroke}{rgb}{0.000000,0.000000,0.000000}%
\pgfsetstrokecolor{currentstroke}%
\pgfsetdash{}{0pt}%
\pgfsys@defobject{currentmarker}{\pgfqpoint{-0.048611in}{0.000000in}}{\pgfqpoint{0.000000in}{0.000000in}}{%
\pgfpathmoveto{\pgfqpoint{0.000000in}{0.000000in}}%
\pgfpathlineto{\pgfqpoint{-0.048611in}{0.000000in}}%
\pgfusepath{stroke,fill}%
}%
\begin{pgfscope}%
\pgfsys@transformshift{0.687500in}{2.872313in}%
\pgfsys@useobject{currentmarker}{}%
\end{pgfscope}%
\end{pgfscope}%
\begin{pgfscope}%
\definecolor{textcolor}{rgb}{0.000000,0.000000,0.000000}%
\pgfsetstrokecolor{textcolor}%
\pgfsetfillcolor{textcolor}%
\pgftext[x=0.412808in,y=2.824119in,left,base]{\color{textcolor}\rmfamily\fontsize{10.000000}{12.000000}\selectfont \(\displaystyle 2.0\)}%
\end{pgfscope}%
\begin{pgfscope}%
\definecolor{textcolor}{rgb}{0.000000,0.000000,0.000000}%
\pgfsetstrokecolor{textcolor}%
\pgfsetfillcolor{textcolor}%
\pgftext[x=0.249228in,y=1.682598in,,bottom,rotate=90.000000]{\color{textcolor}\rmfamily\fontsize{10.000000}{12.000000}\selectfont \(\displaystyle e^{-t\alpha}\bra{\psi}\hat{U}(t)\ket{\psi}\)}%
\end{pgfscope}%
\begin{pgfscope}%
\pgfpathrectangle{\pgfqpoint{0.687500in}{0.373911in}}{\pgfqpoint{4.262500in}{2.617374in}}%
\pgfusepath{clip}%
\pgfsetrectcap%
\pgfsetroundjoin%
\pgfsetlinewidth{1.505625pt}%
\definecolor{currentstroke}{rgb}{0.121569,0.466667,0.705882}%
\pgfsetstrokecolor{currentstroke}%
\pgfsetdash{}{0pt}%
\pgfpathmoveto{\pgfqpoint{0.881250in}{2.872313in}}%
\pgfpathlineto{\pgfqpoint{0.885129in}{2.851707in}}%
\pgfpathlineto{\pgfqpoint{0.889008in}{2.803041in}}%
\pgfpathlineto{\pgfqpoint{0.892887in}{2.727710in}}%
\pgfpathlineto{\pgfqpoint{0.900644in}{2.505580in}}%
\pgfpathlineto{\pgfqpoint{0.908402in}{2.207432in}}%
\pgfpathlineto{\pgfqpoint{0.939433in}{0.864324in}}%
\pgfpathlineto{\pgfqpoint{0.947191in}{0.644709in}}%
\pgfpathlineto{\pgfqpoint{0.951070in}{0.568920in}}%
\pgfpathlineto{\pgfqpoint{0.954949in}{0.518004in}}%
\pgfpathlineto{\pgfqpoint{0.958828in}{0.492882in}}%
\pgfpathlineto{\pgfqpoint{0.962706in}{0.493873in}}%
\pgfpathlineto{\pgfqpoint{0.966585in}{0.520695in}}%
\pgfpathlineto{\pgfqpoint{0.970464in}{0.572474in}}%
\pgfpathlineto{\pgfqpoint{0.978222in}{0.744633in}}%
\pgfpathlineto{\pgfqpoint{0.985980in}{0.992862in}}%
\pgfpathlineto{\pgfqpoint{0.997616in}{1.454001in}}%
\pgfpathlineto{\pgfqpoint{1.013132in}{2.082408in}}%
\pgfpathlineto{\pgfqpoint{1.020890in}{2.338175in}}%
\pgfpathlineto{\pgfqpoint{1.028647in}{2.523653in}}%
\pgfpathlineto{\pgfqpoint{1.032526in}{2.584900in}}%
\pgfpathlineto{\pgfqpoint{1.036405in}{2.623393in}}%
\pgfpathlineto{\pgfqpoint{1.040284in}{2.638477in}}%
\pgfpathlineto{\pgfqpoint{1.044163in}{2.630038in}}%
\pgfpathlineto{\pgfqpoint{1.048042in}{2.598504in}}%
\pgfpathlineto{\pgfqpoint{1.051921in}{2.544829in}}%
\pgfpathlineto{\pgfqpoint{1.059678in}{2.377324in}}%
\pgfpathlineto{\pgfqpoint{1.067436in}{2.144366in}}%
\pgfpathlineto{\pgfqpoint{1.082952in}{1.575352in}}%
\pgfpathlineto{\pgfqpoint{1.094588in}{1.162352in}}%
\pgfpathlineto{\pgfqpoint{1.102346in}{0.940713in}}%
\pgfpathlineto{\pgfqpoint{1.110104in}{0.785145in}}%
\pgfpathlineto{\pgfqpoint{1.113983in}{0.736425in}}%
\pgfpathlineto{\pgfqpoint{1.117862in}{0.708464in}}%
\pgfpathlineto{\pgfqpoint{1.121740in}{0.701699in}}%
\pgfpathlineto{\pgfqpoint{1.125619in}{0.716072in}}%
\pgfpathlineto{\pgfqpoint{1.129498in}{0.751043in}}%
\pgfpathlineto{\pgfqpoint{1.133377in}{0.805605in}}%
\pgfpathlineto{\pgfqpoint{1.141135in}{0.967310in}}%
\pgfpathlineto{\pgfqpoint{1.148893in}{1.185073in}}%
\pgfpathlineto{\pgfqpoint{1.179924in}{2.170532in}}%
\pgfpathlineto{\pgfqpoint{1.187681in}{2.332619in}}%
\pgfpathlineto{\pgfqpoint{1.191560in}{2.388800in}}%
\pgfpathlineto{\pgfqpoint{1.195439in}{2.426779in}}%
\pgfpathlineto{\pgfqpoint{1.199318in}{2.445864in}}%
\pgfpathlineto{\pgfqpoint{1.203197in}{2.445807in}}%
\pgfpathlineto{\pgfqpoint{1.207076in}{2.426799in}}%
\pgfpathlineto{\pgfqpoint{1.210955in}{2.389466in}}%
\pgfpathlineto{\pgfqpoint{1.214834in}{2.334845in}}%
\pgfpathlineto{\pgfqpoint{1.222591in}{2.179797in}}%
\pgfpathlineto{\pgfqpoint{1.234228in}{1.863609in}}%
\pgfpathlineto{\pgfqpoint{1.257501in}{1.184946in}}%
\pgfpathlineto{\pgfqpoint{1.265259in}{1.020550in}}%
\pgfpathlineto{\pgfqpoint{1.273017in}{0.913841in}}%
\pgfpathlineto{\pgfqpoint{1.276896in}{0.885027in}}%
\pgfpathlineto{\pgfqpoint{1.280775in}{0.873365in}}%
\pgfpathlineto{\pgfqpoint{1.284653in}{0.878951in}}%
\pgfpathlineto{\pgfqpoint{1.288532in}{0.901485in}}%
\pgfpathlineto{\pgfqpoint{1.292411in}{0.940281in}}%
\pgfpathlineto{\pgfqpoint{1.300169in}{1.062113in}}%
\pgfpathlineto{\pgfqpoint{1.307927in}{1.232184in}}%
\pgfpathlineto{\pgfqpoint{1.323442in}{1.648881in}}%
\pgfpathlineto{\pgfqpoint{1.335079in}{1.952206in}}%
\pgfpathlineto{\pgfqpoint{1.342837in}{2.115422in}}%
\pgfpathlineto{\pgfqpoint{1.350594in}{2.230412in}}%
\pgfpathlineto{\pgfqpoint{1.354473in}{2.266650in}}%
\pgfpathlineto{\pgfqpoint{1.358352in}{2.287688in}}%
\pgfpathlineto{\pgfqpoint{1.362231in}{2.293194in}}%
\pgfpathlineto{\pgfqpoint{1.366110in}{2.283195in}}%
\pgfpathlineto{\pgfqpoint{1.369989in}{2.258076in}}%
\pgfpathlineto{\pgfqpoint{1.373868in}{2.218564in}}%
\pgfpathlineto{\pgfqpoint{1.381625in}{2.100856in}}%
\pgfpathlineto{\pgfqpoint{1.389383in}{1.941812in}}%
\pgfpathlineto{\pgfqpoint{1.424293in}{1.153296in}}%
\pgfpathlineto{\pgfqpoint{1.432051in}{1.057526in}}%
\pgfpathlineto{\pgfqpoint{1.435930in}{1.029207in}}%
\pgfpathlineto{\pgfqpoint{1.439809in}{1.014725in}}%
\pgfpathlineto{\pgfqpoint{1.443687in}{1.014276in}}%
\pgfpathlineto{\pgfqpoint{1.447566in}{1.027730in}}%
\pgfpathlineto{\pgfqpoint{1.451445in}{1.054640in}}%
\pgfpathlineto{\pgfqpoint{1.455324in}{1.094254in}}%
\pgfpathlineto{\pgfqpoint{1.463082in}{1.207195in}}%
\pgfpathlineto{\pgfqpoint{1.474718in}{1.438341in}}%
\pgfpathlineto{\pgfqpoint{1.497992in}{1.936409in}}%
\pgfpathlineto{\pgfqpoint{1.505749in}{2.057586in}}%
\pgfpathlineto{\pgfqpoint{1.513507in}{2.136621in}}%
\pgfpathlineto{\pgfqpoint{1.517386in}{2.158179in}}%
\pgfpathlineto{\pgfqpoint{1.521265in}{2.167172in}}%
\pgfpathlineto{\pgfqpoint{1.525144in}{2.163518in}}%
\pgfpathlineto{\pgfqpoint{1.529023in}{2.147426in}}%
\pgfpathlineto{\pgfqpoint{1.532902in}{2.119391in}}%
\pgfpathlineto{\pgfqpoint{1.540659in}{2.030787in}}%
\pgfpathlineto{\pgfqpoint{1.548417in}{1.906632in}}%
\pgfpathlineto{\pgfqpoint{1.563933in}{1.601489in}}%
\pgfpathlineto{\pgfqpoint{1.575569in}{1.378721in}}%
\pgfpathlineto{\pgfqpoint{1.583327in}{1.258533in}}%
\pgfpathlineto{\pgfqpoint{1.591085in}{1.173545in}}%
\pgfpathlineto{\pgfqpoint{1.594964in}{1.146596in}}%
\pgfpathlineto{\pgfqpoint{1.598843in}{1.130778in}}%
\pgfpathlineto{\pgfqpoint{1.602721in}{1.126342in}}%
\pgfpathlineto{\pgfqpoint{1.606600in}{1.133277in}}%
\pgfpathlineto{\pgfqpoint{1.610479in}{1.151311in}}%
\pgfpathlineto{\pgfqpoint{1.614358in}{1.179919in}}%
\pgfpathlineto{\pgfqpoint{1.622116in}{1.265594in}}%
\pgfpathlineto{\pgfqpoint{1.629874in}{1.381747in}}%
\pgfpathlineto{\pgfqpoint{1.664784in}{1.960497in}}%
\pgfpathlineto{\pgfqpoint{1.672541in}{2.031344in}}%
\pgfpathlineto{\pgfqpoint{1.676420in}{2.052454in}}%
\pgfpathlineto{\pgfqpoint{1.680299in}{2.063429in}}%
\pgfpathlineto{\pgfqpoint{1.684178in}{2.064118in}}%
\pgfpathlineto{\pgfqpoint{1.688057in}{2.054608in}}%
\pgfpathlineto{\pgfqpoint{1.691936in}{2.035218in}}%
\pgfpathlineto{\pgfqpoint{1.695815in}{2.006492in}}%
\pgfpathlineto{\pgfqpoint{1.703572in}{1.924229in}}%
\pgfpathlineto{\pgfqpoint{1.715209in}{1.755258in}}%
\pgfpathlineto{\pgfqpoint{1.738482in}{1.389738in}}%
\pgfpathlineto{\pgfqpoint{1.746240in}{1.300423in}}%
\pgfpathlineto{\pgfqpoint{1.753998in}{1.241894in}}%
\pgfpathlineto{\pgfqpoint{1.757877in}{1.225770in}}%
\pgfpathlineto{\pgfqpoint{1.761756in}{1.218853in}}%
\pgfpathlineto{\pgfqpoint{1.765634in}{1.221209in}}%
\pgfpathlineto{\pgfqpoint{1.769513in}{1.232691in}}%
\pgfpathlineto{\pgfqpoint{1.773392in}{1.252945in}}%
\pgfpathlineto{\pgfqpoint{1.781150in}{1.317378in}}%
\pgfpathlineto{\pgfqpoint{1.788908in}{1.408008in}}%
\pgfpathlineto{\pgfqpoint{1.804423in}{1.631455in}}%
\pgfpathlineto{\pgfqpoint{1.816060in}{1.795056in}}%
\pgfpathlineto{\pgfqpoint{1.823818in}{1.883554in}}%
\pgfpathlineto{\pgfqpoint{1.831575in}{1.946362in}}%
\pgfpathlineto{\pgfqpoint{1.835454in}{1.966398in}}%
\pgfpathlineto{\pgfqpoint{1.839333in}{1.978284in}}%
\pgfpathlineto{\pgfqpoint{1.843212in}{1.981829in}}%
\pgfpathlineto{\pgfqpoint{1.847091in}{1.977035in}}%
\pgfpathlineto{\pgfqpoint{1.850970in}{1.964094in}}%
\pgfpathlineto{\pgfqpoint{1.854849in}{1.943385in}}%
\pgfpathlineto{\pgfqpoint{1.862606in}{1.881031in}}%
\pgfpathlineto{\pgfqpoint{1.870364in}{1.796206in}}%
\pgfpathlineto{\pgfqpoint{1.905274in}{1.371432in}}%
\pgfpathlineto{\pgfqpoint{1.913032in}{1.319027in}}%
\pgfpathlineto{\pgfqpoint{1.916911in}{1.303295in}}%
\pgfpathlineto{\pgfqpoint{1.920790in}{1.294986in}}%
\pgfpathlineto{\pgfqpoint{1.924668in}{1.294217in}}%
\pgfpathlineto{\pgfqpoint{1.928547in}{1.300930in}}%
\pgfpathlineto{\pgfqpoint{1.932426in}{1.314897in}}%
\pgfpathlineto{\pgfqpoint{1.936305in}{1.335725in}}%
\pgfpathlineto{\pgfqpoint{1.944063in}{1.395638in}}%
\pgfpathlineto{\pgfqpoint{1.955699in}{1.519154in}}%
\pgfpathlineto{\pgfqpoint{1.978973in}{1.787392in}}%
\pgfpathlineto{\pgfqpoint{1.986730in}{1.853219in}}%
\pgfpathlineto{\pgfqpoint{1.994488in}{1.896558in}}%
\pgfpathlineto{\pgfqpoint{1.998367in}{1.908611in}}%
\pgfpathlineto{\pgfqpoint{2.002246in}{1.913920in}}%
\pgfpathlineto{\pgfqpoint{2.006125in}{1.912431in}}%
\pgfpathlineto{\pgfqpoint{2.010004in}{1.904244in}}%
\pgfpathlineto{\pgfqpoint{2.013883in}{1.889615in}}%
\pgfpathlineto{\pgfqpoint{2.021640in}{1.842764in}}%
\pgfpathlineto{\pgfqpoint{2.029398in}{1.776609in}}%
\pgfpathlineto{\pgfqpoint{2.044914in}{1.612990in}}%
\pgfpathlineto{\pgfqpoint{2.056550in}{1.492846in}}%
\pgfpathlineto{\pgfqpoint{2.064308in}{1.427684in}}%
\pgfpathlineto{\pgfqpoint{2.072066in}{1.381272in}}%
\pgfpathlineto{\pgfqpoint{2.075945in}{1.366379in}}%
\pgfpathlineto{\pgfqpoint{2.079824in}{1.357453in}}%
\pgfpathlineto{\pgfqpoint{2.083702in}{1.354639in}}%
\pgfpathlineto{\pgfqpoint{2.087581in}{1.357942in}}%
\pgfpathlineto{\pgfqpoint{2.091460in}{1.367223in}}%
\pgfpathlineto{\pgfqpoint{2.095339in}{1.382212in}}%
\pgfpathlineto{\pgfqpoint{2.103097in}{1.427589in}}%
\pgfpathlineto{\pgfqpoint{2.110855in}{1.489533in}}%
\pgfpathlineto{\pgfqpoint{2.145765in}{1.801287in}}%
\pgfpathlineto{\pgfqpoint{2.153522in}{1.840046in}}%
\pgfpathlineto{\pgfqpoint{2.157401in}{1.851766in}}%
\pgfpathlineto{\pgfqpoint{2.161280in}{1.858050in}}%
\pgfpathlineto{\pgfqpoint{2.165159in}{1.858808in}}%
\pgfpathlineto{\pgfqpoint{2.169038in}{1.854076in}}%
\pgfpathlineto{\pgfqpoint{2.172917in}{1.844020in}}%
\pgfpathlineto{\pgfqpoint{2.176796in}{1.828921in}}%
\pgfpathlineto{\pgfqpoint{2.184553in}{1.785288in}}%
\pgfpathlineto{\pgfqpoint{2.192311in}{1.727506in}}%
\pgfpathlineto{\pgfqpoint{2.223342in}{1.472078in}}%
\pgfpathlineto{\pgfqpoint{2.231100in}{1.431350in}}%
\pgfpathlineto{\pgfqpoint{2.234979in}{1.417561in}}%
\pgfpathlineto{\pgfqpoint{2.238858in}{1.408554in}}%
\pgfpathlineto{\pgfqpoint{2.242736in}{1.404487in}}%
\pgfpathlineto{\pgfqpoint{2.246615in}{1.405406in}}%
\pgfpathlineto{\pgfqpoint{2.250494in}{1.411238in}}%
\pgfpathlineto{\pgfqpoint{2.254373in}{1.421801in}}%
\pgfpathlineto{\pgfqpoint{2.262131in}{1.455864in}}%
\pgfpathlineto{\pgfqpoint{2.269889in}{1.504152in}}%
\pgfpathlineto{\pgfqpoint{2.285404in}{1.623957in}}%
\pgfpathlineto{\pgfqpoint{2.297041in}{1.712185in}}%
\pgfpathlineto{\pgfqpoint{2.304799in}{1.760162in}}%
\pgfpathlineto{\pgfqpoint{2.312556in}{1.794456in}}%
\pgfpathlineto{\pgfqpoint{2.316435in}{1.805524in}}%
\pgfpathlineto{\pgfqpoint{2.320314in}{1.812223in}}%
\pgfpathlineto{\pgfqpoint{2.324193in}{1.814443in}}%
\pgfpathlineto{\pgfqpoint{2.328072in}{1.812178in}}%
\pgfpathlineto{\pgfqpoint{2.331951in}{1.805525in}}%
\pgfpathlineto{\pgfqpoint{2.335830in}{1.794679in}}%
\pgfpathlineto{\pgfqpoint{2.343587in}{1.761659in}}%
\pgfpathlineto{\pgfqpoint{2.351345in}{1.716426in}}%
\pgfpathlineto{\pgfqpoint{2.370739in}{1.579882in}}%
\pgfpathlineto{\pgfqpoint{2.382376in}{1.506971in}}%
\pgfpathlineto{\pgfqpoint{2.390134in}{1.471512in}}%
\pgfpathlineto{\pgfqpoint{2.394013in}{1.458963in}}%
\pgfpathlineto{\pgfqpoint{2.397892in}{1.450234in}}%
\pgfpathlineto{\pgfqpoint{2.401771in}{1.445486in}}%
\pgfpathlineto{\pgfqpoint{2.405649in}{1.444789in}}%
\pgfpathlineto{\pgfqpoint{2.409528in}{1.448118in}}%
\pgfpathlineto{\pgfqpoint{2.413407in}{1.455357in}}%
\pgfpathlineto{\pgfqpoint{2.417286in}{1.466301in}}%
\pgfpathlineto{\pgfqpoint{2.425044in}{1.498075in}}%
\pgfpathlineto{\pgfqpoint{2.432802in}{1.540286in}}%
\pgfpathlineto{\pgfqpoint{2.463833in}{1.727720in}}%
\pgfpathlineto{\pgfqpoint{2.471590in}{1.757788in}}%
\pgfpathlineto{\pgfqpoint{2.475469in}{1.768016in}}%
\pgfpathlineto{\pgfqpoint{2.479348in}{1.774745in}}%
\pgfpathlineto{\pgfqpoint{2.483227in}{1.777854in}}%
\pgfpathlineto{\pgfqpoint{2.487106in}{1.777308in}}%
\pgfpathlineto{\pgfqpoint{2.490985in}{1.773157in}}%
\pgfpathlineto{\pgfqpoint{2.494864in}{1.765531in}}%
\pgfpathlineto{\pgfqpoint{2.502621in}{1.740768in}}%
\pgfpathlineto{\pgfqpoint{2.510379in}{1.705524in}}%
\pgfpathlineto{\pgfqpoint{2.525895in}{1.617802in}}%
\pgfpathlineto{\pgfqpoint{2.537531in}{1.553014in}}%
\pgfpathlineto{\pgfqpoint{2.545289in}{1.517691in}}%
\pgfpathlineto{\pgfqpoint{2.553047in}{1.492354in}}%
\pgfpathlineto{\pgfqpoint{2.556926in}{1.484131in}}%
\pgfpathlineto{\pgfqpoint{2.560805in}{1.479106in}}%
\pgfpathlineto{\pgfqpoint{2.564683in}{1.477363in}}%
\pgfpathlineto{\pgfqpoint{2.568562in}{1.478909in}}%
\pgfpathlineto{\pgfqpoint{2.572441in}{1.483676in}}%
\pgfpathlineto{\pgfqpoint{2.576320in}{1.491522in}}%
\pgfpathlineto{\pgfqpoint{2.584078in}{1.515548in}}%
\pgfpathlineto{\pgfqpoint{2.591836in}{1.548577in}}%
\pgfpathlineto{\pgfqpoint{2.611230in}{1.648623in}}%
\pgfpathlineto{\pgfqpoint{2.622867in}{1.702231in}}%
\pgfpathlineto{\pgfqpoint{2.630624in}{1.728390in}}%
\pgfpathlineto{\pgfqpoint{2.638382in}{1.744184in}}%
\pgfpathlineto{\pgfqpoint{2.642261in}{1.747768in}}%
\pgfpathlineto{\pgfqpoint{2.646140in}{1.748384in}}%
\pgfpathlineto{\pgfqpoint{2.650019in}{1.746045in}}%
\pgfpathlineto{\pgfqpoint{2.653898in}{1.740837in}}%
\pgfpathlineto{\pgfqpoint{2.661655in}{1.722462in}}%
\pgfpathlineto{\pgfqpoint{2.669413in}{1.695142in}}%
\pgfpathlineto{\pgfqpoint{2.681050in}{1.643356in}}%
\pgfpathlineto{\pgfqpoint{2.700444in}{1.555543in}}%
\pgfpathlineto{\pgfqpoint{2.708202in}{1.529199in}}%
\pgfpathlineto{\pgfqpoint{2.715960in}{1.511616in}}%
\pgfpathlineto{\pgfqpoint{2.719839in}{1.506592in}}%
\pgfpathlineto{\pgfqpoint{2.723717in}{1.504219in}}%
\pgfpathlineto{\pgfqpoint{2.727596in}{1.504526in}}%
\pgfpathlineto{\pgfqpoint{2.731475in}{1.507478in}}%
\pgfpathlineto{\pgfqpoint{2.735354in}{1.512981in}}%
\pgfpathlineto{\pgfqpoint{2.743112in}{1.530982in}}%
\pgfpathlineto{\pgfqpoint{2.750870in}{1.556704in}}%
\pgfpathlineto{\pgfqpoint{2.766385in}{1.620932in}}%
\pgfpathlineto{\pgfqpoint{2.781901in}{1.682293in}}%
\pgfpathlineto{\pgfqpoint{2.789658in}{1.704897in}}%
\pgfpathlineto{\pgfqpoint{2.797416in}{1.719338in}}%
\pgfpathlineto{\pgfqpoint{2.801295in}{1.723106in}}%
\pgfpathlineto{\pgfqpoint{2.805174in}{1.724468in}}%
\pgfpathlineto{\pgfqpoint{2.809053in}{1.723419in}}%
\pgfpathlineto{\pgfqpoint{2.812932in}{1.720005in}}%
\pgfpathlineto{\pgfqpoint{2.816811in}{1.714330in}}%
\pgfpathlineto{\pgfqpoint{2.824568in}{1.696850in}}%
\pgfpathlineto{\pgfqpoint{2.832326in}{1.672734in}}%
\pgfpathlineto{\pgfqpoint{2.851720in}{1.599431in}}%
\pgfpathlineto{\pgfqpoint{2.863357in}{1.560018in}}%
\pgfpathlineto{\pgfqpoint{2.871115in}{1.540720in}}%
\pgfpathlineto{\pgfqpoint{2.878873in}{1.528999in}}%
\pgfpathlineto{\pgfqpoint{2.882752in}{1.526295in}}%
\pgfpathlineto{\pgfqpoint{2.886630in}{1.525768in}}%
\pgfpathlineto{\pgfqpoint{2.890509in}{1.527407in}}%
\pgfpathlineto{\pgfqpoint{2.894388in}{1.531153in}}%
\pgfpathlineto{\pgfqpoint{2.902146in}{1.544495in}}%
\pgfpathlineto{\pgfqpoint{2.909904in}{1.564423in}}%
\pgfpathlineto{\pgfqpoint{2.921540in}{1.602311in}}%
\pgfpathlineto{\pgfqpoint{2.940935in}{1.666772in}}%
\pgfpathlineto{\pgfqpoint{2.948692in}{1.686185in}}%
\pgfpathlineto{\pgfqpoint{2.956450in}{1.699197in}}%
\pgfpathlineto{\pgfqpoint{2.960329in}{1.702948in}}%
\pgfpathlineto{\pgfqpoint{2.964208in}{1.704756in}}%
\pgfpathlineto{\pgfqpoint{2.968087in}{1.704600in}}%
\pgfpathlineto{\pgfqpoint{2.971966in}{1.702503in}}%
\pgfpathlineto{\pgfqpoint{2.975845in}{1.698532in}}%
\pgfpathlineto{\pgfqpoint{2.983602in}{1.685449in}}%
\pgfpathlineto{\pgfqpoint{2.991360in}{1.666677in}}%
\pgfpathlineto{\pgfqpoint{3.006876in}{1.619651in}}%
\pgfpathlineto{\pgfqpoint{3.022391in}{1.574575in}}%
\pgfpathlineto{\pgfqpoint{3.030149in}{1.557911in}}%
\pgfpathlineto{\pgfqpoint{3.037907in}{1.547210in}}%
\pgfpathlineto{\pgfqpoint{3.041786in}{1.544387in}}%
\pgfpathlineto{\pgfqpoint{3.045664in}{1.543326in}}%
\pgfpathlineto{\pgfqpoint{3.049543in}{1.544034in}}%
\pgfpathlineto{\pgfqpoint{3.053422in}{1.546477in}}%
\pgfpathlineto{\pgfqpoint{3.061180in}{1.556236in}}%
\pgfpathlineto{\pgfqpoint{3.068938in}{1.571588in}}%
\pgfpathlineto{\pgfqpoint{3.080574in}{1.601709in}}%
\pgfpathlineto{\pgfqpoint{3.099969in}{1.654752in}}%
\pgfpathlineto{\pgfqpoint{3.107726in}{1.671333in}}%
\pgfpathlineto{\pgfqpoint{3.115484in}{1.682919in}}%
\pgfpathlineto{\pgfqpoint{3.119363in}{1.686520in}}%
\pgfpathlineto{\pgfqpoint{3.123242in}{1.688557in}}%
\pgfpathlineto{\pgfqpoint{3.127121in}{1.689000in}}%
\pgfpathlineto{\pgfqpoint{3.131000in}{1.687853in}}%
\pgfpathlineto{\pgfqpoint{3.134879in}{1.685160in}}%
\pgfpathlineto{\pgfqpoint{3.142636in}{1.675474in}}%
\pgfpathlineto{\pgfqpoint{3.150394in}{1.660939in}}%
\pgfpathlineto{\pgfqpoint{3.162031in}{1.633220in}}%
\pgfpathlineto{\pgfqpoint{3.181425in}{1.585902in}}%
\pgfpathlineto{\pgfqpoint{3.189183in}{1.571598in}}%
\pgfpathlineto{\pgfqpoint{3.196941in}{1.561968in}}%
\pgfpathlineto{\pgfqpoint{3.200820in}{1.559170in}}%
\pgfpathlineto{\pgfqpoint{3.204698in}{1.557794in}}%
\pgfpathlineto{\pgfqpoint{3.208577in}{1.557858in}}%
\pgfpathlineto{\pgfqpoint{3.212456in}{1.559346in}}%
\pgfpathlineto{\pgfqpoint{3.220214in}{1.566370in}}%
\pgfpathlineto{\pgfqpoint{3.227972in}{1.578120in}}%
\pgfpathlineto{\pgfqpoint{3.239608in}{1.601981in}}%
\pgfpathlineto{\pgfqpoint{3.262882in}{1.652959in}}%
\pgfpathlineto{\pgfqpoint{3.270639in}{1.665244in}}%
\pgfpathlineto{\pgfqpoint{3.278397in}{1.673172in}}%
\pgfpathlineto{\pgfqpoint{3.282276in}{1.675286in}}%
\pgfpathlineto{\pgfqpoint{3.286155in}{1.676109in}}%
\pgfpathlineto{\pgfqpoint{3.290034in}{1.675636in}}%
\pgfpathlineto{\pgfqpoint{3.293913in}{1.673889in}}%
\pgfpathlineto{\pgfqpoint{3.301670in}{1.666814in}}%
\pgfpathlineto{\pgfqpoint{3.309428in}{1.655623in}}%
\pgfpathlineto{\pgfqpoint{3.321065in}{1.633594in}}%
\pgfpathlineto{\pgfqpoint{3.340459in}{1.594669in}}%
\pgfpathlineto{\pgfqpoint{3.348217in}{1.582458in}}%
\pgfpathlineto{\pgfqpoint{3.355975in}{1.573893in}}%
\pgfpathlineto{\pgfqpoint{3.363732in}{1.569680in}}%
\pgfpathlineto{\pgfqpoint{3.367611in}{1.569314in}}%
\pgfpathlineto{\pgfqpoint{3.371490in}{1.570114in}}%
\pgfpathlineto{\pgfqpoint{3.379248in}{1.575065in}}%
\pgfpathlineto{\pgfqpoint{3.387006in}{1.583993in}}%
\pgfpathlineto{\pgfqpoint{3.398642in}{1.602823in}}%
\pgfpathlineto{\pgfqpoint{3.421916in}{1.644694in}}%
\pgfpathlineto{\pgfqpoint{3.429673in}{1.655232in}}%
\pgfpathlineto{\pgfqpoint{3.437431in}{1.662358in}}%
\pgfpathlineto{\pgfqpoint{3.445189in}{1.665491in}}%
\pgfpathlineto{\pgfqpoint{3.452947in}{1.664426in}}%
\pgfpathlineto{\pgfqpoint{3.460704in}{1.659343in}}%
\pgfpathlineto{\pgfqpoint{3.468462in}{1.650784in}}%
\pgfpathlineto{\pgfqpoint{3.480099in}{1.633340in}}%
\pgfpathlineto{\pgfqpoint{3.503372in}{1.595925in}}%
\pgfpathlineto{\pgfqpoint{3.511130in}{1.586870in}}%
\pgfpathlineto{\pgfqpoint{3.518888in}{1.580997in}}%
\pgfpathlineto{\pgfqpoint{3.526645in}{1.578778in}}%
\pgfpathlineto{\pgfqpoint{3.534403in}{1.580340in}}%
\pgfpathlineto{\pgfqpoint{3.542161in}{1.585468in}}%
\pgfpathlineto{\pgfqpoint{3.549919in}{1.593626in}}%
\pgfpathlineto{\pgfqpoint{3.561555in}{1.609736in}}%
\pgfpathlineto{\pgfqpoint{3.580950in}{1.638300in}}%
\pgfpathlineto{\pgfqpoint{3.588707in}{1.647292in}}%
\pgfpathlineto{\pgfqpoint{3.596465in}{1.653623in}}%
\pgfpathlineto{\pgfqpoint{3.604223in}{1.656770in}}%
\pgfpathlineto{\pgfqpoint{3.611981in}{1.656511in}}%
\pgfpathlineto{\pgfqpoint{3.619738in}{1.652937in}}%
\pgfpathlineto{\pgfqpoint{3.627496in}{1.646438in}}%
\pgfpathlineto{\pgfqpoint{3.639133in}{1.632678in}}%
\pgfpathlineto{\pgfqpoint{3.666285in}{1.597800in}}%
\pgfpathlineto{\pgfqpoint{3.674043in}{1.591212in}}%
\pgfpathlineto{\pgfqpoint{3.681801in}{1.587364in}}%
\pgfpathlineto{\pgfqpoint{3.689558in}{1.586551in}}%
\pgfpathlineto{\pgfqpoint{3.697316in}{1.588786in}}%
\pgfpathlineto{\pgfqpoint{3.705074in}{1.593807in}}%
\pgfpathlineto{\pgfqpoint{3.716710in}{1.605386in}}%
\pgfpathlineto{\pgfqpoint{3.747741in}{1.641018in}}%
\pgfpathlineto{\pgfqpoint{3.755499in}{1.646589in}}%
\pgfpathlineto{\pgfqpoint{3.763257in}{1.649630in}}%
\pgfpathlineto{\pgfqpoint{3.771015in}{1.649915in}}%
\pgfpathlineto{\pgfqpoint{3.778773in}{1.647473in}}%
\pgfpathlineto{\pgfqpoint{3.786530in}{1.642580in}}%
\pgfpathlineto{\pgfqpoint{3.798167in}{1.631771in}}%
\pgfpathlineto{\pgfqpoint{3.825319in}{1.603098in}}%
\pgfpathlineto{\pgfqpoint{3.833077in}{1.597387in}}%
\pgfpathlineto{\pgfqpoint{3.840835in}{1.593837in}}%
\pgfpathlineto{\pgfqpoint{3.848592in}{1.592728in}}%
\pgfpathlineto{\pgfqpoint{3.856350in}{1.594113in}}%
\pgfpathlineto{\pgfqpoint{3.864108in}{1.597816in}}%
\pgfpathlineto{\pgfqpoint{3.875744in}{1.606834in}}%
\pgfpathlineto{\pgfqpoint{3.910654in}{1.638740in}}%
\pgfpathlineto{\pgfqpoint{3.918412in}{1.642640in}}%
\pgfpathlineto{\pgfqpoint{3.926170in}{1.644402in}}%
\pgfpathlineto{\pgfqpoint{3.933928in}{1.643908in}}%
\pgfpathlineto{\pgfqpoint{3.941685in}{1.641248in}}%
\pgfpathlineto{\pgfqpoint{3.953322in}{1.633865in}}%
\pgfpathlineto{\pgfqpoint{3.972716in}{1.616840in}}%
\pgfpathlineto{\pgfqpoint{3.988232in}{1.604584in}}%
\pgfpathlineto{\pgfqpoint{3.995990in}{1.600470in}}%
\pgfpathlineto{\pgfqpoint{4.003747in}{1.598206in}}%
\pgfpathlineto{\pgfqpoint{4.011505in}{1.597961in}}%
\pgfpathlineto{\pgfqpoint{4.019263in}{1.599718in}}%
\pgfpathlineto{\pgfqpoint{4.030900in}{1.605624in}}%
\pgfpathlineto{\pgfqpoint{4.046415in}{1.617407in}}%
\pgfpathlineto{\pgfqpoint{4.065810in}{1.632211in}}%
\pgfpathlineto{\pgfqpoint{4.077446in}{1.637953in}}%
\pgfpathlineto{\pgfqpoint{4.085204in}{1.639708in}}%
\pgfpathlineto{\pgfqpoint{4.092962in}{1.639633in}}%
\pgfpathlineto{\pgfqpoint{4.100719in}{1.637772in}}%
\pgfpathlineto{\pgfqpoint{4.112356in}{1.632117in}}%
\pgfpathlineto{\pgfqpoint{4.127872in}{1.621308in}}%
\pgfpathlineto{\pgfqpoint{4.147266in}{1.608191in}}%
\pgfpathlineto{\pgfqpoint{4.158903in}{1.603346in}}%
\pgfpathlineto{\pgfqpoint{4.166660in}{1.602026in}}%
\pgfpathlineto{\pgfqpoint{4.174418in}{1.602359in}}%
\pgfpathlineto{\pgfqpoint{4.186055in}{1.605781in}}%
\pgfpathlineto{\pgfqpoint{4.197691in}{1.611954in}}%
\pgfpathlineto{\pgfqpoint{4.232601in}{1.632818in}}%
\pgfpathlineto{\pgfqpoint{4.244238in}{1.635866in}}%
\pgfpathlineto{\pgfqpoint{4.251996in}{1.636072in}}%
\pgfpathlineto{\pgfqpoint{4.263632in}{1.633667in}}%
\pgfpathlineto{\pgfqpoint{4.275269in}{1.628571in}}%
\pgfpathlineto{\pgfqpoint{4.314058in}{1.607911in}}%
\pgfpathlineto{\pgfqpoint{4.325694in}{1.605469in}}%
\pgfpathlineto{\pgfqpoint{4.337331in}{1.606015in}}%
\pgfpathlineto{\pgfqpoint{4.348968in}{1.609349in}}%
\pgfpathlineto{\pgfqpoint{4.364483in}{1.616769in}}%
\pgfpathlineto{\pgfqpoint{4.387757in}{1.628508in}}%
\pgfpathlineto{\pgfqpoint{4.399393in}{1.632087in}}%
\pgfpathlineto{\pgfqpoint{4.411030in}{1.633116in}}%
\pgfpathlineto{\pgfqpoint{4.422666in}{1.631462in}}%
\pgfpathlineto{\pgfqpoint{4.434303in}{1.627539in}}%
\pgfpathlineto{\pgfqpoint{4.476971in}{1.609541in}}%
\pgfpathlineto{\pgfqpoint{4.488607in}{1.608065in}}%
\pgfpathlineto{\pgfqpoint{4.500244in}{1.609024in}}%
\pgfpathlineto{\pgfqpoint{4.511881in}{1.612155in}}%
\pgfpathlineto{\pgfqpoint{4.531275in}{1.620178in}}%
\pgfpathlineto{\pgfqpoint{4.550669in}{1.627685in}}%
\pgfpathlineto{\pgfqpoint{4.562306in}{1.630182in}}%
\pgfpathlineto{\pgfqpoint{4.573943in}{1.630544in}}%
\pgfpathlineto{\pgfqpoint{4.585579in}{1.628761in}}%
\pgfpathlineto{\pgfqpoint{4.601095in}{1.623830in}}%
\pgfpathlineto{\pgfqpoint{4.632126in}{1.612643in}}%
\pgfpathlineto{\pgfqpoint{4.643763in}{1.610585in}}%
\pgfpathlineto{\pgfqpoint{4.655399in}{1.610473in}}%
\pgfpathlineto{\pgfqpoint{4.667036in}{1.612271in}}%
\pgfpathlineto{\pgfqpoint{4.682551in}{1.616873in}}%
\pgfpathlineto{\pgfqpoint{4.713582in}{1.626806in}}%
\pgfpathlineto{\pgfqpoint{4.725219in}{1.628483in}}%
\pgfpathlineto{\pgfqpoint{4.736856in}{1.628392in}}%
\pgfpathlineto{\pgfqpoint{4.752371in}{1.625691in}}%
\pgfpathlineto{\pgfqpoint{4.756250in}{1.624657in}}%
\pgfpathlineto{\pgfqpoint{4.756250in}{1.624657in}}%
\pgfusepath{stroke}%
\end{pgfscope}%
\begin{pgfscope}%
\pgfpathrectangle{\pgfqpoint{0.687500in}{0.373911in}}{\pgfqpoint{4.262500in}{2.617374in}}%
\pgfusepath{clip}%
\pgfsetbuttcap%
\pgfsetroundjoin%
\pgfsetlinewidth{1.505625pt}%
\definecolor{currentstroke}{rgb}{1.000000,0.498039,0.054902}%
\pgfsetstrokecolor{currentstroke}%
\pgfsetdash{{5.550000pt}{2.400000pt}}{0.000000pt}%
\pgfpathmoveto{\pgfqpoint{0.881250in}{2.872313in}}%
\pgfpathlineto{\pgfqpoint{0.885129in}{2.851707in}}%
\pgfpathlineto{\pgfqpoint{0.889008in}{2.803041in}}%
\pgfpathlineto{\pgfqpoint{0.892887in}{2.727710in}}%
\pgfpathlineto{\pgfqpoint{0.900644in}{2.505580in}}%
\pgfpathlineto{\pgfqpoint{0.908402in}{2.207432in}}%
\pgfpathlineto{\pgfqpoint{0.939433in}{0.864324in}}%
\pgfpathlineto{\pgfqpoint{0.947191in}{0.644709in}}%
\pgfpathlineto{\pgfqpoint{0.951070in}{0.568920in}}%
\pgfpathlineto{\pgfqpoint{0.954949in}{0.518004in}}%
\pgfpathlineto{\pgfqpoint{0.958828in}{0.492882in}}%
\pgfpathlineto{\pgfqpoint{0.962706in}{0.493873in}}%
\pgfpathlineto{\pgfqpoint{0.966585in}{0.520695in}}%
\pgfpathlineto{\pgfqpoint{0.970464in}{0.572474in}}%
\pgfpathlineto{\pgfqpoint{0.978222in}{0.744633in}}%
\pgfpathlineto{\pgfqpoint{0.985980in}{0.992862in}}%
\pgfpathlineto{\pgfqpoint{0.997616in}{1.454001in}}%
\pgfpathlineto{\pgfqpoint{1.013132in}{2.082408in}}%
\pgfpathlineto{\pgfqpoint{1.020890in}{2.338175in}}%
\pgfpathlineto{\pgfqpoint{1.028647in}{2.523653in}}%
\pgfpathlineto{\pgfqpoint{1.032526in}{2.584900in}}%
\pgfpathlineto{\pgfqpoint{1.036405in}{2.623393in}}%
\pgfpathlineto{\pgfqpoint{1.040284in}{2.638477in}}%
\pgfpathlineto{\pgfqpoint{1.044163in}{2.630038in}}%
\pgfpathlineto{\pgfqpoint{1.048042in}{2.598504in}}%
\pgfpathlineto{\pgfqpoint{1.051921in}{2.544829in}}%
\pgfpathlineto{\pgfqpoint{1.059678in}{2.377324in}}%
\pgfpathlineto{\pgfqpoint{1.067436in}{2.144366in}}%
\pgfpathlineto{\pgfqpoint{1.082952in}{1.575352in}}%
\pgfpathlineto{\pgfqpoint{1.094588in}{1.162352in}}%
\pgfpathlineto{\pgfqpoint{1.102346in}{0.940713in}}%
\pgfpathlineto{\pgfqpoint{1.110104in}{0.785145in}}%
\pgfpathlineto{\pgfqpoint{1.113983in}{0.736425in}}%
\pgfpathlineto{\pgfqpoint{1.117862in}{0.708464in}}%
\pgfpathlineto{\pgfqpoint{1.121740in}{0.701699in}}%
\pgfpathlineto{\pgfqpoint{1.125619in}{0.716072in}}%
\pgfpathlineto{\pgfqpoint{1.129498in}{0.751043in}}%
\pgfpathlineto{\pgfqpoint{1.133377in}{0.805605in}}%
\pgfpathlineto{\pgfqpoint{1.141135in}{0.967310in}}%
\pgfpathlineto{\pgfqpoint{1.148893in}{1.185073in}}%
\pgfpathlineto{\pgfqpoint{1.179924in}{2.170532in}}%
\pgfpathlineto{\pgfqpoint{1.187681in}{2.332619in}}%
\pgfpathlineto{\pgfqpoint{1.191560in}{2.388800in}}%
\pgfpathlineto{\pgfqpoint{1.195439in}{2.426779in}}%
\pgfpathlineto{\pgfqpoint{1.199318in}{2.445864in}}%
\pgfpathlineto{\pgfqpoint{1.203197in}{2.445807in}}%
\pgfpathlineto{\pgfqpoint{1.207076in}{2.426799in}}%
\pgfpathlineto{\pgfqpoint{1.210955in}{2.389466in}}%
\pgfpathlineto{\pgfqpoint{1.214834in}{2.334845in}}%
\pgfpathlineto{\pgfqpoint{1.222591in}{2.179797in}}%
\pgfpathlineto{\pgfqpoint{1.234228in}{1.863609in}}%
\pgfpathlineto{\pgfqpoint{1.257501in}{1.184946in}}%
\pgfpathlineto{\pgfqpoint{1.265259in}{1.020550in}}%
\pgfpathlineto{\pgfqpoint{1.273017in}{0.913841in}}%
\pgfpathlineto{\pgfqpoint{1.276896in}{0.885027in}}%
\pgfpathlineto{\pgfqpoint{1.280775in}{0.873365in}}%
\pgfpathlineto{\pgfqpoint{1.284653in}{0.878951in}}%
\pgfpathlineto{\pgfqpoint{1.288532in}{0.901485in}}%
\pgfpathlineto{\pgfqpoint{1.292411in}{0.940281in}}%
\pgfpathlineto{\pgfqpoint{1.300169in}{1.062113in}}%
\pgfpathlineto{\pgfqpoint{1.307927in}{1.232184in}}%
\pgfpathlineto{\pgfqpoint{1.323442in}{1.648881in}}%
\pgfpathlineto{\pgfqpoint{1.335079in}{1.952206in}}%
\pgfpathlineto{\pgfqpoint{1.342837in}{2.115422in}}%
\pgfpathlineto{\pgfqpoint{1.350594in}{2.230412in}}%
\pgfpathlineto{\pgfqpoint{1.354473in}{2.266650in}}%
\pgfpathlineto{\pgfqpoint{1.358352in}{2.287688in}}%
\pgfpathlineto{\pgfqpoint{1.362231in}{2.293194in}}%
\pgfpathlineto{\pgfqpoint{1.366110in}{2.283195in}}%
\pgfpathlineto{\pgfqpoint{1.369989in}{2.258076in}}%
\pgfpathlineto{\pgfqpoint{1.373868in}{2.218564in}}%
\pgfpathlineto{\pgfqpoint{1.381625in}{2.100856in}}%
\pgfpathlineto{\pgfqpoint{1.389383in}{1.941812in}}%
\pgfpathlineto{\pgfqpoint{1.424293in}{1.153296in}}%
\pgfpathlineto{\pgfqpoint{1.432051in}{1.057526in}}%
\pgfpathlineto{\pgfqpoint{1.435930in}{1.029207in}}%
\pgfpathlineto{\pgfqpoint{1.439809in}{1.014725in}}%
\pgfpathlineto{\pgfqpoint{1.443687in}{1.014276in}}%
\pgfpathlineto{\pgfqpoint{1.447566in}{1.027730in}}%
\pgfpathlineto{\pgfqpoint{1.451445in}{1.054640in}}%
\pgfpathlineto{\pgfqpoint{1.455324in}{1.094254in}}%
\pgfpathlineto{\pgfqpoint{1.463082in}{1.207195in}}%
\pgfpathlineto{\pgfqpoint{1.474718in}{1.438341in}}%
\pgfpathlineto{\pgfqpoint{1.497992in}{1.936409in}}%
\pgfpathlineto{\pgfqpoint{1.505749in}{2.057586in}}%
\pgfpathlineto{\pgfqpoint{1.513507in}{2.136621in}}%
\pgfpathlineto{\pgfqpoint{1.517386in}{2.158179in}}%
\pgfpathlineto{\pgfqpoint{1.521265in}{2.167172in}}%
\pgfpathlineto{\pgfqpoint{1.525144in}{2.163518in}}%
\pgfpathlineto{\pgfqpoint{1.529023in}{2.147426in}}%
\pgfpathlineto{\pgfqpoint{1.532902in}{2.119391in}}%
\pgfpathlineto{\pgfqpoint{1.540659in}{2.030787in}}%
\pgfpathlineto{\pgfqpoint{1.548417in}{1.906632in}}%
\pgfpathlineto{\pgfqpoint{1.563933in}{1.601489in}}%
\pgfpathlineto{\pgfqpoint{1.575569in}{1.378721in}}%
\pgfpathlineto{\pgfqpoint{1.583327in}{1.258533in}}%
\pgfpathlineto{\pgfqpoint{1.591085in}{1.173545in}}%
\pgfpathlineto{\pgfqpoint{1.594964in}{1.146596in}}%
\pgfpathlineto{\pgfqpoint{1.598843in}{1.130778in}}%
\pgfpathlineto{\pgfqpoint{1.602721in}{1.126342in}}%
\pgfpathlineto{\pgfqpoint{1.606600in}{1.133277in}}%
\pgfpathlineto{\pgfqpoint{1.610479in}{1.151311in}}%
\pgfpathlineto{\pgfqpoint{1.614358in}{1.179919in}}%
\pgfpathlineto{\pgfqpoint{1.622116in}{1.265594in}}%
\pgfpathlineto{\pgfqpoint{1.629874in}{1.381747in}}%
\pgfpathlineto{\pgfqpoint{1.664784in}{1.960497in}}%
\pgfpathlineto{\pgfqpoint{1.672541in}{2.031344in}}%
\pgfpathlineto{\pgfqpoint{1.676420in}{2.052454in}}%
\pgfpathlineto{\pgfqpoint{1.680299in}{2.063429in}}%
\pgfpathlineto{\pgfqpoint{1.684178in}{2.064118in}}%
\pgfpathlineto{\pgfqpoint{1.688057in}{2.054608in}}%
\pgfpathlineto{\pgfqpoint{1.691936in}{2.035218in}}%
\pgfpathlineto{\pgfqpoint{1.695815in}{2.006492in}}%
\pgfpathlineto{\pgfqpoint{1.703572in}{1.924229in}}%
\pgfpathlineto{\pgfqpoint{1.715209in}{1.755258in}}%
\pgfpathlineto{\pgfqpoint{1.738482in}{1.389738in}}%
\pgfpathlineto{\pgfqpoint{1.746240in}{1.300423in}}%
\pgfpathlineto{\pgfqpoint{1.753998in}{1.241894in}}%
\pgfpathlineto{\pgfqpoint{1.757877in}{1.225770in}}%
\pgfpathlineto{\pgfqpoint{1.761756in}{1.218853in}}%
\pgfpathlineto{\pgfqpoint{1.765634in}{1.221209in}}%
\pgfpathlineto{\pgfqpoint{1.769513in}{1.232691in}}%
\pgfpathlineto{\pgfqpoint{1.773392in}{1.252945in}}%
\pgfpathlineto{\pgfqpoint{1.781150in}{1.317378in}}%
\pgfpathlineto{\pgfqpoint{1.788908in}{1.408008in}}%
\pgfpathlineto{\pgfqpoint{1.804423in}{1.631455in}}%
\pgfpathlineto{\pgfqpoint{1.816060in}{1.795056in}}%
\pgfpathlineto{\pgfqpoint{1.823818in}{1.883554in}}%
\pgfpathlineto{\pgfqpoint{1.831575in}{1.946362in}}%
\pgfpathlineto{\pgfqpoint{1.835454in}{1.966398in}}%
\pgfpathlineto{\pgfqpoint{1.839333in}{1.978284in}}%
\pgfpathlineto{\pgfqpoint{1.843212in}{1.981829in}}%
\pgfpathlineto{\pgfqpoint{1.847091in}{1.977035in}}%
\pgfpathlineto{\pgfqpoint{1.850970in}{1.964094in}}%
\pgfpathlineto{\pgfqpoint{1.854849in}{1.943385in}}%
\pgfpathlineto{\pgfqpoint{1.862606in}{1.881031in}}%
\pgfpathlineto{\pgfqpoint{1.870364in}{1.796206in}}%
\pgfpathlineto{\pgfqpoint{1.905274in}{1.371432in}}%
\pgfpathlineto{\pgfqpoint{1.913032in}{1.319027in}}%
\pgfpathlineto{\pgfqpoint{1.916911in}{1.303295in}}%
\pgfpathlineto{\pgfqpoint{1.920790in}{1.294986in}}%
\pgfpathlineto{\pgfqpoint{1.924668in}{1.294217in}}%
\pgfpathlineto{\pgfqpoint{1.928547in}{1.300930in}}%
\pgfpathlineto{\pgfqpoint{1.932426in}{1.314897in}}%
\pgfpathlineto{\pgfqpoint{1.936305in}{1.335725in}}%
\pgfpathlineto{\pgfqpoint{1.944063in}{1.395638in}}%
\pgfpathlineto{\pgfqpoint{1.955699in}{1.519154in}}%
\pgfpathlineto{\pgfqpoint{1.978973in}{1.787392in}}%
\pgfpathlineto{\pgfqpoint{1.986730in}{1.853219in}}%
\pgfpathlineto{\pgfqpoint{1.994488in}{1.896558in}}%
\pgfpathlineto{\pgfqpoint{1.998367in}{1.908611in}}%
\pgfpathlineto{\pgfqpoint{2.002246in}{1.913920in}}%
\pgfpathlineto{\pgfqpoint{2.006125in}{1.912431in}}%
\pgfpathlineto{\pgfqpoint{2.010004in}{1.904244in}}%
\pgfpathlineto{\pgfqpoint{2.013883in}{1.889615in}}%
\pgfpathlineto{\pgfqpoint{2.021640in}{1.842764in}}%
\pgfpathlineto{\pgfqpoint{2.029398in}{1.776609in}}%
\pgfpathlineto{\pgfqpoint{2.044914in}{1.612990in}}%
\pgfpathlineto{\pgfqpoint{2.056550in}{1.492846in}}%
\pgfpathlineto{\pgfqpoint{2.064308in}{1.427684in}}%
\pgfpathlineto{\pgfqpoint{2.072066in}{1.381272in}}%
\pgfpathlineto{\pgfqpoint{2.075945in}{1.366379in}}%
\pgfpathlineto{\pgfqpoint{2.079824in}{1.357453in}}%
\pgfpathlineto{\pgfqpoint{2.083702in}{1.354639in}}%
\pgfpathlineto{\pgfqpoint{2.087581in}{1.357942in}}%
\pgfpathlineto{\pgfqpoint{2.091460in}{1.367223in}}%
\pgfpathlineto{\pgfqpoint{2.095339in}{1.382212in}}%
\pgfpathlineto{\pgfqpoint{2.103097in}{1.427589in}}%
\pgfpathlineto{\pgfqpoint{2.110855in}{1.489533in}}%
\pgfpathlineto{\pgfqpoint{2.145765in}{1.801287in}}%
\pgfpathlineto{\pgfqpoint{2.153522in}{1.840046in}}%
\pgfpathlineto{\pgfqpoint{2.157401in}{1.851766in}}%
\pgfpathlineto{\pgfqpoint{2.161280in}{1.858050in}}%
\pgfpathlineto{\pgfqpoint{2.165159in}{1.858808in}}%
\pgfpathlineto{\pgfqpoint{2.169038in}{1.854076in}}%
\pgfpathlineto{\pgfqpoint{2.172917in}{1.844020in}}%
\pgfpathlineto{\pgfqpoint{2.176796in}{1.828921in}}%
\pgfpathlineto{\pgfqpoint{2.184553in}{1.785288in}}%
\pgfpathlineto{\pgfqpoint{2.192311in}{1.727506in}}%
\pgfpathlineto{\pgfqpoint{2.223342in}{1.472078in}}%
\pgfpathlineto{\pgfqpoint{2.231100in}{1.431350in}}%
\pgfpathlineto{\pgfqpoint{2.234979in}{1.417561in}}%
\pgfpathlineto{\pgfqpoint{2.238858in}{1.408554in}}%
\pgfpathlineto{\pgfqpoint{2.242736in}{1.404487in}}%
\pgfpathlineto{\pgfqpoint{2.246615in}{1.405406in}}%
\pgfpathlineto{\pgfqpoint{2.250494in}{1.411238in}}%
\pgfpathlineto{\pgfqpoint{2.254373in}{1.421801in}}%
\pgfpathlineto{\pgfqpoint{2.262131in}{1.455864in}}%
\pgfpathlineto{\pgfqpoint{2.269889in}{1.504152in}}%
\pgfpathlineto{\pgfqpoint{2.285404in}{1.623957in}}%
\pgfpathlineto{\pgfqpoint{2.297041in}{1.712185in}}%
\pgfpathlineto{\pgfqpoint{2.304799in}{1.760162in}}%
\pgfpathlineto{\pgfqpoint{2.312556in}{1.794456in}}%
\pgfpathlineto{\pgfqpoint{2.316435in}{1.805524in}}%
\pgfpathlineto{\pgfqpoint{2.320314in}{1.812223in}}%
\pgfpathlineto{\pgfqpoint{2.324193in}{1.814443in}}%
\pgfpathlineto{\pgfqpoint{2.328072in}{1.812178in}}%
\pgfpathlineto{\pgfqpoint{2.331951in}{1.805525in}}%
\pgfpathlineto{\pgfqpoint{2.335830in}{1.794679in}}%
\pgfpathlineto{\pgfqpoint{2.343587in}{1.761659in}}%
\pgfpathlineto{\pgfqpoint{2.351345in}{1.716426in}}%
\pgfpathlineto{\pgfqpoint{2.370739in}{1.579882in}}%
\pgfpathlineto{\pgfqpoint{2.382376in}{1.506971in}}%
\pgfpathlineto{\pgfqpoint{2.390134in}{1.471512in}}%
\pgfpathlineto{\pgfqpoint{2.394013in}{1.458963in}}%
\pgfpathlineto{\pgfqpoint{2.397892in}{1.450234in}}%
\pgfpathlineto{\pgfqpoint{2.401771in}{1.445486in}}%
\pgfpathlineto{\pgfqpoint{2.405649in}{1.444789in}}%
\pgfpathlineto{\pgfqpoint{2.409528in}{1.448118in}}%
\pgfpathlineto{\pgfqpoint{2.413407in}{1.455357in}}%
\pgfpathlineto{\pgfqpoint{2.417286in}{1.466301in}}%
\pgfpathlineto{\pgfqpoint{2.425044in}{1.498075in}}%
\pgfpathlineto{\pgfqpoint{2.432802in}{1.540286in}}%
\pgfpathlineto{\pgfqpoint{2.463833in}{1.727720in}}%
\pgfpathlineto{\pgfqpoint{2.471590in}{1.757788in}}%
\pgfpathlineto{\pgfqpoint{2.475469in}{1.768016in}}%
\pgfpathlineto{\pgfqpoint{2.479348in}{1.774745in}}%
\pgfpathlineto{\pgfqpoint{2.483227in}{1.777854in}}%
\pgfpathlineto{\pgfqpoint{2.487106in}{1.777308in}}%
\pgfpathlineto{\pgfqpoint{2.490985in}{1.773157in}}%
\pgfpathlineto{\pgfqpoint{2.494864in}{1.765531in}}%
\pgfpathlineto{\pgfqpoint{2.502621in}{1.740768in}}%
\pgfpathlineto{\pgfqpoint{2.510379in}{1.705524in}}%
\pgfpathlineto{\pgfqpoint{2.525895in}{1.617802in}}%
\pgfpathlineto{\pgfqpoint{2.537531in}{1.553014in}}%
\pgfpathlineto{\pgfqpoint{2.545289in}{1.517691in}}%
\pgfpathlineto{\pgfqpoint{2.553047in}{1.492354in}}%
\pgfpathlineto{\pgfqpoint{2.556926in}{1.484131in}}%
\pgfpathlineto{\pgfqpoint{2.560805in}{1.479106in}}%
\pgfpathlineto{\pgfqpoint{2.564683in}{1.477363in}}%
\pgfpathlineto{\pgfqpoint{2.568562in}{1.478909in}}%
\pgfpathlineto{\pgfqpoint{2.572441in}{1.483676in}}%
\pgfpathlineto{\pgfqpoint{2.576320in}{1.491522in}}%
\pgfpathlineto{\pgfqpoint{2.584078in}{1.515548in}}%
\pgfpathlineto{\pgfqpoint{2.591836in}{1.548577in}}%
\pgfpathlineto{\pgfqpoint{2.611230in}{1.648623in}}%
\pgfpathlineto{\pgfqpoint{2.622867in}{1.702231in}}%
\pgfpathlineto{\pgfqpoint{2.630624in}{1.728390in}}%
\pgfpathlineto{\pgfqpoint{2.638382in}{1.744184in}}%
\pgfpathlineto{\pgfqpoint{2.642261in}{1.747768in}}%
\pgfpathlineto{\pgfqpoint{2.646140in}{1.748384in}}%
\pgfpathlineto{\pgfqpoint{2.650019in}{1.746045in}}%
\pgfpathlineto{\pgfqpoint{2.653898in}{1.740837in}}%
\pgfpathlineto{\pgfqpoint{2.661655in}{1.722462in}}%
\pgfpathlineto{\pgfqpoint{2.669413in}{1.695142in}}%
\pgfpathlineto{\pgfqpoint{2.681050in}{1.643356in}}%
\pgfpathlineto{\pgfqpoint{2.700444in}{1.555543in}}%
\pgfpathlineto{\pgfqpoint{2.708202in}{1.529199in}}%
\pgfpathlineto{\pgfqpoint{2.715960in}{1.511616in}}%
\pgfpathlineto{\pgfqpoint{2.719839in}{1.506592in}}%
\pgfpathlineto{\pgfqpoint{2.723717in}{1.504219in}}%
\pgfpathlineto{\pgfqpoint{2.727596in}{1.504526in}}%
\pgfpathlineto{\pgfqpoint{2.731475in}{1.507478in}}%
\pgfpathlineto{\pgfqpoint{2.735354in}{1.512981in}}%
\pgfpathlineto{\pgfqpoint{2.743112in}{1.530982in}}%
\pgfpathlineto{\pgfqpoint{2.750870in}{1.556704in}}%
\pgfpathlineto{\pgfqpoint{2.766385in}{1.620932in}}%
\pgfpathlineto{\pgfqpoint{2.781901in}{1.682293in}}%
\pgfpathlineto{\pgfqpoint{2.789658in}{1.704897in}}%
\pgfpathlineto{\pgfqpoint{2.797416in}{1.719338in}}%
\pgfpathlineto{\pgfqpoint{2.801295in}{1.723106in}}%
\pgfpathlineto{\pgfqpoint{2.805174in}{1.724468in}}%
\pgfpathlineto{\pgfqpoint{2.809053in}{1.723419in}}%
\pgfpathlineto{\pgfqpoint{2.812932in}{1.720005in}}%
\pgfpathlineto{\pgfqpoint{2.816811in}{1.714330in}}%
\pgfpathlineto{\pgfqpoint{2.824568in}{1.696850in}}%
\pgfpathlineto{\pgfqpoint{2.832326in}{1.672734in}}%
\pgfpathlineto{\pgfqpoint{2.851720in}{1.599431in}}%
\pgfpathlineto{\pgfqpoint{2.863357in}{1.560018in}}%
\pgfpathlineto{\pgfqpoint{2.871115in}{1.540720in}}%
\pgfpathlineto{\pgfqpoint{2.878873in}{1.528999in}}%
\pgfpathlineto{\pgfqpoint{2.882752in}{1.526295in}}%
\pgfpathlineto{\pgfqpoint{2.886630in}{1.525768in}}%
\pgfpathlineto{\pgfqpoint{2.890509in}{1.527407in}}%
\pgfpathlineto{\pgfqpoint{2.894388in}{1.531153in}}%
\pgfpathlineto{\pgfqpoint{2.902146in}{1.544495in}}%
\pgfpathlineto{\pgfqpoint{2.909904in}{1.564423in}}%
\pgfpathlineto{\pgfqpoint{2.921540in}{1.602311in}}%
\pgfpathlineto{\pgfqpoint{2.940935in}{1.666772in}}%
\pgfpathlineto{\pgfqpoint{2.948692in}{1.686185in}}%
\pgfpathlineto{\pgfqpoint{2.956450in}{1.699197in}}%
\pgfpathlineto{\pgfqpoint{2.960329in}{1.702948in}}%
\pgfpathlineto{\pgfqpoint{2.964208in}{1.704756in}}%
\pgfpathlineto{\pgfqpoint{2.968087in}{1.704600in}}%
\pgfpathlineto{\pgfqpoint{2.971966in}{1.702503in}}%
\pgfpathlineto{\pgfqpoint{2.975845in}{1.698532in}}%
\pgfpathlineto{\pgfqpoint{2.983602in}{1.685449in}}%
\pgfpathlineto{\pgfqpoint{2.991360in}{1.666677in}}%
\pgfpathlineto{\pgfqpoint{3.006876in}{1.619651in}}%
\pgfpathlineto{\pgfqpoint{3.022391in}{1.574575in}}%
\pgfpathlineto{\pgfqpoint{3.030149in}{1.557911in}}%
\pgfpathlineto{\pgfqpoint{3.037907in}{1.547210in}}%
\pgfpathlineto{\pgfqpoint{3.041786in}{1.544387in}}%
\pgfpathlineto{\pgfqpoint{3.045664in}{1.543326in}}%
\pgfpathlineto{\pgfqpoint{3.049543in}{1.544034in}}%
\pgfpathlineto{\pgfqpoint{3.053422in}{1.546477in}}%
\pgfpathlineto{\pgfqpoint{3.061180in}{1.556236in}}%
\pgfpathlineto{\pgfqpoint{3.068938in}{1.571588in}}%
\pgfpathlineto{\pgfqpoint{3.080574in}{1.601709in}}%
\pgfpathlineto{\pgfqpoint{3.099969in}{1.654752in}}%
\pgfpathlineto{\pgfqpoint{3.107726in}{1.671333in}}%
\pgfpathlineto{\pgfqpoint{3.115484in}{1.682919in}}%
\pgfpathlineto{\pgfqpoint{3.119363in}{1.686520in}}%
\pgfpathlineto{\pgfqpoint{3.123242in}{1.688557in}}%
\pgfpathlineto{\pgfqpoint{3.127121in}{1.689000in}}%
\pgfpathlineto{\pgfqpoint{3.131000in}{1.687853in}}%
\pgfpathlineto{\pgfqpoint{3.134879in}{1.685160in}}%
\pgfpathlineto{\pgfqpoint{3.142636in}{1.675474in}}%
\pgfpathlineto{\pgfqpoint{3.150394in}{1.660939in}}%
\pgfpathlineto{\pgfqpoint{3.162031in}{1.633220in}}%
\pgfpathlineto{\pgfqpoint{3.181425in}{1.585902in}}%
\pgfpathlineto{\pgfqpoint{3.189183in}{1.571598in}}%
\pgfpathlineto{\pgfqpoint{3.196941in}{1.561968in}}%
\pgfpathlineto{\pgfqpoint{3.200820in}{1.559170in}}%
\pgfpathlineto{\pgfqpoint{3.204698in}{1.557794in}}%
\pgfpathlineto{\pgfqpoint{3.208577in}{1.557858in}}%
\pgfpathlineto{\pgfqpoint{3.212456in}{1.559346in}}%
\pgfpathlineto{\pgfqpoint{3.220214in}{1.566370in}}%
\pgfpathlineto{\pgfqpoint{3.227972in}{1.578120in}}%
\pgfpathlineto{\pgfqpoint{3.239608in}{1.601981in}}%
\pgfpathlineto{\pgfqpoint{3.262882in}{1.652959in}}%
\pgfpathlineto{\pgfqpoint{3.270639in}{1.665244in}}%
\pgfpathlineto{\pgfqpoint{3.278397in}{1.673172in}}%
\pgfpathlineto{\pgfqpoint{3.282276in}{1.675286in}}%
\pgfpathlineto{\pgfqpoint{3.286155in}{1.676109in}}%
\pgfpathlineto{\pgfqpoint{3.290034in}{1.675636in}}%
\pgfpathlineto{\pgfqpoint{3.293913in}{1.673889in}}%
\pgfpathlineto{\pgfqpoint{3.301670in}{1.666814in}}%
\pgfpathlineto{\pgfqpoint{3.309428in}{1.655623in}}%
\pgfpathlineto{\pgfqpoint{3.321065in}{1.633594in}}%
\pgfpathlineto{\pgfqpoint{3.340459in}{1.594669in}}%
\pgfpathlineto{\pgfqpoint{3.348217in}{1.582458in}}%
\pgfpathlineto{\pgfqpoint{3.355975in}{1.573893in}}%
\pgfpathlineto{\pgfqpoint{3.363732in}{1.569680in}}%
\pgfpathlineto{\pgfqpoint{3.367611in}{1.569314in}}%
\pgfpathlineto{\pgfqpoint{3.371490in}{1.570114in}}%
\pgfpathlineto{\pgfqpoint{3.379248in}{1.575065in}}%
\pgfpathlineto{\pgfqpoint{3.387006in}{1.583993in}}%
\pgfpathlineto{\pgfqpoint{3.398642in}{1.602823in}}%
\pgfpathlineto{\pgfqpoint{3.421916in}{1.644694in}}%
\pgfpathlineto{\pgfqpoint{3.429673in}{1.655232in}}%
\pgfpathlineto{\pgfqpoint{3.437431in}{1.662358in}}%
\pgfpathlineto{\pgfqpoint{3.445189in}{1.665491in}}%
\pgfpathlineto{\pgfqpoint{3.452947in}{1.664426in}}%
\pgfpathlineto{\pgfqpoint{3.460704in}{1.659343in}}%
\pgfpathlineto{\pgfqpoint{3.468462in}{1.650784in}}%
\pgfpathlineto{\pgfqpoint{3.480099in}{1.633340in}}%
\pgfpathlineto{\pgfqpoint{3.503372in}{1.595925in}}%
\pgfpathlineto{\pgfqpoint{3.511130in}{1.586870in}}%
\pgfpathlineto{\pgfqpoint{3.518888in}{1.580997in}}%
\pgfpathlineto{\pgfqpoint{3.526645in}{1.578778in}}%
\pgfpathlineto{\pgfqpoint{3.534403in}{1.580340in}}%
\pgfpathlineto{\pgfqpoint{3.542161in}{1.585468in}}%
\pgfpathlineto{\pgfqpoint{3.549919in}{1.593626in}}%
\pgfpathlineto{\pgfqpoint{3.561555in}{1.609736in}}%
\pgfpathlineto{\pgfqpoint{3.580950in}{1.638300in}}%
\pgfpathlineto{\pgfqpoint{3.588707in}{1.647292in}}%
\pgfpathlineto{\pgfqpoint{3.596465in}{1.653623in}}%
\pgfpathlineto{\pgfqpoint{3.604223in}{1.656770in}}%
\pgfpathlineto{\pgfqpoint{3.611981in}{1.656511in}}%
\pgfpathlineto{\pgfqpoint{3.619738in}{1.652937in}}%
\pgfpathlineto{\pgfqpoint{3.627496in}{1.646438in}}%
\pgfpathlineto{\pgfqpoint{3.639133in}{1.632678in}}%
\pgfpathlineto{\pgfqpoint{3.666285in}{1.597800in}}%
\pgfpathlineto{\pgfqpoint{3.674043in}{1.591212in}}%
\pgfpathlineto{\pgfqpoint{3.681801in}{1.587364in}}%
\pgfpathlineto{\pgfqpoint{3.689558in}{1.586551in}}%
\pgfpathlineto{\pgfqpoint{3.697316in}{1.588786in}}%
\pgfpathlineto{\pgfqpoint{3.705074in}{1.593807in}}%
\pgfpathlineto{\pgfqpoint{3.716710in}{1.605386in}}%
\pgfpathlineto{\pgfqpoint{3.747741in}{1.641018in}}%
\pgfpathlineto{\pgfqpoint{3.755499in}{1.646589in}}%
\pgfpathlineto{\pgfqpoint{3.763257in}{1.649630in}}%
\pgfpathlineto{\pgfqpoint{3.771015in}{1.649915in}}%
\pgfpathlineto{\pgfqpoint{3.778773in}{1.647473in}}%
\pgfpathlineto{\pgfqpoint{3.786530in}{1.642580in}}%
\pgfpathlineto{\pgfqpoint{3.798167in}{1.631771in}}%
\pgfpathlineto{\pgfqpoint{3.825319in}{1.603098in}}%
\pgfpathlineto{\pgfqpoint{3.833077in}{1.597387in}}%
\pgfpathlineto{\pgfqpoint{3.840835in}{1.593837in}}%
\pgfpathlineto{\pgfqpoint{3.848592in}{1.592728in}}%
\pgfpathlineto{\pgfqpoint{3.856350in}{1.594113in}}%
\pgfpathlineto{\pgfqpoint{3.864108in}{1.597816in}}%
\pgfpathlineto{\pgfqpoint{3.875744in}{1.606834in}}%
\pgfpathlineto{\pgfqpoint{3.910654in}{1.638740in}}%
\pgfpathlineto{\pgfqpoint{3.918412in}{1.642640in}}%
\pgfpathlineto{\pgfqpoint{3.926170in}{1.644402in}}%
\pgfpathlineto{\pgfqpoint{3.933928in}{1.643908in}}%
\pgfpathlineto{\pgfqpoint{3.941685in}{1.641248in}}%
\pgfpathlineto{\pgfqpoint{3.953322in}{1.633865in}}%
\pgfpathlineto{\pgfqpoint{3.972716in}{1.616840in}}%
\pgfpathlineto{\pgfqpoint{3.988232in}{1.604584in}}%
\pgfpathlineto{\pgfqpoint{3.995990in}{1.600470in}}%
\pgfpathlineto{\pgfqpoint{4.003747in}{1.598206in}}%
\pgfpathlineto{\pgfqpoint{4.011505in}{1.597961in}}%
\pgfpathlineto{\pgfqpoint{4.019263in}{1.599718in}}%
\pgfpathlineto{\pgfqpoint{4.030900in}{1.605624in}}%
\pgfpathlineto{\pgfqpoint{4.046415in}{1.617407in}}%
\pgfpathlineto{\pgfqpoint{4.065810in}{1.632211in}}%
\pgfpathlineto{\pgfqpoint{4.077446in}{1.637953in}}%
\pgfpathlineto{\pgfqpoint{4.085204in}{1.639708in}}%
\pgfpathlineto{\pgfqpoint{4.092962in}{1.639633in}}%
\pgfpathlineto{\pgfqpoint{4.100719in}{1.637772in}}%
\pgfpathlineto{\pgfqpoint{4.112356in}{1.632117in}}%
\pgfpathlineto{\pgfqpoint{4.127872in}{1.621308in}}%
\pgfpathlineto{\pgfqpoint{4.147266in}{1.608191in}}%
\pgfpathlineto{\pgfqpoint{4.158903in}{1.603346in}}%
\pgfpathlineto{\pgfqpoint{4.166660in}{1.602026in}}%
\pgfpathlineto{\pgfqpoint{4.174418in}{1.602359in}}%
\pgfpathlineto{\pgfqpoint{4.186055in}{1.605781in}}%
\pgfpathlineto{\pgfqpoint{4.197691in}{1.611954in}}%
\pgfpathlineto{\pgfqpoint{4.232601in}{1.632818in}}%
\pgfpathlineto{\pgfqpoint{4.244238in}{1.635866in}}%
\pgfpathlineto{\pgfqpoint{4.251996in}{1.636072in}}%
\pgfpathlineto{\pgfqpoint{4.263632in}{1.633667in}}%
\pgfpathlineto{\pgfqpoint{4.275269in}{1.628571in}}%
\pgfpathlineto{\pgfqpoint{4.314058in}{1.607911in}}%
\pgfpathlineto{\pgfqpoint{4.325694in}{1.605469in}}%
\pgfpathlineto{\pgfqpoint{4.337331in}{1.606015in}}%
\pgfpathlineto{\pgfqpoint{4.348968in}{1.609349in}}%
\pgfpathlineto{\pgfqpoint{4.364483in}{1.616769in}}%
\pgfpathlineto{\pgfqpoint{4.387757in}{1.628508in}}%
\pgfpathlineto{\pgfqpoint{4.399393in}{1.632087in}}%
\pgfpathlineto{\pgfqpoint{4.411030in}{1.633116in}}%
\pgfpathlineto{\pgfqpoint{4.422666in}{1.631462in}}%
\pgfpathlineto{\pgfqpoint{4.434303in}{1.627539in}}%
\pgfpathlineto{\pgfqpoint{4.476971in}{1.609541in}}%
\pgfpathlineto{\pgfqpoint{4.488607in}{1.608065in}}%
\pgfpathlineto{\pgfqpoint{4.500244in}{1.609024in}}%
\pgfpathlineto{\pgfqpoint{4.511881in}{1.612155in}}%
\pgfpathlineto{\pgfqpoint{4.531275in}{1.620178in}}%
\pgfpathlineto{\pgfqpoint{4.550669in}{1.627685in}}%
\pgfpathlineto{\pgfqpoint{4.562306in}{1.630182in}}%
\pgfpathlineto{\pgfqpoint{4.573943in}{1.630544in}}%
\pgfpathlineto{\pgfqpoint{4.585579in}{1.628761in}}%
\pgfpathlineto{\pgfqpoint{4.601095in}{1.623830in}}%
\pgfpathlineto{\pgfqpoint{4.632126in}{1.612643in}}%
\pgfpathlineto{\pgfqpoint{4.643763in}{1.610585in}}%
\pgfpathlineto{\pgfqpoint{4.655399in}{1.610473in}}%
\pgfpathlineto{\pgfqpoint{4.667036in}{1.612271in}}%
\pgfpathlineto{\pgfqpoint{4.682551in}{1.616873in}}%
\pgfpathlineto{\pgfqpoint{4.713582in}{1.626806in}}%
\pgfpathlineto{\pgfqpoint{4.725219in}{1.628483in}}%
\pgfpathlineto{\pgfqpoint{4.736856in}{1.628392in}}%
\pgfpathlineto{\pgfqpoint{4.752371in}{1.625691in}}%
\pgfpathlineto{\pgfqpoint{4.756250in}{1.624657in}}%
\pgfpathlineto{\pgfqpoint{4.756250in}{1.624657in}}%
\pgfusepath{stroke}%
\end{pgfscope}%
\begin{pgfscope}%
\pgfsetrectcap%
\pgfsetmiterjoin%
\pgfsetlinewidth{0.803000pt}%
\definecolor{currentstroke}{rgb}{0.000000,0.000000,0.000000}%
\pgfsetstrokecolor{currentstroke}%
\pgfsetdash{}{0pt}%
\pgfpathmoveto{\pgfqpoint{0.687500in}{0.373911in}}%
\pgfpathlineto{\pgfqpoint{0.687500in}{2.991285in}}%
\pgfusepath{stroke}%
\end{pgfscope}%
\begin{pgfscope}%
\pgfsetrectcap%
\pgfsetmiterjoin%
\pgfsetlinewidth{0.803000pt}%
\definecolor{currentstroke}{rgb}{0.000000,0.000000,0.000000}%
\pgfsetstrokecolor{currentstroke}%
\pgfsetdash{}{0pt}%
\pgfpathmoveto{\pgfqpoint{4.950000in}{0.373911in}}%
\pgfpathlineto{\pgfqpoint{4.950000in}{2.991285in}}%
\pgfusepath{stroke}%
\end{pgfscope}%
\begin{pgfscope}%
\pgfsetrectcap%
\pgfsetmiterjoin%
\pgfsetlinewidth{0.803000pt}%
\definecolor{currentstroke}{rgb}{0.000000,0.000000,0.000000}%
\pgfsetstrokecolor{currentstroke}%
\pgfsetdash{}{0pt}%
\pgfpathmoveto{\pgfqpoint{0.687500in}{0.373911in}}%
\pgfpathlineto{\pgfqpoint{4.950000in}{0.373911in}}%
\pgfusepath{stroke}%
\end{pgfscope}%
\begin{pgfscope}%
\pgfsetrectcap%
\pgfsetmiterjoin%
\pgfsetlinewidth{0.803000pt}%
\definecolor{currentstroke}{rgb}{0.000000,0.000000,0.000000}%
\pgfsetstrokecolor{currentstroke}%
\pgfsetdash{}{0pt}%
\pgfpathmoveto{\pgfqpoint{0.687500in}{2.991285in}}%
\pgfpathlineto{\pgfqpoint{4.950000in}{2.991285in}}%
\pgfusepath{stroke}%
\end{pgfscope}%
\begin{pgfscope}%
\pgfsetbuttcap%
\pgfsetmiterjoin%
\definecolor{currentfill}{rgb}{1.000000,1.000000,1.000000}%
\pgfsetfillcolor{currentfill}%
\pgfsetfillopacity{0.800000}%
\pgfsetlinewidth{1.003750pt}%
\definecolor{currentstroke}{rgb}{0.800000,0.800000,0.800000}%
\pgfsetstrokecolor{currentstroke}%
\pgfsetstrokeopacity{0.800000}%
\pgfsetdash{}{0pt}%
\pgfpathmoveto{\pgfqpoint{0.715278in}{3.043632in}}%
\pgfpathlineto{\pgfqpoint{4.922222in}{3.043632in}}%
\pgfpathquadraticcurveto{\pgfqpoint{4.950000in}{3.043632in}}{\pgfqpoint{4.950000in}{3.071410in}}%
\pgfpathlineto{\pgfqpoint{4.950000in}{3.253354in}}%
\pgfpathquadraticcurveto{\pgfqpoint{4.950000in}{3.281132in}}{\pgfqpoint{4.922222in}{3.281132in}}%
\pgfpathlineto{\pgfqpoint{0.715278in}{3.281132in}}%
\pgfpathquadraticcurveto{\pgfqpoint{0.687500in}{3.281132in}}{\pgfqpoint{0.687500in}{3.253354in}}%
\pgfpathlineto{\pgfqpoint{0.687500in}{3.071410in}}%
\pgfpathquadraticcurveto{\pgfqpoint{0.687500in}{3.043632in}}{\pgfqpoint{0.715278in}{3.043632in}}%
\pgfpathclose%
\pgfusepath{stroke,fill}%
\end{pgfscope}%
\begin{pgfscope}%
\pgfsetrectcap%
\pgfsetroundjoin%
\pgfsetlinewidth{1.505625pt}%
\definecolor{currentstroke}{rgb}{0.121569,0.466667,0.705882}%
\pgfsetstrokecolor{currentstroke}%
\pgfsetdash{}{0pt}%
\pgfpathmoveto{\pgfqpoint{0.743056in}{3.176271in}}%
\pgfpathlineto{\pgfqpoint{1.020833in}{3.176271in}}%
\pgfusepath{stroke}%
\end{pgfscope}%
\begin{pgfscope}%
\definecolor{textcolor}{rgb}{0.000000,0.000000,0.000000}%
\pgfsetstrokecolor{textcolor}%
\pgfsetfillcolor{textcolor}%
\pgftext[x=1.131944in,y=3.127660in,left,base]{\color{textcolor}\rmfamily\fontsize{10.000000}{12.000000}\selectfont Analítico}%
\end{pgfscope}%
\begin{pgfscope}%
\pgfsetbuttcap%
\pgfsetroundjoin%
\pgfsetlinewidth{1.505625pt}%
\definecolor{currentstroke}{rgb}{1.000000,0.498039,0.054902}%
\pgfsetstrokecolor{currentstroke}%
\pgfsetdash{{5.550000pt}{2.400000pt}}{0.000000pt}%
\pgfpathmoveto{\pgfqpoint{3.852778in}{3.176271in}}%
\pgfpathlineto{\pgfqpoint{4.130556in}{3.176271in}}%
\pgfusepath{stroke}%
\end{pgfscope}%
\begin{pgfscope}%
\definecolor{textcolor}{rgb}{0.000000,0.000000,0.000000}%
\pgfsetstrokecolor{textcolor}%
\pgfsetfillcolor{textcolor}%
\pgftext[x=4.241667in,y=3.127660in,left,base]{\color{textcolor}\rmfamily\fontsize{10.000000}{12.000000}\selectfont Chebyshev}%
\end{pgfscope}%
\begin{pgfscope}%
\pgfsetbuttcap%
\pgfsetmiterjoin%
\definecolor{currentfill}{rgb}{1.000000,1.000000,1.000000}%
\pgfsetfillcolor{currentfill}%
\pgfsetlinewidth{0.000000pt}%
\definecolor{currentstroke}{rgb}{0.000000,0.000000,0.000000}%
\pgfsetstrokecolor{currentstroke}%
\pgfsetstrokeopacity{0.000000}%
\pgfsetdash{}{0pt}%
\pgfpathmoveto{\pgfqpoint{3.025000in}{2.124492in}}%
\pgfpathlineto{\pgfqpoint{4.675000in}{2.124492in}}%
\pgfpathlineto{\pgfqpoint{4.675000in}{2.974289in}}%
\pgfpathlineto{\pgfqpoint{3.025000in}{2.974289in}}%
\pgfpathclose%
\pgfusepath{fill}%
\end{pgfscope}%
\begin{pgfscope}%
\pgfsetbuttcap%
\pgfsetroundjoin%
\definecolor{currentfill}{rgb}{0.000000,0.000000,0.000000}%
\pgfsetfillcolor{currentfill}%
\pgfsetlinewidth{0.803000pt}%
\definecolor{currentstroke}{rgb}{0.000000,0.000000,0.000000}%
\pgfsetstrokecolor{currentstroke}%
\pgfsetdash{}{0pt}%
\pgfsys@defobject{currentmarker}{\pgfqpoint{0.000000in}{-0.048611in}}{\pgfqpoint{0.000000in}{0.000000in}}{%
\pgfpathmoveto{\pgfqpoint{0.000000in}{0.000000in}}%
\pgfpathlineto{\pgfqpoint{0.000000in}{-0.048611in}}%
\pgfusepath{stroke,fill}%
}%
\begin{pgfscope}%
\pgfsys@transformshift{3.100000in}{2.124492in}%
\pgfsys@useobject{currentmarker}{}%
\end{pgfscope}%
\end{pgfscope}%
\begin{pgfscope}%
\definecolor{textcolor}{rgb}{0.000000,0.000000,0.000000}%
\pgfsetstrokecolor{textcolor}%
\pgfsetfillcolor{textcolor}%
\pgftext[x=3.100000in,y=2.027270in,,top]{\color{textcolor}\rmfamily\fontsize{10.000000}{12.000000}\selectfont \(\displaystyle 0\)}%
\end{pgfscope}%
\begin{pgfscope}%
\pgfsetbuttcap%
\pgfsetroundjoin%
\definecolor{currentfill}{rgb}{0.000000,0.000000,0.000000}%
\pgfsetfillcolor{currentfill}%
\pgfsetlinewidth{0.803000pt}%
\definecolor{currentstroke}{rgb}{0.000000,0.000000,0.000000}%
\pgfsetstrokecolor{currentstroke}%
\pgfsetdash{}{0pt}%
\pgfsys@defobject{currentmarker}{\pgfqpoint{0.000000in}{-0.048611in}}{\pgfqpoint{0.000000in}{0.000000in}}{%
\pgfpathmoveto{\pgfqpoint{0.000000in}{0.000000in}}%
\pgfpathlineto{\pgfqpoint{0.000000in}{-0.048611in}}%
\pgfusepath{stroke,fill}%
}%
\begin{pgfscope}%
\pgfsys@transformshift{3.318743in}{2.124492in}%
\pgfsys@useobject{currentmarker}{}%
\end{pgfscope}%
\end{pgfscope}%
\begin{pgfscope}%
\definecolor{textcolor}{rgb}{0.000000,0.000000,0.000000}%
\pgfsetstrokecolor{textcolor}%
\pgfsetfillcolor{textcolor}%
\pgftext[x=3.318743in,y=2.027270in,,top]{\color{textcolor}\rmfamily\fontsize{10.000000}{12.000000}\selectfont \(\displaystyle 10^{0}\)}%
\end{pgfscope}%
\begin{pgfscope}%
\pgfsetbuttcap%
\pgfsetroundjoin%
\definecolor{currentfill}{rgb}{0.000000,0.000000,0.000000}%
\pgfsetfillcolor{currentfill}%
\pgfsetlinewidth{0.803000pt}%
\definecolor{currentstroke}{rgb}{0.000000,0.000000,0.000000}%
\pgfsetstrokecolor{currentstroke}%
\pgfsetdash{}{0pt}%
\pgfsys@defobject{currentmarker}{\pgfqpoint{0.000000in}{-0.048611in}}{\pgfqpoint{0.000000in}{0.000000in}}{%
\pgfpathmoveto{\pgfqpoint{0.000000in}{0.000000in}}%
\pgfpathlineto{\pgfqpoint{0.000000in}{-0.048611in}}%
\pgfusepath{stroke,fill}%
}%
\begin{pgfscope}%
\pgfsys@transformshift{3.812696in}{2.124492in}%
\pgfsys@useobject{currentmarker}{}%
\end{pgfscope}%
\end{pgfscope}%
\begin{pgfscope}%
\definecolor{textcolor}{rgb}{0.000000,0.000000,0.000000}%
\pgfsetstrokecolor{textcolor}%
\pgfsetfillcolor{textcolor}%
\pgftext[x=3.812696in,y=2.027270in,,top]{\color{textcolor}\rmfamily\fontsize{10.000000}{12.000000}\selectfont \(\displaystyle 10^{1}\)}%
\end{pgfscope}%
\begin{pgfscope}%
\pgfsetbuttcap%
\pgfsetroundjoin%
\definecolor{currentfill}{rgb}{0.000000,0.000000,0.000000}%
\pgfsetfillcolor{currentfill}%
\pgfsetlinewidth{0.803000pt}%
\definecolor{currentstroke}{rgb}{0.000000,0.000000,0.000000}%
\pgfsetstrokecolor{currentstroke}%
\pgfsetdash{}{0pt}%
\pgfsys@defobject{currentmarker}{\pgfqpoint{0.000000in}{-0.048611in}}{\pgfqpoint{0.000000in}{0.000000in}}{%
\pgfpathmoveto{\pgfqpoint{0.000000in}{0.000000in}}%
\pgfpathlineto{\pgfqpoint{0.000000in}{-0.048611in}}%
\pgfusepath{stroke,fill}%
}%
\begin{pgfscope}%
\pgfsys@transformshift{4.206434in}{2.124492in}%
\pgfsys@useobject{currentmarker}{}%
\end{pgfscope}%
\end{pgfscope}%
\begin{pgfscope}%
\definecolor{textcolor}{rgb}{0.000000,0.000000,0.000000}%
\pgfsetstrokecolor{textcolor}%
\pgfsetfillcolor{textcolor}%
\pgftext[x=4.206434in,y=2.027270in,,top]{\color{textcolor}\rmfamily\fontsize{10.000000}{12.000000}\selectfont \(\displaystyle 10^{2}\)}%
\end{pgfscope}%
\begin{pgfscope}%
\pgfsetbuttcap%
\pgfsetroundjoin%
\definecolor{currentfill}{rgb}{0.000000,0.000000,0.000000}%
\pgfsetfillcolor{currentfill}%
\pgfsetlinewidth{0.803000pt}%
\definecolor{currentstroke}{rgb}{0.000000,0.000000,0.000000}%
\pgfsetstrokecolor{currentstroke}%
\pgfsetdash{}{0pt}%
\pgfsys@defobject{currentmarker}{\pgfqpoint{0.000000in}{-0.048611in}}{\pgfqpoint{0.000000in}{0.000000in}}{%
\pgfpathmoveto{\pgfqpoint{0.000000in}{0.000000in}}%
\pgfpathlineto{\pgfqpoint{0.000000in}{-0.048611in}}%
\pgfusepath{stroke,fill}%
}%
\begin{pgfscope}%
\pgfsys@transformshift{4.600171in}{2.124492in}%
\pgfsys@useobject{currentmarker}{}%
\end{pgfscope}%
\end{pgfscope}%
\begin{pgfscope}%
\definecolor{textcolor}{rgb}{0.000000,0.000000,0.000000}%
\pgfsetstrokecolor{textcolor}%
\pgfsetfillcolor{textcolor}%
\pgftext[x=4.600171in,y=2.027270in,,top]{\color{textcolor}\rmfamily\fontsize{10.000000}{12.000000}\selectfont \(\displaystyle 10^{3}\)}%
\end{pgfscope}%
\begin{pgfscope}%
\pgfsetbuttcap%
\pgfsetroundjoin%
\definecolor{currentfill}{rgb}{0.000000,0.000000,0.000000}%
\pgfsetfillcolor{currentfill}%
\pgfsetlinewidth{0.602250pt}%
\definecolor{currentstroke}{rgb}{0.000000,0.000000,0.000000}%
\pgfsetstrokecolor{currentstroke}%
\pgfsetdash{}{0pt}%
\pgfsys@defobject{currentmarker}{\pgfqpoint{0.000000in}{-0.027778in}}{\pgfqpoint{0.000000in}{0.000000in}}{%
\pgfpathmoveto{\pgfqpoint{0.000000in}{0.000000in}}%
\pgfpathlineto{\pgfqpoint{0.000000in}{-0.027778in}}%
\pgfusepath{stroke,fill}%
}%
\begin{pgfscope}%
\pgfsys@transformshift{3.100000in}{2.124492in}%
\pgfsys@useobject{currentmarker}{}%
\end{pgfscope}%
\end{pgfscope}%
\begin{pgfscope}%
\pgfsetbuttcap%
\pgfsetroundjoin%
\definecolor{currentfill}{rgb}{0.000000,0.000000,0.000000}%
\pgfsetfillcolor{currentfill}%
\pgfsetlinewidth{0.602250pt}%
\definecolor{currentstroke}{rgb}{0.000000,0.000000,0.000000}%
\pgfsetstrokecolor{currentstroke}%
\pgfsetdash{}{0pt}%
\pgfsys@defobject{currentmarker}{\pgfqpoint{0.000000in}{-0.027778in}}{\pgfqpoint{0.000000in}{0.000000in}}{%
\pgfpathmoveto{\pgfqpoint{0.000000in}{0.000000in}}%
\pgfpathlineto{\pgfqpoint{0.000000in}{-0.027778in}}%
\pgfusepath{stroke,fill}%
}%
\begin{pgfscope}%
\pgfsys@transformshift{3.318743in}{2.124492in}%
\pgfsys@useobject{currentmarker}{}%
\end{pgfscope}%
\end{pgfscope}%
\begin{pgfscope}%
\pgfsetbuttcap%
\pgfsetroundjoin%
\definecolor{currentfill}{rgb}{0.000000,0.000000,0.000000}%
\pgfsetfillcolor{currentfill}%
\pgfsetlinewidth{0.602250pt}%
\definecolor{currentstroke}{rgb}{0.000000,0.000000,0.000000}%
\pgfsetstrokecolor{currentstroke}%
\pgfsetdash{}{0pt}%
\pgfsys@defobject{currentmarker}{\pgfqpoint{0.000000in}{-0.027778in}}{\pgfqpoint{0.000000in}{0.000000in}}{%
\pgfpathmoveto{\pgfqpoint{0.000000in}{0.000000in}}%
\pgfpathlineto{\pgfqpoint{0.000000in}{-0.027778in}}%
\pgfusepath{stroke,fill}%
}%
\begin{pgfscope}%
\pgfsys@transformshift{3.812696in}{2.124492in}%
\pgfsys@useobject{currentmarker}{}%
\end{pgfscope}%
\end{pgfscope}%
\begin{pgfscope}%
\pgfsetbuttcap%
\pgfsetroundjoin%
\definecolor{currentfill}{rgb}{0.000000,0.000000,0.000000}%
\pgfsetfillcolor{currentfill}%
\pgfsetlinewidth{0.602250pt}%
\definecolor{currentstroke}{rgb}{0.000000,0.000000,0.000000}%
\pgfsetstrokecolor{currentstroke}%
\pgfsetdash{}{0pt}%
\pgfsys@defobject{currentmarker}{\pgfqpoint{0.000000in}{-0.027778in}}{\pgfqpoint{0.000000in}{0.000000in}}{%
\pgfpathmoveto{\pgfqpoint{0.000000in}{0.000000in}}%
\pgfpathlineto{\pgfqpoint{0.000000in}{-0.027778in}}%
\pgfusepath{stroke,fill}%
}%
\begin{pgfscope}%
\pgfsys@transformshift{4.206434in}{2.124492in}%
\pgfsys@useobject{currentmarker}{}%
\end{pgfscope}%
\end{pgfscope}%
\begin{pgfscope}%
\pgfsetbuttcap%
\pgfsetroundjoin%
\definecolor{currentfill}{rgb}{0.000000,0.000000,0.000000}%
\pgfsetfillcolor{currentfill}%
\pgfsetlinewidth{0.602250pt}%
\definecolor{currentstroke}{rgb}{0.000000,0.000000,0.000000}%
\pgfsetstrokecolor{currentstroke}%
\pgfsetdash{}{0pt}%
\pgfsys@defobject{currentmarker}{\pgfqpoint{0.000000in}{-0.027778in}}{\pgfqpoint{0.000000in}{0.000000in}}{%
\pgfpathmoveto{\pgfqpoint{0.000000in}{0.000000in}}%
\pgfpathlineto{\pgfqpoint{0.000000in}{-0.027778in}}%
\pgfusepath{stroke,fill}%
}%
\begin{pgfscope}%
\pgfsys@transformshift{4.600171in}{2.124492in}%
\pgfsys@useobject{currentmarker}{}%
\end{pgfscope}%
\end{pgfscope}%
\begin{pgfscope}%
\definecolor{textcolor}{rgb}{0.000000,0.000000,0.000000}%
\pgfsetstrokecolor{textcolor}%
\pgfsetfillcolor{textcolor}%
\pgftext[x=3.850000in,y=1.848381in,,top]{\color{textcolor}\rmfamily\fontsize{10.000000}{12.000000}\selectfont \(\displaystyle N\)}%
\end{pgfscope}%
\begin{pgfscope}%
\pgfsetbuttcap%
\pgfsetroundjoin%
\definecolor{currentfill}{rgb}{0.000000,0.000000,0.000000}%
\pgfsetfillcolor{currentfill}%
\pgfsetlinewidth{0.803000pt}%
\definecolor{currentstroke}{rgb}{0.000000,0.000000,0.000000}%
\pgfsetstrokecolor{currentstroke}%
\pgfsetdash{}{0pt}%
\pgfsys@defobject{currentmarker}{\pgfqpoint{-0.048611in}{0.000000in}}{\pgfqpoint{0.000000in}{0.000000in}}{%
\pgfpathmoveto{\pgfqpoint{0.000000in}{0.000000in}}%
\pgfpathlineto{\pgfqpoint{-0.048611in}{0.000000in}}%
\pgfusepath{stroke,fill}%
}%
\begin{pgfscope}%
\pgfsys@transformshift{3.025000in}{2.163119in}%
\pgfsys@useobject{currentmarker}{}%
\end{pgfscope}%
\end{pgfscope}%
\begin{pgfscope}%
\definecolor{textcolor}{rgb}{0.000000,0.000000,0.000000}%
\pgfsetstrokecolor{textcolor}%
\pgfsetfillcolor{textcolor}%
\pgftext[x=2.858333in,y=2.114925in,left,base]{\color{textcolor}\rmfamily\fontsize{10.000000}{12.000000}\selectfont \(\displaystyle 0\)}%
\end{pgfscope}%
\begin{pgfscope}%
\pgfsetbuttcap%
\pgfsetroundjoin%
\definecolor{currentfill}{rgb}{0.000000,0.000000,0.000000}%
\pgfsetfillcolor{currentfill}%
\pgfsetlinewidth{0.803000pt}%
\definecolor{currentstroke}{rgb}{0.000000,0.000000,0.000000}%
\pgfsetstrokecolor{currentstroke}%
\pgfsetdash{}{0pt}%
\pgfsys@defobject{currentmarker}{\pgfqpoint{-0.048611in}{0.000000in}}{\pgfqpoint{0.000000in}{0.000000in}}{%
\pgfpathmoveto{\pgfqpoint{0.000000in}{0.000000in}}%
\pgfpathlineto{\pgfqpoint{-0.048611in}{0.000000in}}%
\pgfusepath{stroke,fill}%
}%
\begin{pgfscope}%
\pgfsys@transformshift{3.025000in}{2.300881in}%
\pgfsys@useobject{currentmarker}{}%
\end{pgfscope}%
\end{pgfscope}%
\begin{pgfscope}%
\definecolor{textcolor}{rgb}{0.000000,0.000000,0.000000}%
\pgfsetstrokecolor{textcolor}%
\pgfsetfillcolor{textcolor}%
\pgftext[x=2.726581in,y=2.252686in,left,base]{\color{textcolor}\rmfamily\fontsize{10.000000}{12.000000}\selectfont \(\displaystyle 10^{0}\)}%
\end{pgfscope}%
\begin{pgfscope}%
\pgfsetbuttcap%
\pgfsetroundjoin%
\definecolor{currentfill}{rgb}{0.000000,0.000000,0.000000}%
\pgfsetfillcolor{currentfill}%
\pgfsetlinewidth{0.803000pt}%
\definecolor{currentstroke}{rgb}{0.000000,0.000000,0.000000}%
\pgfsetstrokecolor{currentstroke}%
\pgfsetdash{}{0pt}%
\pgfsys@defobject{currentmarker}{\pgfqpoint{-0.048611in}{0.000000in}}{\pgfqpoint{0.000000in}{0.000000in}}{%
\pgfpathmoveto{\pgfqpoint{0.000000in}{0.000000in}}%
\pgfpathlineto{\pgfqpoint{-0.048611in}{0.000000in}}%
\pgfusepath{stroke,fill}%
}%
\begin{pgfscope}%
\pgfsys@transformshift{3.025000in}{2.611966in}%
\pgfsys@useobject{currentmarker}{}%
\end{pgfscope}%
\end{pgfscope}%
\begin{pgfscope}%
\definecolor{textcolor}{rgb}{0.000000,0.000000,0.000000}%
\pgfsetstrokecolor{textcolor}%
\pgfsetfillcolor{textcolor}%
\pgftext[x=2.726581in,y=2.563772in,left,base]{\color{textcolor}\rmfamily\fontsize{10.000000}{12.000000}\selectfont \(\displaystyle 10^{1}\)}%
\end{pgfscope}%
\begin{pgfscope}%
\pgfsetbuttcap%
\pgfsetroundjoin%
\definecolor{currentfill}{rgb}{0.000000,0.000000,0.000000}%
\pgfsetfillcolor{currentfill}%
\pgfsetlinewidth{0.803000pt}%
\definecolor{currentstroke}{rgb}{0.000000,0.000000,0.000000}%
\pgfsetstrokecolor{currentstroke}%
\pgfsetdash{}{0pt}%
\pgfsys@defobject{currentmarker}{\pgfqpoint{-0.048611in}{0.000000in}}{\pgfqpoint{0.000000in}{0.000000in}}{%
\pgfpathmoveto{\pgfqpoint{0.000000in}{0.000000in}}%
\pgfpathlineto{\pgfqpoint{-0.048611in}{0.000000in}}%
\pgfusepath{stroke,fill}%
}%
\begin{pgfscope}%
\pgfsys@transformshift{3.025000in}{2.859937in}%
\pgfsys@useobject{currentmarker}{}%
\end{pgfscope}%
\end{pgfscope}%
\begin{pgfscope}%
\definecolor{textcolor}{rgb}{0.000000,0.000000,0.000000}%
\pgfsetstrokecolor{textcolor}%
\pgfsetfillcolor{textcolor}%
\pgftext[x=2.726581in,y=2.811743in,left,base]{\color{textcolor}\rmfamily\fontsize{10.000000}{12.000000}\selectfont \(\displaystyle 10^{2}\)}%
\end{pgfscope}%
\begin{pgfscope}%
\pgfsetbuttcap%
\pgfsetroundjoin%
\definecolor{currentfill}{rgb}{0.000000,0.000000,0.000000}%
\pgfsetfillcolor{currentfill}%
\pgfsetlinewidth{0.602250pt}%
\definecolor{currentstroke}{rgb}{0.000000,0.000000,0.000000}%
\pgfsetstrokecolor{currentstroke}%
\pgfsetdash{}{0pt}%
\pgfsys@defobject{currentmarker}{\pgfqpoint{-0.027778in}{0.000000in}}{\pgfqpoint{0.000000in}{0.000000in}}{%
\pgfpathmoveto{\pgfqpoint{0.000000in}{0.000000in}}%
\pgfpathlineto{\pgfqpoint{-0.027778in}{0.000000in}}%
\pgfusepath{stroke,fill}%
}%
\begin{pgfscope}%
\pgfsys@transformshift{3.025000in}{2.163119in}%
\pgfsys@useobject{currentmarker}{}%
\end{pgfscope}%
\end{pgfscope}%
\begin{pgfscope}%
\pgfsetbuttcap%
\pgfsetroundjoin%
\definecolor{currentfill}{rgb}{0.000000,0.000000,0.000000}%
\pgfsetfillcolor{currentfill}%
\pgfsetlinewidth{0.602250pt}%
\definecolor{currentstroke}{rgb}{0.000000,0.000000,0.000000}%
\pgfsetstrokecolor{currentstroke}%
\pgfsetdash{}{0pt}%
\pgfsys@defobject{currentmarker}{\pgfqpoint{-0.027778in}{0.000000in}}{\pgfqpoint{0.000000in}{0.000000in}}{%
\pgfpathmoveto{\pgfqpoint{0.000000in}{0.000000in}}%
\pgfpathlineto{\pgfqpoint{-0.027778in}{0.000000in}}%
\pgfusepath{stroke,fill}%
}%
\begin{pgfscope}%
\pgfsys@transformshift{3.025000in}{2.300881in}%
\pgfsys@useobject{currentmarker}{}%
\end{pgfscope}%
\end{pgfscope}%
\begin{pgfscope}%
\pgfsetbuttcap%
\pgfsetroundjoin%
\definecolor{currentfill}{rgb}{0.000000,0.000000,0.000000}%
\pgfsetfillcolor{currentfill}%
\pgfsetlinewidth{0.602250pt}%
\definecolor{currentstroke}{rgb}{0.000000,0.000000,0.000000}%
\pgfsetstrokecolor{currentstroke}%
\pgfsetdash{}{0pt}%
\pgfsys@defobject{currentmarker}{\pgfqpoint{-0.027778in}{0.000000in}}{\pgfqpoint{0.000000in}{0.000000in}}{%
\pgfpathmoveto{\pgfqpoint{0.000000in}{0.000000in}}%
\pgfpathlineto{\pgfqpoint{-0.027778in}{0.000000in}}%
\pgfusepath{stroke,fill}%
}%
\begin{pgfscope}%
\pgfsys@transformshift{3.025000in}{2.611966in}%
\pgfsys@useobject{currentmarker}{}%
\end{pgfscope}%
\end{pgfscope}%
\begin{pgfscope}%
\pgfsetbuttcap%
\pgfsetroundjoin%
\definecolor{currentfill}{rgb}{0.000000,0.000000,0.000000}%
\pgfsetfillcolor{currentfill}%
\pgfsetlinewidth{0.602250pt}%
\definecolor{currentstroke}{rgb}{0.000000,0.000000,0.000000}%
\pgfsetstrokecolor{currentstroke}%
\pgfsetdash{}{0pt}%
\pgfsys@defobject{currentmarker}{\pgfqpoint{-0.027778in}{0.000000in}}{\pgfqpoint{0.000000in}{0.000000in}}{%
\pgfpathmoveto{\pgfqpoint{0.000000in}{0.000000in}}%
\pgfpathlineto{\pgfqpoint{-0.027778in}{0.000000in}}%
\pgfusepath{stroke,fill}%
}%
\begin{pgfscope}%
\pgfsys@transformshift{3.025000in}{2.859937in}%
\pgfsys@useobject{currentmarker}{}%
\end{pgfscope}%
\end{pgfscope}%
\begin{pgfscope}%
\definecolor{textcolor}{rgb}{0.000000,0.000000,0.000000}%
\pgfsetstrokecolor{textcolor}%
\pgfsetfillcolor{textcolor}%
\pgftext[x=2.671026in,y=2.549390in,,bottom,rotate=90.000000]{\color{textcolor}\rmfamily\fontsize{10.000000}{12.000000}\selectfont \(\displaystyle \norm{f - S_{N}^{T}f}_{2}^{2}\)}%
\end{pgfscope}%
\begin{pgfscope}%
\pgfpathrectangle{\pgfqpoint{3.025000in}{2.124492in}}{\pgfqpoint{1.650000in}{0.849797in}}%
\pgfusepath{clip}%
\pgfsetrectcap%
\pgfsetroundjoin%
\pgfsetlinewidth{1.505625pt}%
\definecolor{currentstroke}{rgb}{0.121569,0.466667,0.705882}%
\pgfsetstrokecolor{currentstroke}%
\pgfsetdash{}{0pt}%
\pgfpathmoveto{\pgfqpoint{3.100000in}{2.935661in}}%
\pgfpathlineto{\pgfqpoint{3.318743in}{2.893053in}}%
\pgfpathlineto{\pgfqpoint{3.537486in}{2.893053in}}%
\pgfpathlineto{\pgfqpoint{3.606820in}{2.827365in}}%
\pgfpathlineto{\pgfqpoint{3.656013in}{2.827365in}}%
\pgfpathlineto{\pgfqpoint{3.694170in}{2.774008in}}%
\pgfpathlineto{\pgfqpoint{3.725346in}{2.774008in}}%
\pgfpathlineto{\pgfqpoint{3.751706in}{2.728477in}}%
\pgfpathlineto{\pgfqpoint{3.774539in}{2.728477in}}%
\pgfpathlineto{\pgfqpoint{3.794680in}{2.688557in}}%
\pgfpathlineto{\pgfqpoint{3.812696in}{2.688557in}}%
\pgfpathlineto{\pgfqpoint{3.828994in}{2.652870in}}%
\pgfpathlineto{\pgfqpoint{3.843873in}{2.652870in}}%
\pgfpathlineto{\pgfqpoint{3.857560in}{2.620467in}}%
\pgfpathlineto{\pgfqpoint{3.870233in}{2.620467in}}%
\pgfpathlineto{\pgfqpoint{3.882030in}{2.590657in}}%
\pgfpathlineto{\pgfqpoint{3.893066in}{2.590657in}}%
\pgfpathlineto{\pgfqpoint{3.903433in}{2.562925in}}%
\pgfpathlineto{\pgfqpoint{3.913207in}{2.562925in}}%
\pgfpathlineto{\pgfqpoint{3.922452in}{2.536877in}}%
\pgfpathlineto{\pgfqpoint{3.931223in}{2.536877in}}%
\pgfpathlineto{\pgfqpoint{3.939566in}{2.512209in}}%
\pgfpathlineto{\pgfqpoint{3.947521in}{2.512209in}}%
\pgfpathlineto{\pgfqpoint{3.955122in}{2.488683in}}%
\pgfpathlineto{\pgfqpoint{3.962400in}{2.488683in}}%
\pgfpathlineto{\pgfqpoint{3.969380in}{2.466112in}}%
\pgfpathlineto{\pgfqpoint{3.976087in}{2.466112in}}%
\pgfpathlineto{\pgfqpoint{3.982540in}{2.444344in}}%
\pgfpathlineto{\pgfqpoint{3.988759in}{2.444344in}}%
\pgfpathlineto{\pgfqpoint{3.994760in}{2.401968in}}%
\pgfpathlineto{\pgfqpoint{4.000557in}{2.401968in}}%
\pgfpathlineto{\pgfqpoint{4.006164in}{2.360568in}}%
\pgfpathlineto{\pgfqpoint{4.011593in}{2.360568in}}%
\pgfpathlineto{\pgfqpoint{4.016855in}{2.327113in}}%
\pgfpathlineto{\pgfqpoint{4.021960in}{2.327113in}}%
\pgfpathlineto{\pgfqpoint{4.026916in}{2.299885in}}%
\pgfpathlineto{\pgfqpoint{4.031734in}{2.299885in}}%
\pgfpathlineto{\pgfqpoint{4.036419in}{2.277589in}}%
\pgfpathlineto{\pgfqpoint{4.040979in}{2.277589in}}%
\pgfpathlineto{\pgfqpoint{4.045421in}{2.259230in}}%
\pgfpathlineto{\pgfqpoint{4.049750in}{2.259230in}}%
\pgfpathlineto{\pgfqpoint{4.053972in}{2.244038in}}%
\pgfpathlineto{\pgfqpoint{4.058093in}{2.244038in}}%
\pgfpathlineto{\pgfqpoint{4.062117in}{2.231415in}}%
\pgfpathlineto{\pgfqpoint{4.066048in}{2.231415in}}%
\pgfpathlineto{\pgfqpoint{4.069891in}{2.220888in}}%
\pgfpathlineto{\pgfqpoint{4.073649in}{2.220888in}}%
\pgfpathlineto{\pgfqpoint{4.077326in}{2.212084in}}%
\pgfpathlineto{\pgfqpoint{4.080927in}{2.212084in}}%
\pgfpathlineto{\pgfqpoint{4.084452in}{2.204697in}}%
\pgfpathlineto{\pgfqpoint{4.087907in}{2.204697in}}%
\pgfpathlineto{\pgfqpoint{4.091293in}{2.198481in}}%
\pgfpathlineto{\pgfqpoint{4.094614in}{2.198481in}}%
\pgfpathlineto{\pgfqpoint{4.097871in}{2.193236in}}%
\pgfpathlineto{\pgfqpoint{4.101067in}{2.193236in}}%
\pgfpathlineto{\pgfqpoint{4.104205in}{2.188804in}}%
\pgfpathlineto{\pgfqpoint{4.107286in}{2.188804in}}%
\pgfpathlineto{\pgfqpoint{4.110313in}{2.185053in}}%
\pgfpathlineto{\pgfqpoint{4.113287in}{2.185053in}}%
\pgfpathlineto{\pgfqpoint{4.116210in}{2.181871in}}%
\pgfpathlineto{\pgfqpoint{4.119084in}{2.181871in}}%
\pgfpathlineto{\pgfqpoint{4.121910in}{2.179165in}}%
\pgfpathlineto{\pgfqpoint{4.124691in}{2.179165in}}%
\pgfpathlineto{\pgfqpoint{4.127427in}{2.176863in}}%
\pgfpathlineto{\pgfqpoint{4.130120in}{2.176863in}}%
\pgfpathlineto{\pgfqpoint{4.132771in}{2.174903in}}%
\pgfpathlineto{\pgfqpoint{4.135381in}{2.174903in}}%
\pgfpathlineto{\pgfqpoint{4.137953in}{2.173231in}}%
\pgfpathlineto{\pgfqpoint{4.140486in}{2.173231in}}%
\pgfpathlineto{\pgfqpoint{4.142983in}{2.171800in}}%
\pgfpathlineto{\pgfqpoint{4.145443in}{2.171800in}}%
\pgfpathlineto{\pgfqpoint{4.150260in}{2.170578in}}%
\pgfpathlineto{\pgfqpoint{4.157241in}{2.168636in}}%
\pgfpathlineto{\pgfqpoint{4.163947in}{2.167868in}}%
\pgfpathlineto{\pgfqpoint{4.170401in}{2.166642in}}%
\pgfpathlineto{\pgfqpoint{4.180643in}{2.165739in}}%
\pgfpathlineto{\pgfqpoint{4.190307in}{2.164801in}}%
\pgfpathlineto{\pgfqpoint{4.216398in}{2.163721in}}%
\pgfpathlineto{\pgfqpoint{4.247305in}{2.163234in}}%
\pgfpathlineto{\pgfqpoint{4.600000in}{2.163119in}}%
\pgfpathlineto{\pgfqpoint{4.600000in}{2.163119in}}%
\pgfusepath{stroke}%
\end{pgfscope}%
\begin{pgfscope}%
\pgfsetrectcap%
\pgfsetmiterjoin%
\pgfsetlinewidth{0.803000pt}%
\definecolor{currentstroke}{rgb}{0.000000,0.000000,0.000000}%
\pgfsetstrokecolor{currentstroke}%
\pgfsetdash{}{0pt}%
\pgfpathmoveto{\pgfqpoint{3.025000in}{2.124492in}}%
\pgfpathlineto{\pgfqpoint{3.025000in}{2.974289in}}%
\pgfusepath{stroke}%
\end{pgfscope}%
\begin{pgfscope}%
\pgfsetrectcap%
\pgfsetmiterjoin%
\pgfsetlinewidth{0.803000pt}%
\definecolor{currentstroke}{rgb}{0.000000,0.000000,0.000000}%
\pgfsetstrokecolor{currentstroke}%
\pgfsetdash{}{0pt}%
\pgfpathmoveto{\pgfqpoint{4.675000in}{2.124492in}}%
\pgfpathlineto{\pgfqpoint{4.675000in}{2.974289in}}%
\pgfusepath{stroke}%
\end{pgfscope}%
\begin{pgfscope}%
\pgfsetrectcap%
\pgfsetmiterjoin%
\pgfsetlinewidth{0.803000pt}%
\definecolor{currentstroke}{rgb}{0.000000,0.000000,0.000000}%
\pgfsetstrokecolor{currentstroke}%
\pgfsetdash{}{0pt}%
\pgfpathmoveto{\pgfqpoint{3.025000in}{2.124492in}}%
\pgfpathlineto{\pgfqpoint{4.675000in}{2.124492in}}%
\pgfusepath{stroke}%
\end{pgfscope}%
\begin{pgfscope}%
\pgfsetrectcap%
\pgfsetmiterjoin%
\pgfsetlinewidth{0.803000pt}%
\definecolor{currentstroke}{rgb}{0.000000,0.000000,0.000000}%
\pgfsetstrokecolor{currentstroke}%
\pgfsetdash{}{0pt}%
\pgfpathmoveto{\pgfqpoint{3.025000in}{2.974289in}}%
\pgfpathlineto{\pgfqpoint{4.675000in}{2.974289in}}%
\pgfusepath{stroke}%
\end{pgfscope}%
\end{pgfpicture}%
\makeatother%
\endgroup%

	\caption{Comparación entre el cálculo analítico y la aproximación mediante polinomios de Chebyshev para el valor esperado amortiguado (por un factor $\alpha = 5\times 10^{-3}$) del operador de evolución temporal $\hat{U} = \exp(-i \hat{S}_z \tilde{t})$, $\hat{S}_z$ es la matriz de Pauli para el eje $z$ y $\tilde{t} = \frac{J_{\mathrm{ex}}t}{\hbar}$ donde $J_{\mathrm{ex}}\approxeq 100\milli\electronvolt$ es el acoplamiento de canje. La expresión analítica correspondiente al valor esperado es $2\cos{\tilde{t}}$. En el recuadro se muestra la norma Euclideana cuadrada de la diferencia entre la función correspondiente al cálculo analítico $f$ y su aproximación en serie de Chebyshev de orden $N$.}
	\label{fig:comparison}
\end{figure}

En la Fig.\ref{fig:comparison} se puede observar una demostración de la aproximación mediante los polinomios de Chebyshev para el valor esperado amortiguado del operador de evolución temporal. El amortiguamiento surge mediante el modelo de decoherencia cuántica, el cuál fue utilizado ya que otorga un enfoque más general a la demostración. En la gráfica se puede apreciar como la aproximación se sobrepone al valor analítico de la función, además de presentar que para órdenes de la aproximación mayores a $10^2$ la norma cuadrada de la diferencia entre la función analítica y la aproximación tiende drásticamente a $0$. 