\chapter[Modelo de Cadena Lineal]{Modelo de Cadena Lineal Mono-Atómica}

Como primer modelo a analizar, se utilizará un sólido cristalino mono-atómico de una sola dimensión (cadena lineal) \figref{fig:LCscheme}. 

\begin{figure}
	\centering
	\begin{tikzpicture}[atom site/.style = {circle, draw=red, thick, fill=red!50}]
	\foreach \i in {0, 1, 2}{
		\node[atom site] at ($\i*(1.75, 0)$) ["$x_\i$" below] (x-\i) {$\epsilon_0$};
	}
	\node at ($(x-2) + (0.875, 0)$) {$\cdots$};
	\node[atom site] at ($3*(1.75, 0)$) ["$x_i$" below] (x-4) {$\epsilon_0$};
	\node at ($(x-4) + (0.875, 0)$) {$\cdots$};
	\node[left=2mm of x-0] {$\cdots$};
	\foreach \i/\j in {0/1, 1/2}{
		\draw[Bar-Bar] ($(x-\i.center) - (0, .25)$) -- node[fill=white] {$a$} ($(x-\j.center) - (0, .25)$);
		\draw[Latex-Latex] 
		(x-\i.north) 
		.. controls ($(x-\i.north)!.25!(x-\j.north) + (0, .25)$) and ($(x-\i.north)!.75!(x-\j.north) + (0, .25)$) .. 
		(x-\j.north) 
		node[midway, fill=white] {$t$};
	}
	\end{tikzpicture}
	\caption[Esquema de una cadena lineal mono-atómica]{Esquema de una cadena lineal mono-atómica, con una distancia inter-atómica $a$, energía in-situ $\epsilon_0$ y una amplitud de salto entre vecinos cercanos $t$.\label{fig:LCscheme}}
\end{figure}

Dicho sistema se describe por el siguiente hamiltoniano de enlace fuerte
\begin{equation}\label{eq:LC-TBHam}
\widehat{H} = 
\epsilon_0 \sum_{i=0}^{\infty} \ket{x_i}\bra{x_i} 
- t \sum_{i=0}^{\infty} \bqty{\ket{x_i + a}\bra{x_i} + \ket{x_i}\bra{x_i + a}},
\end{equation}
De aquí en adelante se asume una energía in-situ nula ($\epsilon_0 = 0$).

A este hamiltoniano se le hace un cambio de base, del espacio real ($\ket{x_i}$) al espacio recíproco ($\ket{k_j}$) mediante las relaciones
\begin{gather}
\sum_{k} \ket{k}\bra{k} = 1, \label{eq:k-completness}\\
\bra{x}\ket{k} = e^{ixk}; \label{eq:k-eigenfunction}
\end{gather}
de modo que
\begin{align*}
\widehat{H} &= 
- t \sum_{i} \ket{x_i + a}\bra{x_i} + \ket{x_i}\bra{x_i + a},
\\\intertext{aplicando la relación de completitud \eqref{eq:k-completness}}
\widehat{H} &= 
- t \sum_{i,k,q} \bqty{\ket{k}\bra{k}\ket{x_i + a}\bra{x_i}\ket{q}\bra{q} 
	+ \ket{k}\bra{k}\ket{x_i}\bra{x_i + a}\ket{q}\bra{q}},
\\\intertext{y sustituyendo las autofunciones del número de onda \eqref{eq:k-eigenfunction}, se obtiene}
\widehat{H} &= 
- t \sum_{i,k,q} \bqty{\ket{k}e^{-ik(x_i + a)}e^{iqx_i}\bra{q} + \ket{k}e^{-ikx_i}e^{iq(x_i + a)}\bra{q}},\\
\widehat{H} &= 
- t \sum_{k,q}\ket{k}\bra{q} \pqty{e^{iqa} + e^{-ika}} \sum_{i} e^{i(q-k)x_i}.\\
\intertext{La sumatoria $\sum_{i} e^{i(a-b)c_i}$ corresponde a la delta de kroneker $\delta^a_b$}
\widehat{H} &= 
- t \sum_{k,q}\ket{k}\bra{q} \pqty{e^{iqa} + e^{-ika}} \delta^q_k,\\
\widehat{H} &= 
- t \sum_{k}\ket{k}\bra{k} \pqty{e^{ika} + e^{-ika}},\\
\widehat{H} &= \sum_{k}\ket{k}\bra{k} \pqty{-2t\cos(ka)},
\\\intertext{finalmente se establecen los autovalores $-2t\cos(ka) = \varepsilon(k)$, tal que}
\widehat{H} &= \sum_{k}\ket{k}\bra{k} \varepsilon(k). \numberthis\label{eq:LC-kHAM}
\end{align*}

\section{Conductividad a temperatura cero}
	
La conductividad a temperatura cero de este sistema se puede calcular mediante la formula de Kubo-Greenwood \autocite{Greenwood1958}
\begin{equation}\label{eq:KuboGreenwood}
	\sigma(E) = \frac{\pi \hbar e^2}{\Omega}\Trace[\delta(E - \widehat{H})\widehat{V}\delta(E - \widehat{H})\widehat{V}]
\end{equation}
donde $\Omega$ corresponde al volumen del sistema (en el caso de la cadena lineal $ \Omega = N $ el número de átomos en cadena), 
\begin{equation}\label{eq:trace}
	\Trace[\widehat{A}] = \frac{1}{N}\sum_n^N \bra{n}\widehat{A}\ket{n}
\end{equation}
corresponde a la traza (normalizada) sobre un conjunto base completo, $\delta(x)$ es la distribución de delta de Dirac y 
\begin{equation}\label{eq:VelocityOP}
	\widehat{V}_x = \frac{i}{\hbar} \comm{\widehat{H}}{\widehat{X}}
\end{equation}
es el operador velocidad en la dirección $x$; 
\begin{equation}\label{eq:comm}
	\comm{\widehat A}{\widehat B} = \widehat{A}\widehat{B} - \widehat{B}\widehat{A}
\end{equation}
es el conmutador de $\widehat{A}$ con $\widehat{B}$ y 
\begin{equation}\label{eq:positionOP}
	\widehat{X} = i \pdv{k_x}
\end{equation}
Es el operador velocidad en dirección $x$ escrito en la base del espacio recíproco.


Los elementos de matriz para el operador velocidad de la cadena lineal ($\widehat{V}_x = \widehat{V}$) quedan de la forma
\begin{align*}
	\matrixel{k}{\widehat{V}}{k'} &= \matrixel{k}{\frac{i}{\hbar} \comm{\widehat{H}}{\widehat{X}}}{k'}, \\
	\intertext{desarrollando el conmutador \eqref{eq:comm} y distribuyendo, se obtiene que}
	\matrixel{k}{\widehat{V}}{k'} &= \frac{i}{\hbar} \bqty{\matrixel{k}{\widehat{H}\widehat{X}}{k'} - \matrixel{k}{\widehat{X}\widehat{H}}{k'}}\\
	\intertext{cuya expasión dada por \eqref{eq:LC-TBHam} y \eqref{eq:positionOP} resulta en}
	\matrixel{k}{\widehat{V}}{k'} &= \frac{i}{\hbar} \bqty{\matrixel{k}{\sum_{q}\ketbra{q}\varepsilon(q)i\pdv{k}}{k'} - \matrixel{k}{i\pdv{k}\sum_{q}\ketbra{q}\varepsilon(q)}{k'}}\\
	\intertext{aplicando el ket $\bra{k}$ a la izquierda en el primer término y el ket $\ket{k'}$ en el segundo término, recordando que los vectores del número de ondas forman una base ortogonal, es decir $\braket{k}{q} = \delta^k_q$, entonces}
	\matrixel{k}{\widehat{V}}{k'} &= -\frac{1}{\hbar} \bqty{\matrixel{k}{\varepsilon(k)\pdv{k}}{k'} - \matrixel{k}{\pdv{k}\varepsilon(k')}{k'}},\\
	\intertext{el segundo término se desarrolla mediante la regla del producto,}
	\matrixel{k}{\widehat{V}}{k'} &= -\frac{1}{\hbar} \bqty{\matrixel{k}{\varepsilon(k)\pdv{k}}{k'} - \braket{k}{k'}\pdv{\varepsilon(k')}{k} - \matrixel{k}{\varepsilon(k)\pdv{k}}{k'}},\\
	\intertext{finalmente, eliminando términos iguales se obtiene}
	\matrixel{k}{\widehat{V}}{k'} &= \frac{1}{\hbar} \pdv{\varepsilon(k)}{k} \delta^{k'}_k = \frac{2t}{\hbar} a\sin(ka)\delta^{k'}_k . \numberthis\label{eq:velOPmatrixEl}
\end{align*}

Las deltas de Dirac de la fórmula de Kubo-Greenwood \eqref{eq:KuboGreenwood} se aproximan mediante el KPM \eqref{eq:kpm}
\begin{equation}\label{eq:diracDeltaKPM}
	\delta(E - \widehat{H}) \sim \frac{2}{\pi \Delta E \sqrt{1 - \widetilde{E}^2}} \sum_{m=0}^{M} T_m(\widetilde{E}) \frac{g^J_m T_m(\widetilde{H})}{1 + \delta^m_0},
\end{equation}
donde 
%\begin{equation*}\label{eq:bandWidth}
	$\Delta E = \frac{E_{\mathrm{máx}} - E_{\mathrm{mín}}}{2}$
%\end{equation*}
es el ancho de banda, 
%\begin{equation*}\label{eq:reducedEnergy}
	$\widetilde{E} = \frac{E - \bar{E}}{\Delta E}$ 
%\end{equation*}
es la energía reducida (y $\widetilde{H}$ es correspondientemente el hamiltoniano reducido), con $\bar{E} = \frac{E_{\mathrm{máx}} + E_{\mathrm{mín}}}{2}$ el centro de banda;
Se utiliza el kernel de Jackson, dado por los factores de amortiguamiento
\begin{equation}\label{eq:jacksonKernel}
	g_m^J = \frac{(M + 1 - m)\cos(\frac{\pi m}{M + 1}) + \sin(\frac{\pi m}{M + 1})\cot(\frac{\pi}{M + 1})}{M + 1},
\end{equation}
que ha demostrado ser óptimo para esta aplicación \autocite{Weise2006}.

Con todo lo anterior, la conductividad a temperatura cero se aproxima como
\begin{equation}\label{eq:KGapprox}
	\begin{aligned}
		\sigma(E) &\sim \frac{4\hbar e^2}{\pi\Omega{\Delta E}^2(1 - \widetilde{E}^2)} \sum_{m=0}^{M} \sum_{n=0}^{M} T_m(\widetilde{E})T_n(\widetilde{E}) g^J_m g^J_n \frac{\Trace[T_m(\widetilde{H})\widehat{V}T_n(\widetilde{H})\widehat{V}]}{(1 + \delta^m_0) (1 + \delta^n_0)}, \\ 
		\sigma(E) &\sim \frac{4\hbar e^2}{\pi\Omega{\Delta E}^2(1 - \widetilde{E}^2)} \sum_{m=0}^{M} \sum_{n=0}^{M} T_m(\widetilde{E})T_n(\widetilde{E}) g^J_m g^J_n \mu^m_n,
	\end{aligned}
\end{equation}
de modo que es de interés analizar la traza en la expresión anterior para el caso específico de la cadena lineal.

Para los próximos análisis se asume una amplitud de salto entre vecinos cercanos $t = -½$, dado esto, las auto-energías, dadas por la función $ E(k) = \varepsilon(E) \stackrel{t = -½}{=} \cos(ka) $, están acotadas por los valores $ E_{\mathrm{máx}} = 1 $ y $ E_{\mathrm{mín}} = -1 $, por lo tanto se tiene que $ \Delta E = 1 $, $ \bar{E} = 0 $, lo cuál implica que $ \widetilde{E} = E $ y $ \widetilde{H} = \widehat{H} $. De modo que la traza, desarrollada según \eqref{eq:trace}, queda de la forma
\begin{align*}
	\Trace[T_m(\widetilde{H})\widehat{V}T_n(\widetilde{H})\widehat{V}] &= \frac{1}{N}\sum_{k} \expval{T_m(\widehat{H})\widehat{V}T_n(\widehat{H})\widehat{V}}{k}; \\
	\intertext{para cualquier función $ f $ expandible en serie de potencias, se tiene que $f(\widehat{H}) = \sum_{k} \ketbra{k} f(\varepsilon)$. Para los polinomios de Chebyshev $ T_m(\widehat{H}) = \sum_{k} \ketbra{k} T_m(\varepsilon) = \sum_{k} \ketbra{k} \cos(mka) $, dada la expresión de los polinomios de Chebyshev \eqref{eq:ChebyshevPol}, así que}
	\Trace[T_m(\widetilde{H})\widehat{V}T_n(\widetilde{H})\widehat{V}] &= \frac{1}{N}\sum_{k} \expval{\sum_{k'}\ketbra{k'} \cos(mk'a)\widehat{V}T_n(\widehat{H})\widehat{V}}{k}, \\
	\intertext{aprovechando la relación de completitud \eqref{eq:k-completness}, entonces}
	\Trace[T_m(\widetilde{H})\widehat{V}T_n(\widetilde{H})\widehat{V}] &= \frac{1}{N}\smashoperator{\sum_{k,q,q'}} \bra{k}{\sum_{k'}\ketbra{k'} \cos(mk'a)\ket{q}\bra{q}\widehat{V}\ket{q'}\bra{q'}T_n(\widehat{H})\widehat{V}}{k}\ket{k}, \\
	\intertext{aplicando el $\bra{k}$ por la derecha y expandiendo los elementos de matriz para el operador velocidad \eqref{eq:velOPmatrixEl} se obtiene}
	\Trace[T_m(\widetilde{H})\widehat{V}T_n(\widetilde{H})\widehat{V}] &= -\frac{a}{\hbar N} \sum_{k,q} \bra{k} \ket{q}\cos(mka)\sin(qa)\bra{q}T_n(\widehat{H})\widehat{V}\ket{k}, \\
	\Trace[T_m(\widetilde{H})\widehat{V}T_n(\widetilde{H})\widehat{V}] &= -\frac{a}{\hbar N} \sum_{k} \cos(mka)\sin(ka)\bra{k}T_n(\widehat{H})\widehat{V}\ket{k}, \\
	\intertext{aplicando el mismo procedimiento para el elemento de matriz faltante, da}
	\Trace[T_m(\widetilde{H})\widehat{V}T_n(\widetilde{H})\widehat{V}] &= \pqty{\frac{a}{\hbar}}^2 \frac{1}{N} \sum_{k} \sin[2](ka)\cos(mka)\cos(nka). \\
	\intertext{Adicionalmente se le establecen condiciones de frontera períodica a la cadena lineal, de modo que las auto-energías se cuantizan a aquellas correspodientes a los valores del vector de onda $ak_n = \frac{2\pi n}{N}$, con $ N $ el número de átomos en la cadena y el índice entero $ n $ va de 0 a $N$. Con esto se puede transformar la sumatoria en una integral sobre el espacio-k mediante la definición de la suma de Riemman}
	\Trace[T_m(\widetilde{H})\widehat{V}T_n(\widetilde{H})\widehat{V}] &= \pqty{\frac{a}{\hbar}}^2 \frac{N}{2\pi N} \smashoperator{\int_{0}^{2\pi}} \sin[2](ka)\cos(mka)\cos(nka)d(ka).\label{eq:chebMomTraceInt}\numberthis \\ 
	\intertext{Resolviendo la integral \apdref{ap:chebMomTraceInt},}
	\Trace[T_m(\widehat{H})\widehat{V}T_n(\widehat{H})\widehat{V}] &= \pqty{\frac{a}{\hbar}}^2 \frac{1}{2\pi} \frac{\pi}{2} \bqty{\delta^m_n(1 + \delta^m_0)(1-\tfrac{1}{2}\delta^m_1) - \tfrac{1}{2}(1 - \delta^m_n)(\delta^{m+n}_2 + \delta^{\abs{m-n}}_2)},
\end{align*}
con esto se puede obtener la cantidad $ \mu^m_n $, presentada en la expresión \eqref{eq:KGapprox},
\begin{align*}
	\mu^m_n &= \pqty{\frac{a}{2\hbar}}^2 \bqty{\delta^m_n\frac{(1 + \delta^m_0)(1-\tfrac{1}{2}\delta^m_1)}{(1 + \delta^m_0)^2} - \tfrac{1}{2}(1 - \delta^m_n)\frac{\delta^{m+n}_2 + \delta^{\abs{m-n}}_2}{(1 + \delta^m_0) (1 + \delta^n_0)}},\\
	\mu^m_n &= \pqty{\frac{a}{2\hbar}}^2 \bqty{\delta^m_n\frac{(1-\tfrac{1}{2}\delta^m_1)}{(1 + \delta^m_0)}\frac{1 + \delta^m_1}{1 + \delta^m_1} - \tfrac{1}{2}(1 - \delta^m_n)\frac{\delta^{m+n}_2 + \delta^{\abs{m-n}}_2}{(1 + \delta^m_0) (1 + \delta^n_0)} \frac{1 - \tfrac{1}{2}\delta^m_0 - \tfrac{1}{2}\delta^n_0}{1 - \tfrac{1}{2}\delta^m_0 - \tfrac{1}{2}\delta^n_0}},\\
	\intertext{Para $ m \neq n $ la ecuación $ m + n = 2 $ solo tiene como resultados posibles los valores $ (m, n) = \{(0, 2), (2, 0)\} $ por lo cual $ \delta^{m+n}_2 = \delta^m_0\delta^n_2 + \delta^m_2\delta^n_0 $,}
	\mu^m_n &= \pqty{\frac{a}{2\hbar}}^2 \bqty{\frac{\delta^m_n}{(1 + \delta^m_0 + \delta^m_1)} - \tfrac{1}{2}(1 - \delta^m_n)(\delta^m_0\delta^n_2 + \delta^m_2\delta^n_0 + \delta^{\abs{m-n}}_2)(1 - \tfrac{1}{2}\delta^m_0 - \tfrac{1}{2}\delta^n_0)},\\
	\mu^m_n &= \pqty{\frac{a}{2\hbar}}^2 \bqty{\frac{\delta^m_n}{(1 + \delta^m_0 + \delta^m_1)} - \tfrac{1}{2}(1 - \delta^m_n)\delta^{\abs{m-n}}_2}.\numberthis\\
\end{align*}

Sustituyendo esta expresión en la aproximación para la conductividad \eqref{eq:KGapprox}
\begin{align*}
	\sigma(E) &\sim \frac{4\hbar e^2}{N\pi(1 - E^2)} \pqty{\frac{a}{2\hbar}}^2 \sum_{m=0}^{M} \sum_{n=0}^{M} T_m(E)T_n(E)g_m g_n \bqty{\frac{\delta^m_n}{(1 + \delta^m_0 + \delta^m_1)} - \tfrac{1}{2}(1 - \delta^m_n)\delta^{\abs{m-n}}_2},\\
	\sigma(E) &\sim \frac{a^2e^2}{Nh(1 - E^2)} \sum_{m=0}^{M} \sum_{n=0}^{M} T_m(E)T_n(E)g_m g_n \bqty{\frac{\delta^m_n}{(1 + \delta^m_0 + \delta^m_1)} - \tfrac{1}{2}(1 - \delta^m_n)\delta^{\abs{m-n}}_2}.\numberthis
\end{align*}

	
\section{Zeeman splitting and spin dynamics}

Un tipo de sistema interesante son los ferromagnétos, donde el espín de sus electrones itinerantes interactúan con los momentos  magnéticos localizados típicamente en los orbitales $d$. Para describir esta situación, vamos a suponer que el sistema es ferromagnético y tiene una magnetización homogénea de $\textbf{M}=M_s\hat{u}_{\textbf{M}}$, donde $M_s$ su valor de saturación y \[\hat{u}_{\textbf{M}}=(\sin\theta\cos\phi,\sin\theta\sin\phi,\cos\theta)\] su dirección definida por los ángulos $\phi$ y $\theta$. Luego, asumiremos que la magnetización se acopla con los electrones por medio de una interacción de intercambio cuantificada por el parámetro $J_{\rm Ex }$. Vamos a suponer ademas que el sistema puede ser modelado por una cadena lineal de primeros vecinos, en este caso, el Hamiltoniano es dado por:
\begin{equation}{\label{eq:lc_HAM_Zeeman}}
	\widehat{H} = \widehat{H}_0+\widehat{H}_{ex},
\end{equation}
donde
\begin{equation}
	\widehat{H}_0 =- t  \sum_{i,s} \bqty{\ketbra{x_i + a,s}{x_i,s} + \ketbra{x_i,s}{x_i + a,s}},
\end{equation}
representa la componente orbital, con $t$ el hopping de primeros vecinos. Luego, 
\begin{equation}
	\widehat{H}_{\rm ex} =-J_{\rm ex }  \sum_{i,s,s'} (\hat{u}_{\textbf{M}}\cdot \vector{\sigma})_{s,s'}\ket{x_i,s}\bra{x_i,s'} 
\end{equation}
representa a la interacción de intercambio, donde $ \vector{\sigma}=\sigma_x \hat{\vector x} + \sigma_y \hat{\vector y} + \sigma_z \hat{\vector z} $ es el vector de Pauli y 
\begin{align*}
	\sigma_0 = \mathbbm1_{2\times2}&= \bmqty{\pmat{0}}, & \sigma_x = \sigma_1 &= \bmqty{\pmat{x}}, & \sigma_y = \sigma_2 &= \bmqty{\pmat{y}}, & \sigma_z = \sigma_3 &= \bmqty{\pmat{z}},
\end{align*}
son operadores conocidos como matrices de Pauli que describen la polarización del espín. $s=\pm 1$ identifica a los autovalores de la matriz de Pauli $\sigma_z$, 

El elemento  $\widehat{H}_0$ es diagonal en el espacio de espín, por lo tanto, su representación en el espacio es idéntica a la realizada anteriormente \eqref{eq:LC-kHAM}.
La representación del Hamiltoniano en el espacio $k$ sería
\begin{align*}
	\widehat{H}_0 = \sum_{k, s} \ketbra{k, s} \varepsilon(k) ,
\end{align*}
con $\varepsilon(k)= -2t \cos(ka) $ la relación de dispersión de la cadena lineal.

Por otro lado, el término $\widehat{H}_{\rm ex}$ es diagonal proporcional a la identidad en el espacio de las posiciones por lo que es invariante ante una transformada de Fourier, permaneciendo idéntico en la nueva base
\begin{align*}
	\widehat{H}_{\rm ex}  = -J_{\rm ex } \sum_{k, s} \ket{k,s}\bra{k, s'} (\hat{u}_{\textbf{M}}\cdot \vector{\sigma})_{s,s'} ,
\end{align*}

Nuestra intención es determinar la 
evolución macroscópica en el tiempo de la $i$-ésima componente del vector de spin, una vez que se inyecta una distribución de electrones fuera del equilibrio polarizada en la dirección $\hat{u}_{q}=(\sin\gamma\cos\varphi,\sin\gamma\sin\varphi,\cos\gamma)$ a un nivel de Fermi $E_{\rm F}$ dado. 

Para esto, determinaremos el valor esperado en la representación de Heisenberg:
\begin{equation}
	\langle \hat{S}_i(E_{\rm{F}},\gamma,\varphi,t) \rangle = \operatorname{Tr}\left[\widehat{U}^\dagger(t)\hat{S}_i \widehat{U}(t)  \hat{\rho}(E_{\rm{F}},\gamma,\varphi) \right],
\end{equation}
donde 
\begin{equation}\label{eq:spinOP}
	\widehat{S}_i(E_{\rm F}) = \frac{\hbar}{2}\sigma_i,
\end{equation}
el operador de espín,  $\widehat{U}(t) = \exp(\frac{i\widehat{H}t}{\hbar}) $ el operador de evolución temporal, y $ \hat{\rho}(E_{\rm{F}})$ la matriz densidad.  La ventaja de utilizar la representación de Heisenberg por sobre la Schrödinger proviene del hecho de que la primera permite de extender los resultados a muchos cuerpos de forma directa, mientras que la segunda el proceso es mucho mas complejo. Para nuestro caso consideraremos la siguiente matriz densidad
\begin{equation}
	\hat{\rho}{(E_{\rm{F}},\gamma,\varphi) = P(\gamma,\varphi)\delta(\widehat{H}- E_{\rm F} )P(\gamma,\varphi)},
\end{equation}
donde $P(\gamma,\varphi)=\ket{s,\gamma,\varphi} \bra{s,\gamma,\varphi}$ el operador proyección de spin con autovalor $s=\pm 1$ en la dirección $\hat{u}_{q}$. Esta matriz densidad   se utiliza para representar el hecho de que los únicos electrones móviles son los que viven en el nivel de Fermi, y a estos se les está dotando con una polarización inicial.  Como $\widehat{H}_{\rm ex}$ conmuta con $\widehat{H}_{0}$, la fórmula  de Baker-Campbell-Hausdorff nos indica que podemos factorizar el operador evolución
\begin{equation}%\label{eq:tempEvOP}
	\widehat{U}(t) = \widehat{U}_0(t)\exp(\frac{i\widehat{H}_{\rm ex}t}{\hbar}), 
\end{equation}
donde hemos definido $\widehat{U}_0(t)\equiv\exp(\frac{i\widehat{H}_{0}t}{\hbar})$. Con lo que, si escogemos la base de los momentos y $\sigma_z$ para calcular la traza nos permite escribir:
\begin{multline}
	\langle \hat{S}_i(E_{\rm{F}},\gamma,\varphi,t) \rangle = \sum_{k, s} \bra{k, s}\exp(-\frac{i\widehat{H}_{\rm ex}t}{\hbar})\widehat{U}^\dagger_0(t)\hat{S}_i \widehat{U}_0(t)\exp(\frac{i\widehat{H}_{\rm ex}t}{\hbar}) \\ P(\gamma,\varphi)\delta(\widehat{H} - E)P(\gamma,\varphi)\ket{k, s},
\end{multline}
donde si usamos la unitariedad de $\widehat{U}_0(t)$ y el hecho de que conmuta con el operador de espín \apdref{ap:commU0S}, obtenemos
\begin{equation}
	\langle \hat{S}_i(E_{\rm{F}},t) \rangle = \sum_{k, s} \bra{ s}\exp(-\frac{i\widehat{H}_{\rm ex}t}{\hbar})\hat{S}_i \exp(\frac{i\widehat{H}_{\rm ex}t}{\hbar}) P(\gamma,\varphi)\delta\left (\varepsilon(k) -J_{\rm ex} \hat{u}_{\textbf{M}}\cdot \vector{\sigma}  -E\right)P(\gamma,\varphi) \ket{s}.
\end{equation}
donde ademas usamos el hecho de que la delta de Dirac es diagonal en el espacio de momentos.

Ahora bien, si definimos $\Theta \equiv -J_{\rm ex} t /\hbar$, y $\sigma_u = \hat{u}_{\textbf{M}}\cdot \vector{\sigma}$, podemos demostrar entonces que \apdref{ap:spinRotOP}
\begin{equation}
	\exp(\frac{i\widehat{H}_{\rm ex}t}{\hbar}) = R_u(\Theta)  
\end{equation}
donde $R_u(\Theta)= \exp(i\Theta s_u )$ no es más que el operador de rotación en la dirección definida por la magnetización cuyo ángulo de rotación depende del tiempo, el cual se puede expresar en términos de las matrices de Pauli como:
\begin{equation}
	R_u(\Theta) = \cos \Theta  +i \sigma_{u}\sin\Theta . 
\end{equation}
Por lo que vemos entonces que el papel de la magnetización es hacer el espín precesar. 

\subsection{Magnetización en $ \hat{\vector z} $}

En este caso tenemos
\begin{equation}\label{eq:S_EV:magZpre}
	\expval{\hat{S}_i(E_{\rm{F}},t)} = \sum_{k, s} \exp(\frac{i J_{\rm ex }s t}{\hbar}) \bra{ s}\hat{S}_i \exp(\frac{i\widehat{H}_{\rm ex}t}{\hbar})P(\gamma,\varphi) \delta\left (\varepsilon(k) -J_{\rm ex} \sigma_z  -E\right)P(\gamma,\varphi)\ket{s}.
\end{equation}

\subsubsection{Inyección en $ \hat{\vector z} $}
Si la inyección ocurre en $z$ se tiene que $P(0,0)=\ket{+}\bra{+}$, por lo que el vector espín será \apdref{ap:S_EV:magZ}
\begin{equation}
	\langle \hat{\vector{S} }(E_{\rm{F}},t) \rangle = \sum_k\delta\left (\varepsilon(k) -J_{\rm ex} -E\right) \hat{\vector z}.
\end{equation}

\subsubsection{Inyección en $ \hat{\vector x} $}
Si la inyección ocurre en $x$ se tiene que $P(\frac{\pi}{2},0) = \ketbra{+, \frac{\pi}{2}, 0}$, por lo que el vector espín será
\begin{equation}
	\expval{\vector{\hat{S}}(E_{\rm{F}},t)} = \frac{1}{2} \left[ \cos(\frac{2 J_{\rm ex } t}{\hbar}) \hat{\vector x} + \cos(\frac{2 J_{\rm ex } t}{\hbar}) \hat{\vector y} \right]  \smashoperator{\sum_{k, s}} \delta\left(\varepsilon(k) - sJ_{\rm ex} -E \right).
\end{equation}