\chapter{Antecedentes}

% \blockquote[\fullcite{Makarovsky2017}]{(\dots) Este modelo de correlación de carga 
% es compatible con las simulaciones de Monte Carlo de transporte de electrones y es 
% utilizado para explicar el inesperado aumento de la movilidad en tres veces al 
% aumentar la densidad electrónica. La mayor concentración de portadores y movilidad 
% dan lugar a oscilaciones de Shubnikov-de Haas en la magnetorresistencia, que 
% proporcionan una estimación de la masa ciclotrón del electrón en el grafeno a 
% altas densidades y energías de Fermi de hasta 
% $  \unit{1,2\times10^{13}}{\centi\meter\rpsquared} $ 
% y \unit{400}{\milli\electronvolt}, respectivamente.}
	
% \blockquote[\fullcite{Benitez2017}]{Se ha pronosticado una gran mejora en el 
% acoplamiento espín-órbita del grafeno al interconectarlo con de metales de 
% transición semiconductores dicalcogenuros. Se han informado las marcas de dicha 
% mejora, pero la naturaleza de la relajación del espín en estos sistemas sigue 
% siendo desconocida (\dots) Estos hallazgos proporcionan una plataforma rica para 
% explorar fenómenos acoplados valle-espín y ofrecen estrategias novedosas de 
% manipulación del espín basadas en la anisotropía de relajación del espín en 
% materiales bidimensionales.}

\blockquote[\fullcite{Rocha2018}]{``Los cálculos de estructura electrónica moderna 
	se implementan predominantemente dentro de la representación de super celda en 
	la que las celdas unitarias se organizan periódicamente en el espacio. Incluso 
	en el caso de materiales no cristalinos, las celdas unitarias con defectos 
	incrustados se utilizan comúnmente para describir estructuras dopadas. Sin 
	embargo, este tipo de computo se vuelve prohibitivamente exigente cuando las 
	tasas de convergencia son lo suficientemente lentas y pueden requerir cálculos 
	con celdas unitarias muy grandes. Aquí mostramos que una característica hasta 
	ahora inexplorada mostrada por varios materiales 2D puede usarse para lograr la 
	convergencia en los cálculos de energía de formación y adsorción con tamaños de 
	celda unitaria relativamente pequeños. La generalidad de nuestro método se 
	ilustra con los cálculos de Teoría Funcional de la Densidad para diferentes 
	hospedadores 2D dopados con diferentes impurezas, todos los cuales proporcionan 
	niveles de precisión que de otro modo requerirían celdas unitarias enormemente 
	grandes. Este enfoque proporciona una ruta eficiente para calcular las 
	propiedades físicas de los sistemas 2D en general, pero es particularmente 
	adecuado para los materiales de punto de Dirac dopados con impurezas que rompen 
	su simetría de subred.''}

\blockquote[\fullcite{Garcia2018}]{``Desde su descubrimiento, el grafeno ha sido un 
	material prometedor para la espintrónica: su bajo acoplamiento espín-órbita, su 
	interacción hiperfina despreciable y su alta movilidad de electrones son 
	ventajas obvias para el transporte de información de espín a largas distancias. 
	Sin embargo, tales propiedades de transporte excepcionales también limitan la 
	capacidad de diseñar ingeniería espintrónica activa, donde un fuerte 
	acoplamiento de espín-órbita es crucial para crear y manipular corrientes de 
	espín.
	% Para este fin, los dicalcogenuros de metales de transición, que tienen un 
	% acoplamiento de espín-órbita más grande y una buena correspondencia de 
	% interfaz, parecen ser materiales altamente complementarios para mejorar las 
	% características del grafeno dependientes del espín, al tiempo que mantienen 
	% sus propiedades superiores de transporte de carga.
	(\dots) En esta revisión, presentamos el marco teórico y los experimentos 
	realizados para detectar y caracterizar el acoplamiento espín-órbita y las 
	corrientes de espín en las heteroestructuras de grafeno/dicalcogenuro de metales 
	de transición. Específicamente, nos concentraremos en las mediciones recientes 
	de la precesión de Hanle, la antilocalización débil y el efecto Hall de espín, y 
	proporcionaremos una descripción teórica completa de la interconexión entre 
	estos fenómenos.}
	
% \blckquote[\fullcite{Cummings2017}]{Informamos sobre aspectos fundamentales de la 
% 	dinámica de espín en heteroestructuras de grafeno y metales de transición 
% 	dicalcogenuros (TMDC). Mediante el uso de modelos realistas derivados de los 
% 	primeros principios, calculamos la anisotropía de la vida útil de los espines 
% 	(\dots) Encontramos que la anisotropía puede alcanzar valores de decenas a 
% 	cientos, lo que no tiene precedentes en los sistemas 2D típicos con acoplamiento 
% 	de órbita de espín e indica un régimen cualitativamente nuevo de relajación de 
% 	espín (\dots) los materiales con gigantes anisotropías de tiempo de vida de 
% 	espín pueden proporcionar una plataforma emocionante para manipular los grados 
% 	de libertad de los valles y espín, y para diseñar nuevos dispositivos 
% 	espintrónicos.}

\blockquote[\fullcite{Ziglmann2018}]{``(\dots) Las mediciones dependientes del 
	dopaje muestran que la relajación de espín de los espines en el plano está 
	dominada en gran medida por un acoplamiento valle-Zeeman de espín-órbita y que 
	el acoplamiento intrínseco espín-órbita juega un papel menor en la relajación 
	del espín. El fuerte acoplamiento espín-valle abre nuevas posibilidades para 
	explorar el grado de libertad de espín y valle en grafeno con la realización de 
	nuevos conceptos en la manipulación de espín.''}

\blockquote[\fullcite{Boross2015}]{``El grafeno mono-capa es un ejemplo de material 
	con estructura electrónica multi-valle. En tales materiales, el índice de valle 
	está siendo considerado como un portador de información. En consecuencia, los 
	mecanismos de relajación que conducen a la pérdida de información del valle son 
	de interés. Aquí, calculamos la tasa de relajación del valle inducida por 
	impurezas cargadas en el grafeno (\dots) Obtenemos la tasa de relajación de 
	valle resolviendo la ecuación de Boltzmann, para el caso de electrones no 
	interactuantes, así como para el caso en el que se analiza el potencial de 
	impureza debido a la interacción electrón-electrón (\dots) Nuestros principales 
	hallazgos son los siguientes: (i) La tasa de relajación del valle es 
	proporcional a la densidad electrónica de los estados en la energía de Fermi. 
	(ii) Las impurezas cargadas ubicadas en las proximidades del plano del grafeno, 
	a una distancia $ d \leq \unit{0,3}{\angstrom} $, son mucho más eficientes para 
	inducir la relajación de valle que las más alejadas, el efecto de este último se 
	suprime exponencialmente al aumentar la distancia grafeno-impureza $ d $. (iii) 
	Tanto en ausencia como en presencia de interacción electrón-electrón, la tasa de 
	relajación de valle muestra una pronunciada dependencia del radio de Bohr 
	efectivo $ a_{eB} $. Las tendencias son diferentes en los dos casos: en ausencia 
	(presencia) de apantallamiento, la tasa de relajación del valle disminuye 
	(aumenta) para aumentos de radio efectivo de Bohr. Este último resultado destaca 
	que un cálculo cuantitativo de la tasa de relajación de valle debe incorporar 
	interacciones electrón-electrón, así como un conocimiento preciso de las 
	funciones de onda electrónica en la escala de longitud atómica.''}
	
% \blockquote[\fullcite{DasSarma2011}]{Se proporciona una amplia revisión de las 
% 	propiedades electrónicas fundamentales del grafeno bidimensional con énfasis en 
% 	el transporte de portadores dependiente de la densidad y la temperatura en 
% 	estructuras de grafeno dopadas o cerradas. Una característica destacada de esta 
% 	revisión es una comparación crítica entre el transporte de portadores en el 
% 	grafeno y en los sistemas semiconductores bidimensionales (p. Ej., 
% 	Heteroestructuras, pozos cuánticos, capas de inversión), de modo que las 
% 	características únicas de las propiedades electrónicas del grafeno que surgen de 
% 	su espectro de Dirac sin brecha, sin masa, quiral (\dots)}