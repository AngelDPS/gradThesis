\chapter{Resumen}
\markboth{RESUMEN}{}

Investigamos la aplicación de aproximaciones numéricas mediante expansiones de Chebyshev y el KPM para estudiar las dinámicas de valle presentes en grafeno sometido a campos magnéticos externos. El uso de técnicas de Transporte Cuántico de Escalamiento Lineal (LSQT por sus siglas en inglés) permite la aproximación realista de los operadores de evolución temporal y de proyección cuántica; validamos los algoritmos resultantes por comparación con un sistema de cadena lineal bajo condiciones de borde periódicas, el cual debido a su simplicidad nos permite aproximar sus propiedades con términos exactos. De interés específico es el estudio del tiempo de relajación en el grafeno, el cuál es de importancia en la caracterización de la valletrónica, un área emergente, similar a la espintrónica, que se centra en la transferencia de información mediante valles de Dirac así como la manipulación de sus grados de libertad.

%\vspace*{\baselineskip}\hrule\vspace*{\baselineskip}